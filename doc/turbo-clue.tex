% -*- Mode:TeX -*-
%%%----------------------------------------------------------------------------------+
%%%                                                                                  |
%%%                          TEXAS INSTRUMENTS INCORPORATED                          |
%%%                                  P.O. BOX 149149                                 |
%%%                             AUSTIN, TEXAS 78714-9149                             |
%%%                                                                                  |
%%%                Copyright (C) 1989 Texas Instruments Incorporated.                |
%%%                                                                                  |
%%% Permission is granted to any individual or institution to use, copy, modify, and |
%%% distribute this document, provided that  this complete copyright and  permission |
%%% notice is maintained, intact, in all copies.                                     |
%%%                                                                                  |
%%% Texas Instruments Incorporated provides this document "as is" without express or |
%%% implied warranty.                                                                |
%%%                                                                                  |
%%%----------------------------------------------------------------------------------+
\documentstyle[twoside,12pt]{article} 
\pagestyle{plain} 
%%
%% Inserted from home:/usr/local/hacks/tex/setmargins.tex:
%%
%%    Original: Glenn Manuel 12-17-86
%%    Added optional [printer-offset]: Glenn Manuel 2-27-87
%% Sets the margins, taking into account the fact that LaTex
%% likes to start the left margin 1.0 inch from the left edge
%% of the paper.
%%   This macro prints on the screen and in the log file
%%   the values it sets for:
%%     Textwidth, Odd Page Left Margin, Even Page Left Margin,
%%     Marginparwidth, Printer Offset
%%     (the values are in points: 72.27 pts/inch).
%%     The Odd and Even Page Left Margins include LaTeX's
%%     built-in 1.0in offset, but NOT the [printer-offset],
%%     so these values always indicate what the actual
%%     printout SHOULD measure.
%%
%% USAGE:  Place the following between the
%%         \documentstyle   and  \begin{document} commands:
%% \input{this-file's-name}
%% \setmargins[printer-offset]{line-length}{inside-margin-width}
%%
%%   where ALL arguments to \setmargins are DIMENSIONS
%%           like {6.5in}, {65pt}, {23cm}, etc.
%%   The units MUST BE SUPPLIED, even if the dimension is zero {0in}.
%%
%%   [printer-offset] is optional.  Default is zero.
%%
%%   Examples:
%% \setmargins{0in}{0in}         % default line length & default margins
%% \setmargins[-.12in]{0in}{0in} % compensate for printer offset
%% \setmargins{0in}{1in}         % default line length, 1 inch inner margin
%% \setmargins{6.5in}{0in}       % 6.5 inch line length & default margins
%% \setmargins{5in}{1.5in}       % 5 inch line length & 1.5 inch inner margin
%%
%%   {inside-margin-width} is defined as follows:
%%       For 1-sided printing: left margin for all pages.
%%       For 2-sided printing: left  margin for odd pages,
%%                             right margin for even pages.
%%
%%   Defaults:
%%   Each argument has a default if {0in} is used as the argument:
%%      line-length default = 6.0in
%%      inside-margin-width default:
%%         For 1-sided printing, text is centered on the page
%%                     (each margin = [1/2]*[8.5in - line length]);
%%         For 2-sided printing, inside margin is twice the outside margin
%%                     (inside  margin = [2/3]*[8.5in - line length],
%%                      outside margin = [1/3]*[8.5in - line length]).
%%      printer-offset default = 0in
%%
%%  For all cases, the outside margin (and marginparwidth, the
%%  width of margin notes) is just whatever is left over after
%%  accounting for the inside margin and the line length.
%%
%% Note: LaTeX's built-in offset of 1.0 inch can vary somewhat,
%%       depending upon the alignment of the Laser printer.
%%       If you need it to be EXACT, you will have to supply
%%       the optional [printer-offset] argument.
%%       Subtract the actual measured left margin on an
%%       odd-numbered page from the printed Odd Page Left Margin
%%       value, and use the result as the [printer-offset].
%%       Positive values shift everything to the right,
%%       negative values shift everything to the left.
%%
\makeatletter
\def\setmargins{\@ifnextchar[{\@setmargins}{\@setmargins[0in]}}
\def\@setmargins[#1]#2#3{
%%%  Uses temporary dimension registers \dimen0, \dimen2, \dimen3, \dimen1
    \dimen1=#1                         % 1st argument [printer offset]
    \dimen2=#2                         % 2nd argument (line length)
    \dimen3=#3                         % 3rd argument (inner margin)
     \advance\dimen1 by -1.0in         % for LaTeX built-in offset
    \ifdim\dimen2=0in 
        \textwidth=6in  \dimen2=6in
    \else \textwidth=\dimen2
    \fi
    \dimen0=8.5in
    \advance\dimen0 by -\dimen2         % 8.5in - line length
    \if@twoside
       \ifdim\dimen3=0in  % use defaults: 2/3 inside, 1/3 outside
          \divide\dimen0 by 3           % (8.5in-line length)/3
          \dimen2=2\dimen0              % (2/3)*(8.5in-line length)
          \oddsidemargin=\dimen2
          \advance\oddsidemargin by \dimen1   % add in offset
          \dimen2=\dimen0               % (8.5in-line length)/3
          \evensidemargin=\dimen2
          \advance\evensidemargin by \dimen1  % add in offset
%  allow for space on each side of marginal note
          \advance\dimen0 by -2\marginparsep
          \marginparwidth=\dimen0
       \else            % use supplied 2-sided value
          \oddsidemargin=\dimen3             % inside-margin-width
          \advance\oddsidemargin by \dimen1  % add in offset
          \advance\dimen0 by -\dimen3   % 8.5in-line length-inside margin
          \evensidemargin=\dimen0
          \advance\evensidemargin by \dimen1 % add in offset
%  allow for space on each side of marginal note
          \advance\dimen0 by -2\marginparsep
          \marginparwidth=\dimen0
       \fi
%  one-sided
    \else \ifdim\dimen3=0in  % use defaults: center text 
              \divide\dimen0 by 2         % (8.5in-line length)/2
              \oddsidemargin=\dimen0      % (8.5in-line length)/2
              \advance\oddsidemargin by \dimen1   % add in offset
              \evensidemargin=\dimen0     % (8.5in-line length)/2
              \advance\evensidemargin by \dimen1  % add in offset
%  allow for space on each side of marginal note
              \advance\dimen0 by -2\marginparsep
              \marginparwidth=\dimen0
          \else  % use supplied values
              \advance\dimen0 by -\dimen3  % 8.5in-line length-left margin
%  allow for space on each side of marginal note
              \advance\dimen0 by -2\marginparsep
              \marginparwidth=\dimen0
              \advance\dimen3 by \dimen1   % add in offset
              \oddsidemargin=\dimen3
              \evensidemargin=\dimen3
          \fi
    \fi
  \immediate\write16{Textwidth = \the\textwidth}
  \dimen0=1.0in
  \advance\dimen0 by \oddsidemargin
  \immediate\write16{Odd Page Left Margin = \the\dimen0}
  \dimen0=1.0in
  \advance\dimen0 by \evensidemargin
  \immediate\write16{Even Page Left Margin = \the\dimen0}
  \immediate\write16{Marginparwidth = \the\marginparwidth}
  \dimen0=#1
  \immediate\write16{Printer Offset = \the\dimen0}
 }
%
\def\@outputpage{\begingroup\catcode`\ =10 \if@specialpage 
     \global\@specialpagefalse\@nameuse{ps@\@specialstyle}\fi
     \if@twoside 
       \ifodd\count\z@ \let\@thehead\@oddhead \let\@thefoot\@oddfoot
                       \let\@themargin\oddsidemargin
% treat page 0 (title page) as if it is an odd-numbered page
        \else \ifnum\count\z@=0 \let\@thehead\@oddhead \let\@thefoot\@oddfoot
                                \let\@themargin\oddsidemargin
              \else \let\@thehead\@evenhead
                    \let\@thefoot\@evenfoot \let\@themargin\evensidemargin
     \fi\fi\fi
     \shipout
     \vbox{\normalsize \baselineskip\z@ \lineskip\z@
           \vskip \topmargin \moveright\@themargin
           \vbox{\setbox\@tempboxa
                   \vbox to\headheight{\vfil \hbox to\textwidth{\@thehead}}
                 \dp\@tempboxa\z@
                 \box\@tempboxa
                 \vskip \headsep
                 \box\@outputbox
                 \baselineskip\footskip
                 \hbox to\textwidth{\@thefoot}}}\global\@colht\textheight
           \endgroup\stepcounter{page}\let\firstmark\botmark}
%
\makeatother
%%
%% End of home:/usr/local/hacks/tex/setmargins.tex:
%%
\setmargins{6.5in}{1in}
\topmargin = 0in
\headheight = 0mm
\headsep = 0mm
\textheight = 9in
\makeindex
\begin{document}
\setlength{\parskip}{6 mm}
\setlength{\parindent}{0 in}

\title{A Quick and Dirty Guide to CLUE}
\author{Kerry Kimbrough}
\date{Version 6.0 \\
July, 1989\\[2 in]
\copyright 1989\  Texas Instruments Incorporated\\[.5in]
\parbox{3.5in}{
     Permission is granted to any individual or institution to use,
     copy, modify and distribute this document, provided that  this
     complete  copyright  and   permission  notice   is maintained,
     intact, in  all  copies  and  supporting documentation.  Texas
     Instruments Incorporated  makes  no  representations about the
     suitability of the software described herein for any  purpose.
     It is provided ``as is'' without express or implied warranty.
}}
\maketitle

This guide is a no-frills introduction to programming with  the Common Lisp User
Interface Environment (CLUE). CLUE is a high-level programming interface to the
X Window System which extends (but does not supersede) the basic CLX\index{CLX} interface.
CLUE is also an object-oriented programming system based on the Common Lisp
Object System (CLOS\index{CLOS}). Moreover, CLUE is a ``toolkit'' for constructing X user
interfaces. As a result, CLUE is modelled closely on the standard toolkit used
by C programmers, commonly known as Xt\index{Xt}, or the X Toolkit.

All of these related systems are described in separate documents (these
references are listed at the end of the guide), but we won't spend much time on
them here. Read this guide and take the plunge into CLUE; you can dry off with
the details later.

\section{The Big Picture}
Before you can understand CLUE, you have to look at three ideas about how
to build an interactive application. 


\subsection{The Application Is Event-Driven} 
An interactive application is controlled by a human user. Typically, nothing
happens until the user lifts his finger to the side of his mouse and causes
an input {\bf event}.  There are two implications here.
\index{event} 

First, the fundamental
application control structure is an {\bf event loop}: wait for an event, figure
out how to handle it, process the event, then go back and wait for the next one.
The {\em structure} of the event loop is generic; there is nothing
application-specific about it. 
\index{event, loop}

Second, the application is passive. It knows how to do certain things, but it's
waiting for the user's command to do them. You can imagine programs that do
things on their own, without waiting for user instruction (for example, a
nuclear reactor control system). But, the {\em interactive} part of such a
program is still an event loop, which must synchronize with the separate
background task(s) to provide the user control.


\subsection{The UI Is Separate} 
The user interface part has nothing to do with the essential function of the
application. An oft-cited analogy is linguistic communication, which has
independent modules for lexical, syntactic, and semantic processing. You can
understand me whether I speak to you or whether I write to you, because these
are merely two different lexical forms (e.g. sound patterns or marks) for the
same language.  Also, you can express the same idea either in French or in
English; the lexical and syntactic elements of each language are distinct from
their corresponding meaning, or semantics. 

An interactive application can be modularized along the same lines. The UI
contains the lexical and syntactic components, while the application {\em per
se} is the semantic component. Take some application command, say, ``Stop
Nuclear Reactor.'' The user interface to this command could use various lexical
forms (press the {\tt STOP} key or type the string ``{\tt stop}'') and
various syntactic forms (prompt with ``{\tt Are you sure you want to stop?}'' or
not), but the application semantics remain the same (hopefully!). From here on,
we will
acknowledge this modularization by referring to the whole application program as
``the program,'' to the user interface
components as ``the UI,'' and to the application semantic components as ``the
application.''

Separating the UI and semantic modules of an interactive program is good
programming technique. This allows you to more easily create a new UI for an
existing application. Or to apply an existing set of UI techniques to a new
application, so that it is consistent in style with others. Or to divide the
programming effort more efficiently; a domain expert can write the application
semantics while a human factors specialist can design the UI and a window system
hacker can implement it. 

\subsection{The UI Is Object-Oriented} This idea is more recent than the
other two, because it wasn't until a few years ago that graphical user
interfaces and window systems came along.  When they did, UI's began to
be composed of multiple user activities represented by visually-distinct
and physically-manipulable objects on the screen.  Object-oriented
programming (OOP) is thus a natural programming style which has been
associated with UI since its beginning\footnotemark\footnotetext{In
fact, true OOP and window systems were invented at the same time at
Xerox PARC in the Smalltalk system.}.  For several reasons, OOP is a
nice fit for UI programming.  UI's tend to generate profuse variation of
detail around a basic theme (consider, for example, menus); that is,
they demonstrate several subclasses of a basic class.  The modularity
of objects also helps to separate the UI from the semantics; that is,
the implementation of UI methods from the program which simply calls
them.

The OOP methodology leads to a UI which is composed of user interface objects
(command lines, messages, menus, scroll bars, dialogs, etc.). Each of these UI
objects is an ``agent'' for some portion of the  underlying application.
Each UI object is responsible for presenting some ``view'' of the application to
the user. It is also responsible for receiving the users events which are
directed to it and for translating these into the appropriate application
responses. 


\subsection{How CLUE Works\label{sec:how-clue-works}} 
CLUE combines all of these ideas. A CLUE program consists of an event loop, a
set of UI objects called {\bf contacts}, and a set of application functions
called {\bf callbacks}. All CLUE programs look pretty much like this:
\index{event, loop}

\begin{verbatim}
;; Open a connection to the X server.
(let ((display (open-contact-display 'my-application :host "server-host" ...)))
  ;; Initialize UI objects and callbacks.  
  (setf c (make-contact ... ))
  (add-callback c ... )
       ...
  ;; Process events until the event loop is terminated.
  (catch :event-loop
    (loop
      (process-next-event display)))  
  (close-display display))
\end{verbatim}

A CLUE program creates a {\bf contact-display} object which
\index{contact-display}
represents a connection to a specific X server.  This {\tt contact-display} then
becomes a two-way channel through which the program requests the creation and
display of contact windows and receives input events sent to its contacts. 

Each contact is a UI ``agent'' that is prepared to present some
application information, to accept user
events which manipulate this information, and then to report the results back to
the application. This reporting action is the critical connection between the UI
and the application, and it's done using {\bf callbacks}.\index{callback}
\index{callback, name}\index{callback, function}
A callback consists of
a {\bf callback name} and an associated {\bf callback function}. A contact is programmed
to associate a specific result with a callback name, which it reports by calling
the associated callback function (with a predefined argument list). 

Here is where CLUE contributes to the separation of UI and application. 
\index{application programmer}\index{contact programmer}
Every
CLUE program usually has two programmers! A {\bf contact programmer} is one
who defines a contact class and implements its methods. An {\bf application
programmer} is one who decides to create a contact instance and employ it for a
particular UI role. The contact programmer decides what callback names his
contact class will use and what the argument lists to the associated callback
functions have to be. The application programmer defines callback functions to
perform specific jobs and plugs them into the appropriate contact callback.
The contact programmer will frequently use the basic CLX\index{CLX} interface, typically to
display the contact's information. The application programmer will use CLUE
interfaces but will almost never need to call CLX functions directly.

Now we can see the operation of a CLUE program in a bit more detail. The {\tt
make-contact} function is used to create and initialize contacts. 
\index{make-contact}
The {\tt
add-callback} function \index{add-callback} associates application semantics
(i.e. callback
functions) with each contact. Then, the program enters the event loop. Each call
to {\tt process-next-event} \index{process-next-event} reads the next input event and dispatches it to the
contact to which it belongs. This contact then goes through a process of {\bf
event translation}\index{event translation}, which determines which
contact {\bf action}\index{action} functions will be
executed. Contact action functions, in turn, perform all the details of input
feedback and display. Finally, if the user event has completed an application
result, the action function will look up and call the associated callback
function. What has happened? CLUE has provided all the machinery, the contact
programmer has defined the detailed UI behavior, and the application programmer has
furnished the real results.

The rest of this guide will be a closer look at each of these parts of a CLUE
program.

\section{The Contact Display\label{sec:contact-display}}
\index{contact-display}
\index{open-contact-display} 
A {\tt contact-display} is created by the {\tt
open-contact-display} function.  {\tt op\-en\--con\-tact\-dis\-play} has one required
argument: a symbol which acts as the name of your program.  Technically, this is
a {\bf resource name}\index{resource, name} which is used to access the values of
{\bf resources}\index{resource} associated with your program.  CLUE resource
management is explained later in Section~\ref{sec:resources}.  Generally
speaking, this program resource name is used to distinguish your program from
other application programs that may also be running. {\tt open-contact-display}
has several optional keyword arguments (mainly the same ones used by the CLX\index{CLX}
{\tt open-display} function), but the most frequently used is the {\tt host}, a
string which gives the network name for the X server host to which you are
connecting.

\begin{verbatim}
(setf display
      (open-contact-display
        'blink                    ; Application name
        :host "server-host"))     ; Server host name
\end{verbatim}

{\tt contact-display} objects are really a subclass of the {\tt display} data
type defined by CLX\index{CLX}\footnotemark\footnotetext{However, neither {\tt display}
objects nor {\tt contact-display} objects are actually implemented as CLOS
classes. There's no reason to define {\tt contact-display} methods or
subclasses.}.
\index{display}

\begin{center}
\framebox[5.5in]{
\hspace*{\fill}
{\bf Note}
\hspace*{\fill}
\parbox[t]{4.5in}{
Any CLX\index{CLX} function which takes a {\tt display} argument can be called with a {\tt
contact-display} argument instead.}
\hspace*{\fill}}
\end{center}

The CLX function {\tt close-display}\index{close-display} should always be used
to close the server connection when the program terminates.

\section{Contacts\label{sec:contacts}}
In simplest terms, a {\tt contact} is a kind of window. The behavior of a window
object,
which is represented in CLX\index{CLX} as an instance of the
{\tt window} data type,
is completely specified by the X Window System
protocol.  In CLUE, the class of {\tt contact} objects is defined to be
a subclass of the class of {\tt window} objects\footnotemark\footnotetext{To do
this, CLUE actually changes the definition of the CLX {\tt window} type to
make it a CLOS class.}.  Thus, all CLX {\tt window} functions
apply to any {\tt contact}. 

\begin{center}
\framebox[5.5in]{
\hspace*{\fill}
{\bf Note}
\hspace*{\fill}
\parbox[t]{4.5in}{
Any \index{CLX}CLX function which takes a {\tt window} argument can be called with a {\tt
contact} argument instead.}
\hspace*{\fill}}
\end{center}


As a window, a contact inherits several basic attributes,
such as a
rectangular geometry (given by the position of its upper left corner, its width,
and its height in
pixel coordinates), a border width and border color/pattern, a depth (i.e.
bits per pixel), and a background color/pattern.
\index{contact, attributes}
Contacts possess a few additional attributes (or {\bf slots}) not shared
by ordinary windows. 
\begin{itemize}
\index{contact, name}
\item {\tt name}: The name of
the contact is a
symbol  which can be used to access contact resources
stored in a resource database. See Section~\ref{sec:resources}.

\index{contact, parent}
\item {\tt parent}: All
contacts have a parent contact (except for a {\bf root}\index{root} contact, which is
created automatically by CLUE and which is the ancestor of all contacts on a
single display screen).  The parent-child relationship
is fundamentally one between two windows, as defined by the X
Window System. For example, a child window is always visually
contained
by its parent window, and its position is specified relative to its
parent's upper-left corner. However, in CLUE this relationship is
extended, with the parent assuming additional responsibilities, such as
geometry management (see Section~\ref{sec:composites}).

\index{contact, state}
\item {\tt state}: A switch which controls the visual effect of the contact on
the UI. If the {\tt state} is {\tt :withdrawn}, then the contact is invisible and
unavailable for input. If the {\tt state} is {\tt :mapped}, then the contact is
``viewable'' (i.e. visible unless it is covered by other windows) and available
for input. The {\tt state} can also have a third value --- {\tt :managed} ---
which is discussed in
Section~\ref{sec:resources}. The default {\tt state} is {\tt :mapped}.


\index{contact, event-translations}
\item {\tt event-translations}: A list which associates
different types of input events with contact actions. This is discussed in more
detail in Section~\ref{sec:events}.

\index{contact, callbacks}
\item {\tt callbacks}: An association list of callback name/function pairs.
\index{callback}
\end{itemize}

\subsection{Using Contacts}
\index{contact, application programmer interface}
If you are an application programmer, your basic view of a contact --- the generic
contact protocol --- is quite simple and consists of only a few functions.
\index{contact, generic protocol}

{\tt make-contact} 
\index{make-contact}
creates a new instance of any con\-tact subclass. This function is a
syn\-tac\-ti\-cal\-ly-identical extension of the basic CLOS\index{CLOS} {\tt make-instance}
function. Keyword arguments to  {\tt make-contact}  provide initial slot values
for the new instance, most of which can be defaulted. However, a {\tt parent}
argument must be given; specifying a {\tt contact-display} as the {\tt parent}
creates a {\bf top-level contact}\index{contact, top-level}. 

\begin{center}
\framebox[5.5in]{
\hspace*{\fill}
{\bf Note}
\hspace*{\fill}
\parbox[t]{4.5in}{
A contact ``belongs'' to the {\tt contact-display} connection on which it is
created. More precisely, since a contact must have a top-level ancestor, it
belongs to the {\tt contact-display} given as this ancestor's {\tt parent}.
This {\tt contact-display} connection receives all requests and events related
to the contact; it is also available as the value of the contact's {\tt display}
slot.} \hspace*{\fill}} \end{center}


Usually (and always
for top-level contacts), you must also specify the contact's {\tt width} and {\tt
height}. A non-top-level contact may find that its initial values for 
{\tt x}, {\tt y}, {\tt width}, {\tt
height}, and {\tt border-width} have been modified by the geometry management
policy of its parent (see Section~\ref{sec:composites}).

\begin{verbatim}
(setf a-blinker
      (make-contact
        'blinker                  ; Make an instance of the blinker class
        :parent display           ; Required argument
        :width  300               ; Usually required
        :height 400               ; Usually required
        :background 0))           ; Optional, defaults to parent's
\end{verbatim}



The {\tt (setf contact-state)} function 
changes the {\tt state} value of the contact.
\index{contact, state}

\begin{verbatim}
(setf (contact-state a-blinker) :mapped)    ; Make it viewable
(setf (contact-state a-blinker) :withdrawn) ; Make it invisible
\end{verbatim}

{\tt change-geometry} 
\index{change-geometry} 
requests a change to one or more components of the contact's geometry.
The keyword
arguments to {\tt change-geometry} --- {\tt x}, {\tt y}, {\tt width}, {\tt
height}, and {\tt border-width} --- specify the changed component(s); omitting a
keyword means ``Leave the current value unchanged.'' But the actual effect of
the request depends on the geometry management\index{geometry management}
policy applied to the contact (See Section~\ref{sec:composites}). If the
request is refused, then the return values give  acceptable alternatives for {\tt x}, {\tt y}, {\tt width}, {\tt
height}, and {\tt border-width}.  You may want to give an {\tt accept-p} keyword
value of {\tt t}, which means ``Go ahead and replace my request with whatever
alternative geometry may be returned.''

\begin{verbatim}
(change-geometry a-blinker
                 :x 100           ; Request new upper-left position...
                 :y 200
                 :accept-p t)     ; ...but accept closest alternative

(change-geometry b :accept-p t)   ; Just validate current values
\end{verbatim}

\begin{center}
\framebox[5.5in]{
\hspace*{\fill}
{\bf Note}
\hspace*{\fill}
\parbox[t]{4.5in}{
There are no accessor functions for {\tt setf}'ing a contact's {\tt x}, {\tt y}, {\tt width}, {\tt
height}, and {\tt bor\-der\--width} slots. Never modify these slots directly. Always
use {\tt change-geometry} instead.} 
\hspace*{\fill}}
\end{center}


The {\tt change-priority} function \index{change-priority}
requests a change to the contact's stacking priority relative to other
windows.  The stacking priority determines which window is ``on top'' of other
windows. This request is also subject to geometry management policy. 

The {\tt destroy} function is called \index{destroy}
only when the contact will no longer be referenced. This frees any display
resources allocated to the contact.

In addition to the generic protocol, the application programmer interface to a
contact includes its callback protocol, plus any special class-dependent functions (typically, for
initialization).

\subsection{Defining Contacts}

If you are a contact programmer, you will use the {\tt defcontact} macro to
define a new contact class.  The syntax of {\tt defcontact} is nearly identical
to that of the basic CLOS\index{CLOS} {\tt defclass} macro.  The only difference is the
additional {\tt :resources} option for specifying contact resources (see
Section~\ref{sec:resources}).

\begin{verbatim}
(defcontact
  blinker                         ; Class name  
  (contact)                       ; Superclasses  
  ((color                         ; Slot specs, with name...
     :type     pixel              ; ...data type...
     :accessor blinker-color      ; ...accessor function name ...
     :initarg  :color             ; ...initarg keyword for make-contact...
     :initform 0)                 ; ...and default initial value.
   (on-p
     :type     boolean
     :accessor blinker-on-p
     :initform nil))  
  (:resources color))             ; Resource specs
\end{verbatim}

A contact programmer must also define a {\tt display} method for a new contact
class. CLUE calls the {\tt display} function automatically whenever any portion
\index{display}\index{contact, exposure}\index{exposure}
of the contact image must be (re)displayed. In particular, {\tt display} is
called whenever an invisible contact is changed to the {\tt :mapped} state, or
when some previously-hidden part of a {\tt :mapped} contact is exposed.

\begin{verbatim}
(defmethod                        ; Define a method ...
  display                         ; ... for the display function...
  ((contact blinker)              ; ... when the contact arg is a blinker.
   &optional x y width height)    ; Defines the rectangular piece exposed.
  ...)
\end{verbatim}

\section{Composites and Geometry Management\label{sec:composites}}
A contact which is the parent of another contact is known as a
{\bf composite}\index{composite} and is an instance of the {\tt composite}
subclass of {\tt contact} objects. A {\tt composite} may be the parent of another
{\tt composite},
leading to a tree-structured contact hierarchy. A contact is said to be an
{\bf ancestor}\index{ancestor} of another contact (its 
{\bf descendant}\index{descendant}) when
it is its parent or an ancestor of its parent. 

A {\tt composite} represents a set of contacts which can be manipulated
(positioned, presented,
etc.) as a unit.  A {\tt composite} is useful whenever several contacts act in
concert to provide a
single component of the UI.  Typical examples include
``control panels'' and ``dialog boxes'' --- groups of contacts that are presented
together and are used to make related adjustments to application data.  
The fundamental aspects of the contact parent-child relationship are the same as
those of the window hierarchy defined by the X Window System. 
\index{geometry management}
But in CLUE, a {\tt composite} is also expected to act as the {\bf geometry
manager}\index{geometry manager} for its child contacts; that is, to
implement a style of layout for its children. 


Here is how it works.  A request to change the geometry of a contact is
forwarded to the contact's parent, which actually performs the resulting change.
It is important to understand that, due to its constraints, a geometry manager
may not be able to perform a change as requested.  For example, a request to
increase the size of a contact might be refused if its geometry manager enforces
a maximum size.  Even if a requested change cannot be done, a geometry manager
may be able to suggest a slightly different change which would be acceptable.

Placing geometry control in the hands of a geometry manager in this way
has several advantages.
\begin{itemize}
\item   A geometry manager can arbitrate the competing geometry change requests
        of several contacts in order to implement contraints among them.
        
\item   A given layout style can be applied to any collection of contacts.
        
\item   Contact layout style can be changed without the knowledge of individual
        contacts.
\end{itemize}


A composite's geometry management policy applies only to the set of its children
which are {\bf managed}. \index{contact, managed}
An unmanaged child is ignored by its geometry manager. Any
geometry change to an unmanaged child is performed immediately as requested. 
However, changes to a managed child are arbitrated by its parent's geometry
management policy. This can mean, for example, that a change to one child's
position/size/priority can affect that of other children. Indeed, requesting a
bigger size for a child might result in the parent asking {\em its} geometry
manager to become bigger, leading to a ripple of geometry changes throughout
the contact hierarchy.

How do you know if a contact is managed? This is determined by its {\tt
state}.\index{contact, state}
A contact is managed if and only if its {\tt state} is not {\tt :withdrawn}. This means
that a {\tt :mapped} (and viewable) contact is managed. But there is also
another possibility: a contact's {\tt state} may also be {\tt :managed}. This
value represents the (rather rare) case of a contact which is not visible, but
which is nevertheless taken into account by its geometry manager. 

\begin{center}
\framebox[5.5in]{
\hspace*{\fill}
{\bf Note}
\hspace*{\fill}
\parbox[t]{4.5in}{
An
unmanaged child can {\em never} be visible.}
\hspace*{\fill}}
\end{center}


Implementing geometry management is a topic for contact programmers only. If you
define a new {\tt composite} subclass, you will usually need to implement
methods for the following three functions.
\begin{itemize} 
\item {\tt change-layout}: 
\index{change-layout}
This function is called whenever a composite's set of
managed children changes (e.g. by creating or destroying a managed child or by
changing the {\tt state} of a child).

\item {\tt manage-geometry}: 
\index{manage-geometry}
This function is called by {\tt change-geometry} to
approve a geometry request. It must either approve the change or return a set of
alternative geometry values which will be acceptable.

\item {\tt manage-priority}: 
\index{manage-priority}
This function is called by {\tt change-priority} to
approve a stacking priority request. It must either approve the change or return
an alternative priority which will be acceptable.

\end{itemize}

\section{Events\label{sec:events}}
\index{event}\index{input}
Handling input events is CLUE's main job, and it's an area where CLUE does a lot
of the work for programmers automatically. Application programmers don't have to
worry about much more than setting up an event loop, as shown in
Section~\ref{sec:how-clue-works}. Contact programmers need to know how to use
CLUE's {\bf event translation}\index{event translation} mechanism in order to
implement the details of a contact's input behavior.

What exactly are these events we're talking about? Fundamentally, an event is
a message to the program from the X server, a packet of data describing
some occurrence that the program ought to be interested in.
Most events
represent an action performed by the user with  the X server's 
\index{input, devices}
keyboard or pointing device (i.e. mouse, tablet,
etc.). Keyboard
actions, for example, can generate {\tt :key-press} and {\tt :key-release}
events. The pointer may be used to generate {\tt :button-press} and {\tt
:button-release} (when the user presses one of the mouse buttons) and {\tt
:motion-notify} events (when the user changes the pointer position). These kind of
{\bf user events}\index{event, user} contain data for the button or key
involved, the coordinates of the pointer position, and other useful things, such as
the up/down state of {\tt :shift}, {\tt :control}, and the other modifier keys.
In addition to user events, there is another important group of {\bf side-effect
events},\index{event, side-effect}
events which occur as an indirect result of other user actions. For example,
when the user causes part of an obscured window to become visible, an {\tt
:exposure}\index{exposure} event may be sent. Other side-effect events include
{\tt :enter-notify}, which signifies that the user has moved the pointer cursor
inside a particular window, and {\tt :focus-in}, which happens when the user has
identified a particular window as the {\bf focus} for keyboard
events.\index{focus}\index{input, focus}
This is just the beginning, but the definition and meaning for all of the events
received by CLUE contacts can be found by reading the X Protocol Specification.


To understand CLUE event processing, let's start with the basic program event
loop\index{event, loop} and follow its operation, step by step.
A call to {\tt
process-next-event} \index{process-next-event}\index{timeout}
gets the ball rolling. {\tt
process-next-event} causes CLUE to read
the next event from the given {\tt contact-display} connection. By default,
{\tt pro\-cess\--next\--ev\-ent} does not return until the next event has been
completely processed. This means that if no event is yet available, then {\tt
process-next-event} will generally wait until one finally comes along. However,
an optional {\tt timeout} argument may be given, which says how long {\tt
process-next-event} will wait before giving up. Set the {\tt timeout} to {\tt 0}
if you don't want {\tt process-next-event} to wait at all. {\tt
process-next-event} returns {\tt nil} if a timeout occurred; otherwise, it
returns {\tt t}.

\begin{verbatim}
(catch :event-loop
  (loop
    (unless 
      (process-next-event display 5)  ; Timeout after waiting 5 sec.

      ;; No events yet -- do something useful
      (do-background-task))))
\end{verbatim}

\begin{center}
\framebox[5.5in]{
\hspace*{\fill}
{\bf Note}
\hspace*{\fill}
\parbox[t]{4.5in}{
{\tt process-next-event} does not return any result of the actual processing of
an event. This means
that all responses of a CLUE program to user input  --- even the termination
of the event loop --- occur as ``side-effects'' of contact event handling.
Always escape the event loop by {\tt throw}'ing to some well-defined tag.}
\hspace*{\fill}}
\end{center}

Inside {\tt process-next-event}, the next thing that happens is that the event is
{\bf dispatched}\index{event,
dispatching}. In other words, CLUE figures out which contact is supposed to
handle the event. This is fairly easy because almost every event message
contains an identifier for the {\tt window} object which is the
``addressee'' of the
event. CLUE knows how to convert this identifier into the corresponding {\tt
contact} object. At this point, the event
is {\bf handled}\index{event, handling} by calling the {\tt handle-event}
function\index{handle-event} with the receiving {\tt contact} and the event as
its arguments. {\tt handle-event} then implements the process of event
translation, i.e. figuring out which methods of the receiving {\tt contact} to
call in response to the event\footnotemark\footnotetext{{\tt
handle-event} is also the function which automatically calls the
contact's {\tt display} method, if the event is an {\tt :exposure}.\index{display}\index{exposure}}.
What happens next is of interest only to contact programmers.

\subsection{Event Translation}
Event translation involves searching through one or more association lists,
looking for an entry that matches the event. Each such entry is called an {\bf
event translation}, and it is a list containing an {\bf event
specification}\index{event, specification} and one or more {\bf
action}\index{action} names. An event specification describes a certain sort of
event, and an action name is simply the name of a special kind of function
--- 
an action function. So, during event translation, the event is compared with an
event
specification, and when it matches, then each of the corresponding action functions are
invoked in sequence. When all of them have completed, then event
translation is done; {\tt handle-event} returns and the event is said to
be ``handled.'' The {\tt process-next-event} function also returns, and we
go back to the top of the event loop to deal with the next event.
But where are these event translations and how are they created? There are two
different ways to create event translations: the {\tt add-event} function and
the {\tt defevent} macro.

{\tt add-event}\index{add-event}
creates an event translation and adds it to the association list
found in the {\tt event-translations} slot of a specific contact. This kind of
event translation is thus an {\bf instance event
translation}\index{event translation,
instance} which affects only one contact instance. An instance event
translation says ``When an event is dispatched to {\em this} contact and it
matches {\em this} event specification, then call {\em these} actions.''

\begin{verbatim}
(add-event a-blinker                 ; When this contact gets...
           '(:key-press #\ctrl-b)    ; ...this character event ...
           'beep)                    ; ...call the beep action.

(add-event a-blinker                 ; When this contact gets...
           '(:key-release #\ctrl-b)  ; ...this character event ...
           '(beep 1))                ; ...call the beep action with an arg.
\end{verbatim}

Note that in place of a simple action name, an event translation can also give a
list containing the action name plus a list of arguments to pass.
The process of event translation starts with the instance event translations in
the {\tt event-translations} slot of the receiving contact. Each entry is
examined in order, until a match is found. What happens if no match is found? In
this case, the process continues to search the event translations defined by the
{\tt defevent} macro.

{\tt defevent}\index{defevent}
creates a  {\bf class event translation}\index{event translation, class}
that applies to every instance of a given class.  A class event
translation says  ``When an event is dispatched to a contact of {\em this} class
and it matches {\em this} event specification, then call {\em these} actions.''

\begin{verbatim}
(defevent blinker                    ; When any blinker gets...
          (:button-press :button-1)  ; ...this button event...
          (blink t))                 ; ...blink on.

(defevent blinker                    ; When any blinker gets...
          (:button-release :button-1); ...this button event...
          (blink nil))               ; ...blink off.
\end{verbatim}

Class event translations are searched for each contact superclass of the
receiving contact, starting with the class of the contact, then continuing back
up the {\bf class precedence list}\index{class precedence list} which this
contact inherits\footnotemark\footnotetext{See the CLOS\index{CLOS} specification for a precise definition of the
class precedence list.}.  The idea is to match a class event translation
for the
most specific class possible.  However, if a matching event translation is {\em
still} not found, then CLUE gives up, and the event is ignored. After
all this, it couldn't be a very important event anyway!

\subsection{Event Specifications}
\index{event, specification}
As you may have noticed, an event specification is usually a list
composed of an event type keyword and some other qualifiers. In fact, an event
type keyword alone is also a valid event specification.
The exact syntax for event specifications is fairly complex. A sophisticated
programmer can even define his own event specification syntax. All of this is
completely spelled out in the CLUE specification. But for now, you can get an
idea of what an event specification is by looking at the following examples.

\begin{verbatim} 
:enter-notify         ; Matches any :enter-notify event

(:motion-notify       ; Matches a :motion-notify event..
  :button-1)          ; ...if :button-1 is down.

(:button-press        ; Matches a :button-press...
  :button-1           ; ...if it's :button-1...
  :shift)             ; ...and the shift key is down.

(:button-release      ; Matches a :button-release...
  :button-3           ; ...if it's :button-3...
  (:shift :control)   ; ...and the shift,control keys are down...
  :all)               ; ...and all other modifiers are up.

(:key-press           ; Matches a :key-press...
  :any                ; ...for any key...
  (:control :hyper))  ; ...when the control,hyper keys are down.

(:button-press        ; Matches a :button-press...
  :button-3           ; ...if it's :button-3...
  :double-click)      ; ...and it's a double-click.
\end{verbatim}

\begin{center}
\framebox[5.5in]{
\hspace*{\fill}
{\bf Note}
\hspace*{\fill}
\parbox[t]{4.5in}{A contact will receive {\em only} the types of events described
by
its event specifications, i.e. by event specifications in its instance and class
event translations. CLUE automatically sets the 
{\bf event mask}\index{event, mask} of the contact window, based on the
contact's event specifications. Event masks are a detail of CLUE internals, but
you can find more information about them in the X Protocol
Specification and the CLUE specification.}\hspace*{\fill}} \end{center}

\subsection{Actions and Callbacks}
\index{action}
An action is a  function which is designed to handle an input
event.  An action therefore implements a particular input behavior exhibited by
each instance of a certain contact class. 
In fact, an action function is typically a CLOS generic function, and actions
usually are defined as methods of a particular contact class. 

\begin{verbatim} 
(defmethod blink ((contact blinker) blink-on-p)
  (with-slots (on-p) contact
    ;; Set internal state variable on/off
    (setf on-p blink-on-p)
    ;; Redisplay based on new on-p state
    (display contact)
    ;; Invoke callback with new state
    (apply-callback contact :blink on-p))
\end{verbatim}

In general, an action represents a well-defined contact behavior that
could be done in
response to {\em any} event. In fact, it's possible to personalize an existing UI by
modifying its event translations so that contact actions are rebound to
the types of events (i.e. event specifications) which
are more to your taste.  In some cases,
however, an action may be designed to handle a very specific event type.

When an action computes a result that is important to an application, it invokes
a callback.\index{callback}
This is done via the {\tt apply-callback} macro.\index{apply-callback}
The example above shows that a {\tt blinker} contact has a callback whose name
is {\tt :blink}, which is called with a single argument, i.e. a boolean
representing the new on/off state of the {\tt blinker}. What significance
this has to the application is unknown to the contact programmer who wrote this
action; it all depends on what callback function the application programmer has
associated with {\tt :blink} for this {\tt blinker} instance.
In fact, the application programmer may have decided not to define a callback
function for {\tt :blink}, in which case {\tt :blink} has no application meaning
at all (and {\tt apply-callback} simply does nothing).
As shown in the {\tt :blink} example,
{\tt apply-callback} is usually (although not
necessarily)
called directly by an action method or somewhere within its dynamic extent.

\begin{center}
\framebox[5.5in]{
\hspace*{\fill}
{\bf Note}
\hspace*{\fill}
\parbox[t]{4.5in}{
Callback names are shared by a class, i.e. referenced by the class's
action methods.  But
callback name/function pairs are instance data and are recorded in a contact's
{\tt callbacks} slot. In our example, every {\tt blinker} is
expected to have a {\tt :blink} callback function, but each {\tt blinker}
usually has a different {\tt :blink} callback function. That is, each {\tt
blinker} usually has different application semantics.} \hspace*{\fill}}
\end{center}

In the previous example, the {\tt blink} action method does not depend on the
event which causes it to be invoked.  But what happens when an action's behavior
depends on information contained in the event?  CLUE handles this by
representing an event as an instance of the {\tt event} class.  An {\tt event}
object has slots which contact various kind of interesting information about the
event.  Since there are many types of X events, there are many different {\tt
event} slots, even though for a given event some slots are irrelevant and
therefore have a {\tt nil} value. You can look in the X Protocol Specification for
a complete description of event slots (and look in the CLX specification to find
out how these slot values are represented in Lisp).

But how does an action access an event object? This is done by using a special
CLUE macro called {\tt with-event}. For example:

\begin{verbatim}
(defmethod beep ((contact blinker) &optional (per-cent-volume 0))
  (with-event (state)
    ;; Was the shift key down?
    (when (member :shift (make-state-keys state))
      ;; Ring server's chime!
      (bell (contact-display contact) per-cent-volume)  
      ;; Invoke callback
      (apply-callback contact :beep))))
\end{verbatim}

The {\tt beep} action will beep only if it determines that the shift key was
down when the event occurred.  It does this by examining the {\tt state} slot of
the event, which defines which modifier keys were pressed at the time of the
event.  Notice the use of the {\tt with-event} macro.\index{with-event} This is
similar to the {\tt with-slots} macro of CLOS\index{CLOS}.  {\tt with-event}
binds slots of the ``current'' event argument within its lexical extent, so that
the action code can refer to them.  But why is the current event object hidden
in this way?  Because this allows for a much more efficient implementation of
{\tt event} objects than would be possible if their structure was fully exposed
to programmers.  So, there!

\subsection{Advanced Input}
\index{input}
Input events are a subject so dear to the heart of CLUE that it
contains many more input programming functions than can be
covered in this guide. Here's a quick list of CLUE's special input
features, all of which are discussed in complete detail in the CLUE
specification.
\begin{itemize}
\item {\bf Before actions},\index{before action}\index{action, before}
action functions which can be set to execute for {\em every} event
dispatched to a contact of a certain class, before event translation begins. 

\item {\bf Timers},\index{timer}
which send special {\tt :timer} events to a contact at a regular,
specified rate.

\item {\bf Synchronizing}\index{event, synchronizing}
event processing, so that a program can stop to process all pending
events before continuing.

\item {\bf
Sensitivity},\index{contact,
sensitivity} which allows a contact's input to be temporarily ``switched
off'' without changing its visibility.

\item {\bf Event compression},\index{event, compression}
which removes certain redundant events for faster performance.
\end{itemize}

\section{Resources\label{sec:resources}}
Imagine that you are the user of an interactive program. Maybe you are
not a programmer, but there are certain things about the UI that you
wish you could change ever so slightly. For example, maybe you wish you
could make the program use a different set of colors for various
objects. Or maybe you're left-handed, and you'd prefer to use the right
mouse button in place of the left one. Or maybe you can't read Japanese,
which happens to be the language all the menu items are written in.
Although most of the UI is defined by programmers  --- contact
programmers who designed the various UI objects and application
programmers who placed them neatly on the screen --- the truth is that
some aspects of the UI ought to be up to the {\em user}. Programmers,
keep your hands off!

Nowadays, thoughtful UI programmers recognize this problem, so CLUE
allows  programmers and users to cooperate in
defining the UI. The basis for this cooperation is {\bf resource
management}. A user can store various UI values as {\bf
resources}\index{resource} in a {\bf resource database}\index{resource,
database}, and CLUE contacts can read these resource values and modify
the UI accordingly. 

\subsection{The Resource Database}
Before a user can assign a resource value to a particular UI object, he
must have a way of identifying it, some kind of  name which
can be referenced outside of the program which creates the object.
Remember that a contact has a {\tt name} slot\index{contact, name}; this
is a symbol that could certainly help identify the contact. Consider
also that a contact is generally part of a nested hierarchy of UI
objects. For example, a paint program might have pattern choice
contact which contains several button contact children, one for each available
pattern style. A more complete name for a button child would be a
list which also includes its parent's name, e.g. {\tt '(pattern-choice
checkered)}. We can extend this notion all the way up the hierarchy to
come up with a {\bf complete resource name}\index{resource, complete
name} for the {\tt checkered} pattern button. This would be a list of
resource name symbols, starting with the resource name for the program
itself (i.e. the required argument to {\tt
open-contact-display}\index{open-contact-display}) and continuing with
symbols for the {\tt root}\index{root} ancestor, the top-level ancestor,
and so on:\index{resource, name}

\begin{verbatim}
    '(paint screen-1 pattern-choice checkered)
\end{verbatim}

Assuming that the {\tt checkered} button has a resource for its label
string named {\tt label} and another resource for the label font named
{\tt font,}
then the complete names for these resources would obviously be:


\begin{verbatim}
    '(paint screen-1 pattern-choice checkered label)
    '(paint screen-1 pattern-choice checkered font)
\end{verbatim}

As you might expect when dealing with long path names like these, it's
often useful to insert ``wildcards'' in the places where you don't want
to be so specific. For example, the following resource name would refer to
the {\tt font} of {\em everything} in the pattern choice contact, regardless of
the screen where it appears:

\begin{verbatim}
    '(paint * pattern-choice * font)
\end{verbatim}

Another way of identifying a contact is by its object class. So, for
example, you could refer to the font of {\em every} {\tt button} object in the
paint program with the following resource name.

\begin{verbatim}
    '(paint * button font)
\end{verbatim}

By associating values with these sorts of resource names in {\bf
resource bindings}\index{resource, binding}, a user can
create a resource database which represents his UI preferences.
CLUE\index{CLUE} provides a {\tt define-resources} macro for adding
resource bindings to the database.\index{define-resources}
\begin{verbatim}
(define-resources
  (paint * button background)               white
  (paint * pattern-choice * font)           helvetica-12
  (paint * pattern-choice checkered label) "Checkered")
\end{verbatim}

CLUE automatically reads the resource database whenever a contact is
created. A contact has a class and possibly other contact superclasses
on its class precedence list\index{class precedence list}. Each such
(super)class has a list of resource names that it uses (defined by the
{\tt :resources} option of the defining {\tt
defcontact}\index{defcontact}
form). Together, these form the set of contact resources that are looked
up in the resource database. Of course, finding a resource value is rather complicated,
because names in the database may not be complete resource
names and may contain a mixture of resource names and class symbols.
Nevertheless, during initialization,
CLUE will find the resource binding that is the closest match for each
contact resource. In general, a contact gets its resource values  first
from the user's resource database, then from programmer-specified
defaults when no user value is found.

\begin{center}
\framebox[5.5in]{
\hspace*{\fill}
{\bf Note}
\hspace*{\fill}
\parbox[t]{4.5in}{
The contact programmer decides which values a contact will look up in
the resource database; the user decides what values it will find there.
The contact programmer identifies resources by listing them in the {\tt
:resources} option of the {\tt defcontact} form for his contact class.\index{defcontact}}
\hspace*{\fill}} \end{center}

\subsection{Converting Resource Values} 
\index{resource, conversion}
\index{resource, types}
Another convenient feature of
CLUE resource management is automatic resource type conversion.  Often
the type of value specified by the user in the resource database is not
the data type actually used by the program.  For example, a user might
identify a font by a string such as ``{\tt helvetica},'' but the program
must convert this name into a CLX {\tt font} object before text is
displayed.  Another example is color: where a user might specify ``{\tt
red},'' a program must somehow determine a colormap and pixel value that
will yield this hue.

CLUE automatically converts user values out of the user's resource
database into the correct target data type (as specified in the {\tt
:resources} option of a {\tt defcontact}\index{defcontact} form). This
is done by calling the {\tt convert} function\index{convert}. For
example:

\begin{verbatim}
(setf color (convert         ; Convert a resource...
              a-blinker      ; ...for a blinker...            
              "red"          ; ...from a string value...              
              'pixel))       ; ...to a pixel target type.
\end{verbatim}

CLUE defines a number of {\tt convert} methods for various combinations
of source and target data types. Contact programmers can extend
this mechanism by defining their own special {\tt convert} methods.

\section{Programming Tips\label{sec:tips}}
\subsection*{Managing the Server Connection}
You'll get into trouble if you aren't careful to close down the X server
connection when your program terminates. The reason? Most X servers have a limit
to the number of client connections which they can serve at one time. When the
limit is reached, the server simply refuses to open any new connections, and
you'll end up in the error handler staring at an ``Unable to connect'' message.  Make
sure your program
{\em always} terminates with a call to {\tt close-display},\index{close-display}
even when it aborts unexpectedly. The best way is to put your event loop inside
an {\tt unwind-protect} form:

\begin{verbatim}
...
(unwind-protect 
  (catch :event-loop
    (loop
      (process-next-event display)))  
  (close-display display))
...
\end{verbatim}

One nifty thing about using this technique is that it's always safe to
abort out of a damaged program and start over.

Also, you should generally avoid binding {\em global} special variables
to {\tt contact-display} objects representing open server connections. It's too
easy to lose track of a global variable or, even worse, to garbage a {\tt
contact-display} while it's still open.\index{contact-display}

\subsection*{Debugging}
\index{debugging}
Debugging a CLUE program (or, indeed, any program using the X Window System)
requires an awareness of a simple fact of life: {\bf client-server communication
is buffered}. Just calling the CLX {\tt draw-line} function won't necessarily
cause a line to appear on the screen. Instead, CLX places the corresponding
server request into an output buffer and moves on; the request is sent to the
server and executed later, when the output buffer is flushed. Normally, you
don't have to worry about this because CLX will automatically flush the output
buffer at the ``right'' time\footnotemark\footnotetext{When the output buffer is
full and (by default) before reading the next input event.}. However, during
debugging, you may want requests to be executed immediately. There are
two ways to do this.

\begin{enumerate}
\item Call {\tt display-force-output} manually at the appropriate time.
\index{display-force-output}
\item Invoke {\tt (setf (dis\-play-af\-ter-func\-tion dis\-play)
\#'dis\-play-force-out\-put)}. \\
This tells CLX to flush the
output buffer automatically as soon as each request
is made.
\index{display-after-function} 
\end{enumerate}

\index{errors}
Output buffering also means that when a request is in error, you won't
see the error immediately. The X server reports an error by sending an {\bf
error reply} back to the program (in this sense, an error reply is another kind
of input ``event'').
The error will be reported after the invalid
request is sent to the server, but by this time, your program has usually sent a
number of other requests, too. The result? When you arrive at the error handler,
you almost never be anywhere close to the real scene of the crime.
Using the {\tt display-after-function}\index{display-after-function} for
``single-step'' output, as shown above, can help to mitigate this problem. But
at any rate, you will need to pay careful attention to the contents of the error
reply which is printed out in CLX's error message. Sometimes, your best recourse
is to compare the error
reply with the error behavior defined by the X Protocol Specification for the
offending request and thereby deduce the cause of the bug.

Another useful debugging technique is to (locally) bind your {\tt
contact-display} and important {\tt contact} objects to special variables within
your program. This makes it easier to {\tt break} at a convenient
point (say, while the program is idle, waiting for an event to occur) and then
examine the state of various contacts or make server requests to return
interesting data.

\section{Getting Started\label{sec:getting-started}}
\subsection{Building CLUE}
If you're an Explorer user, you'll want to use an Explorer Release 6 load
band. This band has the latest versions of CLOS, CLX, and CLUE already built in.
If this applies to you, you can stop here --- you're ready to go!

But if your system doesn't come with CLUE built in, you'll need to go through
the following procedures to get things properly loaded.
Three systems need to be loaded: CLOS\index{CLOS}, CLX\index{CLX}, and CLUE
itself. CLX needs to be loaded specially  --- a small patch to the standard X11
R3 version of CLX is needed to make it work with CLUE.

\begin{enumerate}
\item Decide which version of CLOS you want. Possible choices include:

\begin{itemize}
\item {\tt PCL}, the Portable Common Loops implementation from Xerox that has
recently been made compatible with the CLOS specification. See CLUE release
notes for more details.\index{PCL}
\item {\tt clos-kludge}, a simple implementation of a CLOS subset which comes
with
the other CLUE software and which is sufficient for getting started with CLUE.
See the {\tt defsystem} file in the {\tt clos-kludge} directory.\index{clos-kludge}
{\tt clos-kludge} works but beware: you should use this only as a temporary
stop-gap until you have a ``real'' CLOS in place.
\item Something else. Maybe you've done a CLOS implementation yourself! Or maybe
you've gotten one from the vendor of your Common Lisp.
\end{itemize}

\item Compile and load your preferred CLOS.

\begin{center} \framebox[5.5in]{
\hspace*{\fill}
{\bf Note}
\hspace*{\fill}
\parbox[t]{4.5in}{
Make sure that before both CLX and CLUE are {\em compiled}, you have loaded your
preferred version of CLOS. If you switch CLOS versions later, be sure to
recompile both CLX and CLUE.} \hspace*{\fill}} \end{center}


\item Compile and load CLX, with the special CLUE patches with are included with
the other CLUE software.  See the CLUE {\tt defsystem} file.  Lisp machine
users may find it convenient to perform {\tt (make-system 'clx-clos)}.

\item Compile and load CLUE. See the CLUE {\tt defsystem} file.  Lisp machine
users may find it convenient to perform {\tt (make-system 'clue)}

\end{enumerate}

     
  
\subsection{Packages}
\index{packages, CLUE}
\index{packages, CLUEI}\index{intrinsics}
All symbols defined by the CLUE ``intrinsics'' are external in the {\tt
CLUEI} package.  The {\tt CLUE} package exports all symbols defined by the
standard contact set, 
\index{contact, standard set}
in addition to all of the
``intrinsics'' in {\tt CLUEI}\footnotemark\footnotetext{For now, the standard
contact set is undefined so using either the {\tt CLUE} or {\tt CLUEI}
package is effectively equivalent. Similarly, the CLUE specification currently
describes
only the intrinsics and the CLUE package, and does not yet mention the
standard contact set.}.

All symbols defined by CLX\index{CLX} are external in the {\tt XLIB} package. The {\tt
CLUE} and {\tt CLUEI} packages both use {\tt XLIB}.

The package containing CLOS\index{CLOS} symbols may vary, depending on which
implementation of CLOS you are using. Usually, you will want to use the {\tt
CLOS} package which exports the Explorer CLOS system. 

\index{packages, CLOS}\index{packages, LISP}\index{packages, TICL}
\begin{center}
\framebox[5.5in]{
\hspace*{\fill}
{\bf Note}
\hspace*{\fill}
\parbox[t]{4.5in}{
If you are using the Explorer {\tt CLOS} package, be sure to
also use only {\tt LISP}, the standard Common Lisp package, not {\tt TICL}.
Using both {\tt TICL} and {\tt CLOS} together will result in some nasty name
collisions (for example, on {\tt make-instance} and {\tt defmethod}.)}
\hspace*{\fill}} \end{center}


Application programmers should define a separate package for application symbols
which uses either:
\begin{itemize}
\item {\tt LISP}, {\tt CLOS} and {\tt CLUE}, or
\item {\tt LISP}, {\tt CLOS}, {\tt CLUEI}, and a non-standard contact package.
\end{itemize}

Contact programmers should define a separate package for each set of
related contacts, which uses either: 
\begin{itemize}
\item {\tt LISP}, {\tt CLOS}, {\tt XLIB}, and {\tt CLUE}, (if using standard
contacts), or 
\item {\tt LISP}, {\tt CLOS}, {\tt XLIB}, and {\tt CLUEI}.
\end{itemize}

\subsection{Warming Up}
Here are some CLUE calisthenics to limber you up. Some of them
will require further study of CLUE, CLX, and the X Window System. Some of them
are rather challenging.

\begin{enumerate}
\item Complete the {\tt blinker} contact example. Write a {\tt display} method
that will fill the {\tt blinker} with its {\tt color} when its {\tt on-p}
state is
true and fill it with its {\tt background} otherwise. Ensure
that the {\tt blinker-on-p} accessor updates the display correctly.

\item Create two  {\tt blinker} instances --- one that will print out either
``{\tt Off!}'' when
its {\tt :blink} callback is called with a {\tt nil} argument or ``{\tt On!}''
otherwise, and another which will make a funny noise on your Explorer {\em only}
when its {\tt :blink} callback is called with a non-{\tt nil} argument.

\item Try out different contact attributes on your {\tt blinker}. For example,
try a different {\tt border-width}, {\tt border}, and {\tt background}. Try
initializing the {\tt background} to a {\tt '(float 0 1)} value. What happened?
Why?

\item Define resources to change initial {\tt blinker} attributes.

\item Create two ``funny noise'' {\tt blinker} instances, as in the previous
exercise, but give them different funny noises. Do {\em not} change the
definition of the {\tt blinker} class.

\item Change the {\tt blinker} class so that whenever the cursor enters a {\tt
blinker}, it turns into a picture of Gumby. Hint: use the {\tt xfd} program
(font displayer) to find an appropriate element of the {\tt cursor} font.
Another hint: Use ``{\tt man xfd}'' to learn how to use {\tt xfd}.

\item Change the {\tt blinker} class to include a {\tt string} slot, containing
a string which can be displayed in any font. Display the string so that it
appears in the {\tt blinker}'s {\tt color}, is centered horizontally and
vertically within the {\tt blinker}'s current interior size, and is visible
regardless of the {\tt blinker}'s {\tt on-p} state. Ensure that the {\tt string}
remains centered when the user changes the font or the size of the {\tt
blinker}.

\item Define a {\tt composite} class that {\em always} has a single {\tt blinker}
child, which is always centered horizontally and vertically within the {\tt
composite}. Ensure that the {\tt blinker} child
remains centered when the user changes the size of the {\tt composite}.

\item Define an {\tt etch-a-sketch} contact that will:
\begin{itemize}
\item 
Rubberband a horizontal
line as the user moves the pointer with {\tt :button-1} down, then
snap it into place when the button is released.
\item Rubberband a vertical
line as the user moves the pointer with {\tt :button-2} down, then
snap it into place when the button is released.
\item Print out a copy of its current line drawing on the Imagen
when the {\tt \#$\backslash$return} key is pressed.

\end{itemize}

\end{enumerate}
\pagebreak
\begin{thebibliography}{9}

\bibitem{clos} Bobrow, Daniel G., et al. The Common Lisp Object System
Specification (X3J13-88-002). American National Standards Institute, June,
1988. 

{\em [The definition of CLOS. Start with Chapter 1, which is a thorough and
readable explanation of all concepts. Then consult with Chapter 2 for
programming details.]}

\bibitem{clos-guide} Keene, Sonya E. {\em Object Oriented Programming in Common
Lisp}. Addison-Wesley (1989).

{\em [A textbook-style guide to CLOS, written by a member of its ANSI
development team. Discusses all CLOS features, with examples.]}

\bibitem{Xt} McCormack, Joel, et al. The X-Toolkit Intrinsics, Version
11, Revision 2 (March, 1988).

{\em [Xt, the C language counterpart to CLUE. This is important because
CLUE will tend to evolve in the same direction as Xt.]}

\bibitem{x-protocol} Scheifler, Robert W. The X Window System Protocol, Version
11, Revision 3 (December, 1988).

{\em [The X Bible. The definitive word on how X works. A reference manual with
no tutorials or examples.]}

\bibitem{x-window-system} Scheifler, Robert W. and Gettys, Jim. The X Window
System. {\em ACM Transactions on Graphics}, Vol. 5, No. 2 (April 1986).

{\em [An excellent technical overview of the design and features of the X Window
System. Somewhat obsolete (written before X Version 11) but still informative.]} 

\bibitem{clx} Scheifler, Robert W., et al. CLX Interface
Specification, Version 4 (September 1987).

{\em [Terse, but this is the definition of CLX. Describes only the language
binding, so it must be read in conjunction with the X Protocol Specification.]}

\end{thebibliography}


\pagebreak
\begin{theindex}
% -*- Mode:TeX -*-
%%%----------------------------------------------------------------------------------+
%%%                                                                                  |
%%%                          TEXAS INSTRUMENTS INCORPORATED                          |
%%%                                  P.O. BOX 149149                                 |
%%%                             AUSTIN, TEXAS 78714-9149                             |
%%%                                                                                  |
%%%                Copyright (C) 1989 Texas Instruments Incorporated.                |
%%%                                                                                  |
%%% Permission is granted to any individual or institution to use, copy, modify, and |
%%% distribute this document, provided that  this complete copyright and  permission |
%%% notice is maintained, intact, in all copies.                                     |
%%%                                                                                  |
%%% Texas Instruments Incorporated provides this document "as is" without express or |
%%% implied warranty.                                                                |
%%%                                                                                  |
%%%----------------------------------------------------------------------------------+
\documentstyle[twoside,12pt]{article} 
\pagestyle{plain} 
%%
%% Inserted from home:/usr/local/hacks/tex/setmargins.tex:
%%
%%    Original: Glenn Manuel 12-17-86
%%    Added optional [printer-offset]: Glenn Manuel 2-27-87
%% Sets the margins, taking into account the fact that LaTex
%% likes to start the left margin 1.0 inch from the left edge
%% of the paper.
%%   This macro prints on the screen and in the log file
%%   the values it sets for:
%%     Textwidth, Odd Page Left Margin, Even Page Left Margin,
%%     Marginparwidth, Printer Offset
%%     (the values are in points: 72.27 pts/inch).
%%     The Odd and Even Page Left Margins include LaTeX's
%%     built-in 1.0in offset, but NOT the [printer-offset],
%%     so these values always indicate what the actual
%%     printout SHOULD measure.
%%
%% USAGE:  Place the following between the
%%         \documentstyle   and  \begin{document} commands:
%% \input{this-file's-name}
%% \setmargins[printer-offset]{line-length}{inside-margin-width}
%%
%%   where ALL arguments to \setmargins are DIMENSIONS
%%           like {6.5in}, {65pt}, {23cm}, etc.
%%   The units MUST BE SUPPLIED, even if the dimension is zero {0in}.
%%
%%   [printer-offset] is optional.  Default is zero.
%%
%%   Examples:
%% \setmargins{0in}{0in}         % default line length & default margins
%% \setmargins[-.12in]{0in}{0in} % compensate for printer offset
%% \setmargins{0in}{1in}         % default line length, 1 inch inner margin
%% \setmargins{6.5in}{0in}       % 6.5 inch line length & default margins
%% \setmargins{5in}{1.5in}       % 5 inch line length & 1.5 inch inner margin
%%
%%   {inside-margin-width} is defined as follows:
%%       For 1-sided printing: left margin for all pages.
%%       For 2-sided printing: left  margin for odd pages,
%%                             right margin for even pages.
%%
%%   Defaults:
%%   Each argument has a default if {0in} is used as the argument:
%%      line-length default = 6.0in
%%      inside-margin-width default:
%%         For 1-sided printing, text is centered on the page
%%                     (each margin = [1/2]*[8.5in - line length]);
%%         For 2-sided printing, inside margin is twice the outside margin
%%                     (inside  margin = [2/3]*[8.5in - line length],
%%                      outside margin = [1/3]*[8.5in - line length]).
%%      printer-offset default = 0in
%%
%%  For all cases, the outside margin (and marginparwidth, the
%%  width of margin notes) is just whatever is left over after
%%  accounting for the inside margin and the line length.
%%
%% Note: LaTeX's built-in offset of 1.0 inch can vary somewhat,
%%       depending upon the alignment of the Laser printer.
%%       If you need it to be EXACT, you will have to supply
%%       the optional [printer-offset] argument.
%%       Subtract the actual measured left margin on an
%%       odd-numbered page from the printed Odd Page Left Margin
%%       value, and use the result as the [printer-offset].
%%       Positive values shift everything to the right,
%%       negative values shift everything to the left.
%%
\makeatletter
\def\setmargins{\@ifnextchar[{\@setmargins}{\@setmargins[0in]}}
\def\@setmargins[#1]#2#3{
%%%  Uses temporary dimension registers \dimen0, \dimen2, \dimen3, \dimen1
    \dimen1=#1                         % 1st argument [printer offset]
    \dimen2=#2                         % 2nd argument (line length)
    \dimen3=#3                         % 3rd argument (inner margin)
     \advance\dimen1 by -1.0in         % for LaTeX built-in offset
    \ifdim\dimen2=0in 
        \textwidth=6in  \dimen2=6in
    \else \textwidth=\dimen2
    \fi
    \dimen0=8.5in
    \advance\dimen0 by -\dimen2         % 8.5in - line length
    \if@twoside
       \ifdim\dimen3=0in  % use defaults: 2/3 inside, 1/3 outside
          \divide\dimen0 by 3           % (8.5in-line length)/3
          \dimen2=2\dimen0              % (2/3)*(8.5in-line length)
          \oddsidemargin=\dimen2
          \advance\oddsidemargin by \dimen1   % add in offset
          \dimen2=\dimen0               % (8.5in-line length)/3
          \evensidemargin=\dimen2
          \advance\evensidemargin by \dimen1  % add in offset
%  allow for space on each side of marginal note
          \advance\dimen0 by -2\marginparsep
          \marginparwidth=\dimen0
       \else            % use supplied 2-sided value
          \oddsidemargin=\dimen3             % inside-margin-width
          \advance\oddsidemargin by \dimen1  % add in offset
          \advance\dimen0 by -\dimen3   % 8.5in-line length-inside margin
          \evensidemargin=\dimen0
          \advance\evensidemargin by \dimen1 % add in offset
%  allow for space on each side of marginal note
          \advance\dimen0 by -2\marginparsep
          \marginparwidth=\dimen0
       \fi
%  one-sided
    \else \ifdim\dimen3=0in  % use defaults: center text 
              \divide\dimen0 by 2         % (8.5in-line length)/2
              \oddsidemargin=\dimen0      % (8.5in-line length)/2
              \advance\oddsidemargin by \dimen1   % add in offset
              \evensidemargin=\dimen0     % (8.5in-line length)/2
              \advance\evensidemargin by \dimen1  % add in offset
%  allow for space on each side of marginal note
              \advance\dimen0 by -2\marginparsep
              \marginparwidth=\dimen0
          \else  % use supplied values
              \advance\dimen0 by -\dimen3  % 8.5in-line length-left margin
%  allow for space on each side of marginal note
              \advance\dimen0 by -2\marginparsep
              \marginparwidth=\dimen0
              \advance\dimen3 by \dimen1   % add in offset
              \oddsidemargin=\dimen3
              \evensidemargin=\dimen3
          \fi
    \fi
  \immediate\write16{Textwidth = \the\textwidth}
  \dimen0=1.0in
  \advance\dimen0 by \oddsidemargin
  \immediate\write16{Odd Page Left Margin = \the\dimen0}
  \dimen0=1.0in
  \advance\dimen0 by \evensidemargin
  \immediate\write16{Even Page Left Margin = \the\dimen0}
  \immediate\write16{Marginparwidth = \the\marginparwidth}
  \dimen0=#1
  \immediate\write16{Printer Offset = \the\dimen0}
 }
%
\def\@outputpage{\begingroup\catcode`\ =10 \if@specialpage 
     \global\@specialpagefalse\@nameuse{ps@\@specialstyle}\fi
     \if@twoside 
       \ifodd\count\z@ \let\@thehead\@oddhead \let\@thefoot\@oddfoot
                       \let\@themargin\oddsidemargin
% treat page 0 (title page) as if it is an odd-numbered page
        \else \ifnum\count\z@=0 \let\@thehead\@oddhead \let\@thefoot\@oddfoot
                                \let\@themargin\oddsidemargin
              \else \let\@thehead\@evenhead
                    \let\@thefoot\@evenfoot \let\@themargin\evensidemargin
     \fi\fi\fi
     \shipout
     \vbox{\normalsize \baselineskip\z@ \lineskip\z@
           \vskip \topmargin \moveright\@themargin
           \vbox{\setbox\@tempboxa
                   \vbox to\headheight{\vfil \hbox to\textwidth{\@thehead}}
                 \dp\@tempboxa\z@
                 \box\@tempboxa
                 \vskip \headsep
                 \box\@outputbox
                 \baselineskip\footskip
                 \hbox to\textwidth{\@thefoot}}}\global\@colht\textheight
           \endgroup\stepcounter{page}\let\firstmark\botmark}
%
\makeatother
%%
%% End of home:/usr/local/hacks/tex/setmargins.tex:
%%
\setmargins{6.5in}{1in}
\topmargin = 0in
\headheight = 0mm
\headsep = 0mm
\textheight = 9in
\makeindex
\begin{document}
\setlength{\parskip}{6 mm}
\setlength{\parindent}{0 in}

\title{A Quick and Dirty Guide to CLUE}
\author{Kerry Kimbrough}
\date{Version 6.0 \\
July, 1989\\[2 in]
\copyright 1989\  Texas Instruments Incorporated\\[.5in]
\parbox{3.5in}{
     Permission is granted to any individual or institution to use,
     copy, modify and distribute this document, provided that  this
     complete  copyright  and   permission  notice   is maintained,
     intact, in  all  copies  and  supporting documentation.  Texas
     Instruments Incorporated  makes  no  representations about the
     suitability of the software described herein for any  purpose.
     It is provided ``as is'' without express or implied warranty.
}}
\maketitle

This guide is a no-frills introduction to programming with  the Common Lisp User
Interface Environment (CLUE). CLUE is a high-level programming interface to the
X Window System which extends (but does not supersede) the basic CLX\index{CLX} interface.
CLUE is also an object-oriented programming system based on the Common Lisp
Object System (CLOS\index{CLOS}). Moreover, CLUE is a ``toolkit'' for constructing X user
interfaces. As a result, CLUE is modelled closely on the standard toolkit used
by C programmers, commonly known as Xt\index{Xt}, or the X Toolkit.

All of these related systems are described in separate documents (these
references are listed at the end of the guide), but we won't spend much time on
them here. Read this guide and take the plunge into CLUE; you can dry off with
the details later.

\section{The Big Picture}
Before you can understand CLUE, you have to look at three ideas about how
to build an interactive application. 


\subsection{The Application Is Event-Driven} 
An interactive application is controlled by a human user. Typically, nothing
happens until the user lifts his finger to the side of his mouse and causes
an input {\bf event}.  There are two implications here.
\index{event} 

First, the fundamental
application control structure is an {\bf event loop}: wait for an event, figure
out how to handle it, process the event, then go back and wait for the next one.
The {\em structure} of the event loop is generic; there is nothing
application-specific about it. 
\index{event, loop}

Second, the application is passive. It knows how to do certain things, but it's
waiting for the user's command to do them. You can imagine programs that do
things on their own, without waiting for user instruction (for example, a
nuclear reactor control system). But, the {\em interactive} part of such a
program is still an event loop, which must synchronize with the separate
background task(s) to provide the user control.


\subsection{The UI Is Separate} 
The user interface part has nothing to do with the essential function of the
application. An oft-cited analogy is linguistic communication, which has
independent modules for lexical, syntactic, and semantic processing. You can
understand me whether I speak to you or whether I write to you, because these
are merely two different lexical forms (e.g. sound patterns or marks) for the
same language.  Also, you can express the same idea either in French or in
English; the lexical and syntactic elements of each language are distinct from
their corresponding meaning, or semantics. 

An interactive application can be modularized along the same lines. The UI
contains the lexical and syntactic components, while the application {\em per
se} is the semantic component. Take some application command, say, ``Stop
Nuclear Reactor.'' The user interface to this command could use various lexical
forms (press the {\tt STOP} key or type the string ``{\tt stop}'') and
various syntactic forms (prompt with ``{\tt Are you sure you want to stop?}'' or
not), but the application semantics remain the same (hopefully!). From here on,
we will
acknowledge this modularization by referring to the whole application program as
``the program,'' to the user interface
components as ``the UI,'' and to the application semantic components as ``the
application.''

Separating the UI and semantic modules of an interactive program is good
programming technique. This allows you to more easily create a new UI for an
existing application. Or to apply an existing set of UI techniques to a new
application, so that it is consistent in style with others. Or to divide the
programming effort more efficiently; a domain expert can write the application
semantics while a human factors specialist can design the UI and a window system
hacker can implement it. 

\subsection{The UI Is Object-Oriented} This idea is more recent than the
other two, because it wasn't until a few years ago that graphical user
interfaces and window systems came along.  When they did, UI's began to
be composed of multiple user activities represented by visually-distinct
and physically-manipulable objects on the screen.  Object-oriented
programming (OOP) is thus a natural programming style which has been
associated with UI since its beginning\footnotemark\footnotetext{In
fact, true OOP and window systems were invented at the same time at
Xerox PARC in the Smalltalk system.}.  For several reasons, OOP is a
nice fit for UI programming.  UI's tend to generate profuse variation of
detail around a basic theme (consider, for example, menus); that is,
they demonstrate several subclasses of a basic class.  The modularity
of objects also helps to separate the UI from the semantics; that is,
the implementation of UI methods from the program which simply calls
them.

The OOP methodology leads to a UI which is composed of user interface objects
(command lines, messages, menus, scroll bars, dialogs, etc.). Each of these UI
objects is an ``agent'' for some portion of the  underlying application.
Each UI object is responsible for presenting some ``view'' of the application to
the user. It is also responsible for receiving the users events which are
directed to it and for translating these into the appropriate application
responses. 


\subsection{How CLUE Works\label{sec:how-clue-works}} 
CLUE combines all of these ideas. A CLUE program consists of an event loop, a
set of UI objects called {\bf contacts}, and a set of application functions
called {\bf callbacks}. All CLUE programs look pretty much like this:
\index{event, loop}

\begin{verbatim}
;; Open a connection to the X server.
(let ((display (open-contact-display 'my-application :host "server-host" ...)))
  ;; Initialize UI objects and callbacks.  
  (setf c (make-contact ... ))
  (add-callback c ... )
       ...
  ;; Process events until the event loop is terminated.
  (catch :event-loop
    (loop
      (process-next-event display)))  
  (close-display display))
\end{verbatim}

A CLUE program creates a {\bf contact-display} object which
\index{contact-display}
represents a connection to a specific X server.  This {\tt contact-display} then
becomes a two-way channel through which the program requests the creation and
display of contact windows and receives input events sent to its contacts. 

Each contact is a UI ``agent'' that is prepared to present some
application information, to accept user
events which manipulate this information, and then to report the results back to
the application. This reporting action is the critical connection between the UI
and the application, and it's done using {\bf callbacks}.\index{callback}
\index{callback, name}\index{callback, function}
A callback consists of
a {\bf callback name} and an associated {\bf callback function}. A contact is programmed
to associate a specific result with a callback name, which it reports by calling
the associated callback function (with a predefined argument list). 

Here is where CLUE contributes to the separation of UI and application. 
\index{application programmer}\index{contact programmer}
Every
CLUE program usually has two programmers! A {\bf contact programmer} is one
who defines a contact class and implements its methods. An {\bf application
programmer} is one who decides to create a contact instance and employ it for a
particular UI role. The contact programmer decides what callback names his
contact class will use and what the argument lists to the associated callback
functions have to be. The application programmer defines callback functions to
perform specific jobs and plugs them into the appropriate contact callback.
The contact programmer will frequently use the basic CLX\index{CLX} interface, typically to
display the contact's information. The application programmer will use CLUE
interfaces but will almost never need to call CLX functions directly.

Now we can see the operation of a CLUE program in a bit more detail. The {\tt
make-contact} function is used to create and initialize contacts. 
\index{make-contact}
The {\tt
add-callback} function \index{add-callback} associates application semantics
(i.e. callback
functions) with each contact. Then, the program enters the event loop. Each call
to {\tt process-next-event} \index{process-next-event} reads the next input event and dispatches it to the
contact to which it belongs. This contact then goes through a process of {\bf
event translation}\index{event translation}, which determines which
contact {\bf action}\index{action} functions will be
executed. Contact action functions, in turn, perform all the details of input
feedback and display. Finally, if the user event has completed an application
result, the action function will look up and call the associated callback
function. What has happened? CLUE has provided all the machinery, the contact
programmer has defined the detailed UI behavior, and the application programmer has
furnished the real results.

The rest of this guide will be a closer look at each of these parts of a CLUE
program.

\section{The Contact Display\label{sec:contact-display}}
\index{contact-display}
\index{open-contact-display} 
A {\tt contact-display} is created by the {\tt
open-contact-display} function.  {\tt op\-en\--con\-tact\-dis\-play} has one required
argument: a symbol which acts as the name of your program.  Technically, this is
a {\bf resource name}\index{resource, name} which is used to access the values of
{\bf resources}\index{resource} associated with your program.  CLUE resource
management is explained later in Section~\ref{sec:resources}.  Generally
speaking, this program resource name is used to distinguish your program from
other application programs that may also be running. {\tt open-contact-display}
has several optional keyword arguments (mainly the same ones used by the CLX\index{CLX}
{\tt open-display} function), but the most frequently used is the {\tt host}, a
string which gives the network name for the X server host to which you are
connecting.

\begin{verbatim}
(setf display
      (open-contact-display
        'blink                    ; Application name
        :host "server-host"))     ; Server host name
\end{verbatim}

{\tt contact-display} objects are really a subclass of the {\tt display} data
type defined by CLX\index{CLX}\footnotemark\footnotetext{However, neither {\tt display}
objects nor {\tt contact-display} objects are actually implemented as CLOS
classes. There's no reason to define {\tt contact-display} methods or
subclasses.}.
\index{display}

\begin{center}
\framebox[5.5in]{
\hspace*{\fill}
{\bf Note}
\hspace*{\fill}
\parbox[t]{4.5in}{
Any CLX\index{CLX} function which takes a {\tt display} argument can be called with a {\tt
contact-display} argument instead.}
\hspace*{\fill}}
\end{center}

The CLX function {\tt close-display}\index{close-display} should always be used
to close the server connection when the program terminates.

\section{Contacts\label{sec:contacts}}
In simplest terms, a {\tt contact} is a kind of window. The behavior of a window
object,
which is represented in CLX\index{CLX} as an instance of the
{\tt window} data type,
is completely specified by the X Window System
protocol.  In CLUE, the class of {\tt contact} objects is defined to be
a subclass of the class of {\tt window} objects\footnotemark\footnotetext{To do
this, CLUE actually changes the definition of the CLX {\tt window} type to
make it a CLOS class.}.  Thus, all CLX {\tt window} functions
apply to any {\tt contact}. 

\begin{center}
\framebox[5.5in]{
\hspace*{\fill}
{\bf Note}
\hspace*{\fill}
\parbox[t]{4.5in}{
Any \index{CLX}CLX function which takes a {\tt window} argument can be called with a {\tt
contact} argument instead.}
\hspace*{\fill}}
\end{center}


As a window, a contact inherits several basic attributes,
such as a
rectangular geometry (given by the position of its upper left corner, its width,
and its height in
pixel coordinates), a border width and border color/pattern, a depth (i.e.
bits per pixel), and a background color/pattern.
\index{contact, attributes}
Contacts possess a few additional attributes (or {\bf slots}) not shared
by ordinary windows. 
\begin{itemize}
\index{contact, name}
\item {\tt name}: The name of
the contact is a
symbol  which can be used to access contact resources
stored in a resource database. See Section~\ref{sec:resources}.

\index{contact, parent}
\item {\tt parent}: All
contacts have a parent contact (except for a {\bf root}\index{root} contact, which is
created automatically by CLUE and which is the ancestor of all contacts on a
single display screen).  The parent-child relationship
is fundamentally one between two windows, as defined by the X
Window System. For example, a child window is always visually
contained
by its parent window, and its position is specified relative to its
parent's upper-left corner. However, in CLUE this relationship is
extended, with the parent assuming additional responsibilities, such as
geometry management (see Section~\ref{sec:composites}).

\index{contact, state}
\item {\tt state}: A switch which controls the visual effect of the contact on
the UI. If the {\tt state} is {\tt :withdrawn}, then the contact is invisible and
unavailable for input. If the {\tt state} is {\tt :mapped}, then the contact is
``viewable'' (i.e. visible unless it is covered by other windows) and available
for input. The {\tt state} can also have a third value --- {\tt :managed} ---
which is discussed in
Section~\ref{sec:resources}. The default {\tt state} is {\tt :mapped}.


\index{contact, event-translations}
\item {\tt event-translations}: A list which associates
different types of input events with contact actions. This is discussed in more
detail in Section~\ref{sec:events}.

\index{contact, callbacks}
\item {\tt callbacks}: An association list of callback name/function pairs.
\index{callback}
\end{itemize}

\subsection{Using Contacts}
\index{contact, application programmer interface}
If you are an application programmer, your basic view of a contact --- the generic
contact protocol --- is quite simple and consists of only a few functions.
\index{contact, generic protocol}

{\tt make-contact} 
\index{make-contact}
creates a new instance of any con\-tact subclass. This function is a
syn\-tac\-ti\-cal\-ly-identical extension of the basic CLOS\index{CLOS} {\tt make-instance}
function. Keyword arguments to  {\tt make-contact}  provide initial slot values
for the new instance, most of which can be defaulted. However, a {\tt parent}
argument must be given; specifying a {\tt contact-display} as the {\tt parent}
creates a {\bf top-level contact}\index{contact, top-level}. 

\begin{center}
\framebox[5.5in]{
\hspace*{\fill}
{\bf Note}
\hspace*{\fill}
\parbox[t]{4.5in}{
A contact ``belongs'' to the {\tt contact-display} connection on which it is
created. More precisely, since a contact must have a top-level ancestor, it
belongs to the {\tt contact-display} given as this ancestor's {\tt parent}.
This {\tt contact-display} connection receives all requests and events related
to the contact; it is also available as the value of the contact's {\tt display}
slot.} \hspace*{\fill}} \end{center}


Usually (and always
for top-level contacts), you must also specify the contact's {\tt width} and {\tt
height}. A non-top-level contact may find that its initial values for 
{\tt x}, {\tt y}, {\tt width}, {\tt
height}, and {\tt border-width} have been modified by the geometry management
policy of its parent (see Section~\ref{sec:composites}).

\begin{verbatim}
(setf a-blinker
      (make-contact
        'blinker                  ; Make an instance of the blinker class
        :parent display           ; Required argument
        :width  300               ; Usually required
        :height 400               ; Usually required
        :background 0))           ; Optional, defaults to parent's
\end{verbatim}



The {\tt (setf contact-state)} function 
changes the {\tt state} value of the contact.
\index{contact, state}

\begin{verbatim}
(setf (contact-state a-blinker) :mapped)    ; Make it viewable
(setf (contact-state a-blinker) :withdrawn) ; Make it invisible
\end{verbatim}

{\tt change-geometry} 
\index{change-geometry} 
requests a change to one or more components of the contact's geometry.
The keyword
arguments to {\tt change-geometry} --- {\tt x}, {\tt y}, {\tt width}, {\tt
height}, and {\tt border-width} --- specify the changed component(s); omitting a
keyword means ``Leave the current value unchanged.'' But the actual effect of
the request depends on the geometry management\index{geometry management}
policy applied to the contact (See Section~\ref{sec:composites}). If the
request is refused, then the return values give  acceptable alternatives for {\tt x}, {\tt y}, {\tt width}, {\tt
height}, and {\tt border-width}.  You may want to give an {\tt accept-p} keyword
value of {\tt t}, which means ``Go ahead and replace my request with whatever
alternative geometry may be returned.''

\begin{verbatim}
(change-geometry a-blinker
                 :x 100           ; Request new upper-left position...
                 :y 200
                 :accept-p t)     ; ...but accept closest alternative

(change-geometry b :accept-p t)   ; Just validate current values
\end{verbatim}

\begin{center}
\framebox[5.5in]{
\hspace*{\fill}
{\bf Note}
\hspace*{\fill}
\parbox[t]{4.5in}{
There are no accessor functions for {\tt setf}'ing a contact's {\tt x}, {\tt y}, {\tt width}, {\tt
height}, and {\tt bor\-der\--width} slots. Never modify these slots directly. Always
use {\tt change-geometry} instead.} 
\hspace*{\fill}}
\end{center}


The {\tt change-priority} function \index{change-priority}
requests a change to the contact's stacking priority relative to other
windows.  The stacking priority determines which window is ``on top'' of other
windows. This request is also subject to geometry management policy. 

The {\tt destroy} function is called \index{destroy}
only when the contact will no longer be referenced. This frees any display
resources allocated to the contact.

In addition to the generic protocol, the application programmer interface to a
contact includes its callback protocol, plus any special class-dependent functions (typically, for
initialization).

\subsection{Defining Contacts}

If you are a contact programmer, you will use the {\tt defcontact} macro to
define a new contact class.  The syntax of {\tt defcontact} is nearly identical
to that of the basic CLOS\index{CLOS} {\tt defclass} macro.  The only difference is the
additional {\tt :resources} option for specifying contact resources (see
Section~\ref{sec:resources}).

\begin{verbatim}
(defcontact
  blinker                         ; Class name  
  (contact)                       ; Superclasses  
  ((color                         ; Slot specs, with name...
     :type     pixel              ; ...data type...
     :accessor blinker-color      ; ...accessor function name ...
     :initarg  :color             ; ...initarg keyword for make-contact...
     :initform 0)                 ; ...and default initial value.
   (on-p
     :type     boolean
     :accessor blinker-on-p
     :initform nil))  
  (:resources color))             ; Resource specs
\end{verbatim}

A contact programmer must also define a {\tt display} method for a new contact
class. CLUE calls the {\tt display} function automatically whenever any portion
\index{display}\index{contact, exposure}\index{exposure}
of the contact image must be (re)displayed. In particular, {\tt display} is
called whenever an invisible contact is changed to the {\tt :mapped} state, or
when some previously-hidden part of a {\tt :mapped} contact is exposed.

\begin{verbatim}
(defmethod                        ; Define a method ...
  display                         ; ... for the display function...
  ((contact blinker)              ; ... when the contact arg is a blinker.
   &optional x y width height)    ; Defines the rectangular piece exposed.
  ...)
\end{verbatim}

\section{Composites and Geometry Management\label{sec:composites}}
A contact which is the parent of another contact is known as a
{\bf composite}\index{composite} and is an instance of the {\tt composite}
subclass of {\tt contact} objects. A {\tt composite} may be the parent of another
{\tt composite},
leading to a tree-structured contact hierarchy. A contact is said to be an
{\bf ancestor}\index{ancestor} of another contact (its 
{\bf descendant}\index{descendant}) when
it is its parent or an ancestor of its parent. 

A {\tt composite} represents a set of contacts which can be manipulated
(positioned, presented,
etc.) as a unit.  A {\tt composite} is useful whenever several contacts act in
concert to provide a
single component of the UI.  Typical examples include
``control panels'' and ``dialog boxes'' --- groups of contacts that are presented
together and are used to make related adjustments to application data.  
The fundamental aspects of the contact parent-child relationship are the same as
those of the window hierarchy defined by the X Window System. 
\index{geometry management}
But in CLUE, a {\tt composite} is also expected to act as the {\bf geometry
manager}\index{geometry manager} for its child contacts; that is, to
implement a style of layout for its children. 


Here is how it works.  A request to change the geometry of a contact is
forwarded to the contact's parent, which actually performs the resulting change.
It is important to understand that, due to its constraints, a geometry manager
may not be able to perform a change as requested.  For example, a request to
increase the size of a contact might be refused if its geometry manager enforces
a maximum size.  Even if a requested change cannot be done, a geometry manager
may be able to suggest a slightly different change which would be acceptable.

Placing geometry control in the hands of a geometry manager in this way
has several advantages.
\begin{itemize}
\item   A geometry manager can arbitrate the competing geometry change requests
        of several contacts in order to implement contraints among them.
        
\item   A given layout style can be applied to any collection of contacts.
        
\item   Contact layout style can be changed without the knowledge of individual
        contacts.
\end{itemize}


A composite's geometry management policy applies only to the set of its children
which are {\bf managed}. \index{contact, managed}
An unmanaged child is ignored by its geometry manager. Any
geometry change to an unmanaged child is performed immediately as requested. 
However, changes to a managed child are arbitrated by its parent's geometry
management policy. This can mean, for example, that a change to one child's
position/size/priority can affect that of other children. Indeed, requesting a
bigger size for a child might result in the parent asking {\em its} geometry
manager to become bigger, leading to a ripple of geometry changes throughout
the contact hierarchy.

How do you know if a contact is managed? This is determined by its {\tt
state}.\index{contact, state}
A contact is managed if and only if its {\tt state} is not {\tt :withdrawn}. This means
that a {\tt :mapped} (and viewable) contact is managed. But there is also
another possibility: a contact's {\tt state} may also be {\tt :managed}. This
value represents the (rather rare) case of a contact which is not visible, but
which is nevertheless taken into account by its geometry manager. 

\begin{center}
\framebox[5.5in]{
\hspace*{\fill}
{\bf Note}
\hspace*{\fill}
\parbox[t]{4.5in}{
An
unmanaged child can {\em never} be visible.}
\hspace*{\fill}}
\end{center}


Implementing geometry management is a topic for contact programmers only. If you
define a new {\tt composite} subclass, you will usually need to implement
methods for the following three functions.
\begin{itemize} 
\item {\tt change-layout}: 
\index{change-layout}
This function is called whenever a composite's set of
managed children changes (e.g. by creating or destroying a managed child or by
changing the {\tt state} of a child).

\item {\tt manage-geometry}: 
\index{manage-geometry}
This function is called by {\tt change-geometry} to
approve a geometry request. It must either approve the change or return a set of
alternative geometry values which will be acceptable.

\item {\tt manage-priority}: 
\index{manage-priority}
This function is called by {\tt change-priority} to
approve a stacking priority request. It must either approve the change or return
an alternative priority which will be acceptable.

\end{itemize}

\section{Events\label{sec:events}}
\index{event}\index{input}
Handling input events is CLUE's main job, and it's an area where CLUE does a lot
of the work for programmers automatically. Application programmers don't have to
worry about much more than setting up an event loop, as shown in
Section~\ref{sec:how-clue-works}. Contact programmers need to know how to use
CLUE's {\bf event translation}\index{event translation} mechanism in order to
implement the details of a contact's input behavior.

What exactly are these events we're talking about? Fundamentally, an event is
a message to the program from the X server, a packet of data describing
some occurrence that the program ought to be interested in.
Most events
represent an action performed by the user with  the X server's 
\index{input, devices}
keyboard or pointing device (i.e. mouse, tablet,
etc.). Keyboard
actions, for example, can generate {\tt :key-press} and {\tt :key-release}
events. The pointer may be used to generate {\tt :button-press} and {\tt
:button-release} (when the user presses one of the mouse buttons) and {\tt
:motion-notify} events (when the user changes the pointer position). These kind of
{\bf user events}\index{event, user} contain data for the button or key
involved, the coordinates of the pointer position, and other useful things, such as
the up/down state of {\tt :shift}, {\tt :control}, and the other modifier keys.
In addition to user events, there is another important group of {\bf side-effect
events},\index{event, side-effect}
events which occur as an indirect result of other user actions. For example,
when the user causes part of an obscured window to become visible, an {\tt
:exposure}\index{exposure} event may be sent. Other side-effect events include
{\tt :enter-notify}, which signifies that the user has moved the pointer cursor
inside a particular window, and {\tt :focus-in}, which happens when the user has
identified a particular window as the {\bf focus} for keyboard
events.\index{focus}\index{input, focus}
This is just the beginning, but the definition and meaning for all of the events
received by CLUE contacts can be found by reading the X Protocol Specification.


To understand CLUE event processing, let's start with the basic program event
loop\index{event, loop} and follow its operation, step by step.
A call to {\tt
process-next-event} \index{process-next-event}\index{timeout}
gets the ball rolling. {\tt
process-next-event} causes CLUE to read
the next event from the given {\tt contact-display} connection. By default,
{\tt pro\-cess\--next\--ev\-ent} does not return until the next event has been
completely processed. This means that if no event is yet available, then {\tt
process-next-event} will generally wait until one finally comes along. However,
an optional {\tt timeout} argument may be given, which says how long {\tt
process-next-event} will wait before giving up. Set the {\tt timeout} to {\tt 0}
if you don't want {\tt process-next-event} to wait at all. {\tt
process-next-event} returns {\tt nil} if a timeout occurred; otherwise, it
returns {\tt t}.

\begin{verbatim}
(catch :event-loop
  (loop
    (unless 
      (process-next-event display 5)  ; Timeout after waiting 5 sec.

      ;; No events yet -- do something useful
      (do-background-task))))
\end{verbatim}

\begin{center}
\framebox[5.5in]{
\hspace*{\fill}
{\bf Note}
\hspace*{\fill}
\parbox[t]{4.5in}{
{\tt process-next-event} does not return any result of the actual processing of
an event. This means
that all responses of a CLUE program to user input  --- even the termination
of the event loop --- occur as ``side-effects'' of contact event handling.
Always escape the event loop by {\tt throw}'ing to some well-defined tag.}
\hspace*{\fill}}
\end{center}

Inside {\tt process-next-event}, the next thing that happens is that the event is
{\bf dispatched}\index{event,
dispatching}. In other words, CLUE figures out which contact is supposed to
handle the event. This is fairly easy because almost every event message
contains an identifier for the {\tt window} object which is the
``addressee'' of the
event. CLUE knows how to convert this identifier into the corresponding {\tt
contact} object. At this point, the event
is {\bf handled}\index{event, handling} by calling the {\tt handle-event}
function\index{handle-event} with the receiving {\tt contact} and the event as
its arguments. {\tt handle-event} then implements the process of event
translation, i.e. figuring out which methods of the receiving {\tt contact} to
call in response to the event\footnotemark\footnotetext{{\tt
handle-event} is also the function which automatically calls the
contact's {\tt display} method, if the event is an {\tt :exposure}.\index{display}\index{exposure}}.
What happens next is of interest only to contact programmers.

\subsection{Event Translation}
Event translation involves searching through one or more association lists,
looking for an entry that matches the event. Each such entry is called an {\bf
event translation}, and it is a list containing an {\bf event
specification}\index{event, specification} and one or more {\bf
action}\index{action} names. An event specification describes a certain sort of
event, and an action name is simply the name of a special kind of function
--- 
an action function. So, during event translation, the event is compared with an
event
specification, and when it matches, then each of the corresponding action functions are
invoked in sequence. When all of them have completed, then event
translation is done; {\tt handle-event} returns and the event is said to
be ``handled.'' The {\tt process-next-event} function also returns, and we
go back to the top of the event loop to deal with the next event.
But where are these event translations and how are they created? There are two
different ways to create event translations: the {\tt add-event} function and
the {\tt defevent} macro.

{\tt add-event}\index{add-event}
creates an event translation and adds it to the association list
found in the {\tt event-translations} slot of a specific contact. This kind of
event translation is thus an {\bf instance event
translation}\index{event translation,
instance} which affects only one contact instance. An instance event
translation says ``When an event is dispatched to {\em this} contact and it
matches {\em this} event specification, then call {\em these} actions.''

\begin{verbatim}
(add-event a-blinker                 ; When this contact gets...
           '(:key-press #\ctrl-b)    ; ...this character event ...
           'beep)                    ; ...call the beep action.

(add-event a-blinker                 ; When this contact gets...
           '(:key-release #\ctrl-b)  ; ...this character event ...
           '(beep 1))                ; ...call the beep action with an arg.
\end{verbatim}

Note that in place of a simple action name, an event translation can also give a
list containing the action name plus a list of arguments to pass.
The process of event translation starts with the instance event translations in
the {\tt event-translations} slot of the receiving contact. Each entry is
examined in order, until a match is found. What happens if no match is found? In
this case, the process continues to search the event translations defined by the
{\tt defevent} macro.

{\tt defevent}\index{defevent}
creates a  {\bf class event translation}\index{event translation, class}
that applies to every instance of a given class.  A class event
translation says  ``When an event is dispatched to a contact of {\em this} class
and it matches {\em this} event specification, then call {\em these} actions.''

\begin{verbatim}
(defevent blinker                    ; When any blinker gets...
          (:button-press :button-1)  ; ...this button event...
          (blink t))                 ; ...blink on.

(defevent blinker                    ; When any blinker gets...
          (:button-release :button-1); ...this button event...
          (blink nil))               ; ...blink off.
\end{verbatim}

Class event translations are searched for each contact superclass of the
receiving contact, starting with the class of the contact, then continuing back
up the {\bf class precedence list}\index{class precedence list} which this
contact inherits\footnotemark\footnotetext{See the CLOS\index{CLOS} specification for a precise definition of the
class precedence list.}.  The idea is to match a class event translation
for the
most specific class possible.  However, if a matching event translation is {\em
still} not found, then CLUE gives up, and the event is ignored. After
all this, it couldn't be a very important event anyway!

\subsection{Event Specifications}
\index{event, specification}
As you may have noticed, an event specification is usually a list
composed of an event type keyword and some other qualifiers. In fact, an event
type keyword alone is also a valid event specification.
The exact syntax for event specifications is fairly complex. A sophisticated
programmer can even define his own event specification syntax. All of this is
completely spelled out in the CLUE specification. But for now, you can get an
idea of what an event specification is by looking at the following examples.

\begin{verbatim} 
:enter-notify         ; Matches any :enter-notify event

(:motion-notify       ; Matches a :motion-notify event..
  :button-1)          ; ...if :button-1 is down.

(:button-press        ; Matches a :button-press...
  :button-1           ; ...if it's :button-1...
  :shift)             ; ...and the shift key is down.

(:button-release      ; Matches a :button-release...
  :button-3           ; ...if it's :button-3...
  (:shift :control)   ; ...and the shift,control keys are down...
  :all)               ; ...and all other modifiers are up.

(:key-press           ; Matches a :key-press...
  :any                ; ...for any key...
  (:control :hyper))  ; ...when the control,hyper keys are down.

(:button-press        ; Matches a :button-press...
  :button-3           ; ...if it's :button-3...
  :double-click)      ; ...and it's a double-click.
\end{verbatim}

\begin{center}
\framebox[5.5in]{
\hspace*{\fill}
{\bf Note}
\hspace*{\fill}
\parbox[t]{4.5in}{A contact will receive {\em only} the types of events described
by
its event specifications, i.e. by event specifications in its instance and class
event translations. CLUE automatically sets the 
{\bf event mask}\index{event, mask} of the contact window, based on the
contact's event specifications. Event masks are a detail of CLUE internals, but
you can find more information about them in the X Protocol
Specification and the CLUE specification.}\hspace*{\fill}} \end{center}

\subsection{Actions and Callbacks}
\index{action}
An action is a  function which is designed to handle an input
event.  An action therefore implements a particular input behavior exhibited by
each instance of a certain contact class. 
In fact, an action function is typically a CLOS generic function, and actions
usually are defined as methods of a particular contact class. 

\begin{verbatim} 
(defmethod blink ((contact blinker) blink-on-p)
  (with-slots (on-p) contact
    ;; Set internal state variable on/off
    (setf on-p blink-on-p)
    ;; Redisplay based on new on-p state
    (display contact)
    ;; Invoke callback with new state
    (apply-callback contact :blink on-p))
\end{verbatim}

In general, an action represents a well-defined contact behavior that
could be done in
response to {\em any} event. In fact, it's possible to personalize an existing UI by
modifying its event translations so that contact actions are rebound to
the types of events (i.e. event specifications) which
are more to your taste.  In some cases,
however, an action may be designed to handle a very specific event type.

When an action computes a result that is important to an application, it invokes
a callback.\index{callback}
This is done via the {\tt apply-callback} macro.\index{apply-callback}
The example above shows that a {\tt blinker} contact has a callback whose name
is {\tt :blink}, which is called with a single argument, i.e. a boolean
representing the new on/off state of the {\tt blinker}. What significance
this has to the application is unknown to the contact programmer who wrote this
action; it all depends on what callback function the application programmer has
associated with {\tt :blink} for this {\tt blinker} instance.
In fact, the application programmer may have decided not to define a callback
function for {\tt :blink}, in which case {\tt :blink} has no application meaning
at all (and {\tt apply-callback} simply does nothing).
As shown in the {\tt :blink} example,
{\tt apply-callback} is usually (although not
necessarily)
called directly by an action method or somewhere within its dynamic extent.

\begin{center}
\framebox[5.5in]{
\hspace*{\fill}
{\bf Note}
\hspace*{\fill}
\parbox[t]{4.5in}{
Callback names are shared by a class, i.e. referenced by the class's
action methods.  But
callback name/function pairs are instance data and are recorded in a contact's
{\tt callbacks} slot. In our example, every {\tt blinker} is
expected to have a {\tt :blink} callback function, but each {\tt blinker}
usually has a different {\tt :blink} callback function. That is, each {\tt
blinker} usually has different application semantics.} \hspace*{\fill}}
\end{center}

In the previous example, the {\tt blink} action method does not depend on the
event which causes it to be invoked.  But what happens when an action's behavior
depends on information contained in the event?  CLUE handles this by
representing an event as an instance of the {\tt event} class.  An {\tt event}
object has slots which contact various kind of interesting information about the
event.  Since there are many types of X events, there are many different {\tt
event} slots, even though for a given event some slots are irrelevant and
therefore have a {\tt nil} value. You can look in the X Protocol Specification for
a complete description of event slots (and look in the CLX specification to find
out how these slot values are represented in Lisp).

But how does an action access an event object? This is done by using a special
CLUE macro called {\tt with-event}. For example:

\begin{verbatim}
(defmethod beep ((contact blinker) &optional (per-cent-volume 0))
  (with-event (state)
    ;; Was the shift key down?
    (when (member :shift (make-state-keys state))
      ;; Ring server's chime!
      (bell (contact-display contact) per-cent-volume)  
      ;; Invoke callback
      (apply-callback contact :beep))))
\end{verbatim}

The {\tt beep} action will beep only if it determines that the shift key was
down when the event occurred.  It does this by examining the {\tt state} slot of
the event, which defines which modifier keys were pressed at the time of the
event.  Notice the use of the {\tt with-event} macro.\index{with-event} This is
similar to the {\tt with-slots} macro of CLOS\index{CLOS}.  {\tt with-event}
binds slots of the ``current'' event argument within its lexical extent, so that
the action code can refer to them.  But why is the current event object hidden
in this way?  Because this allows for a much more efficient implementation of
{\tt event} objects than would be possible if their structure was fully exposed
to programmers.  So, there!

\subsection{Advanced Input}
\index{input}
Input events are a subject so dear to the heart of CLUE that it
contains many more input programming functions than can be
covered in this guide. Here's a quick list of CLUE's special input
features, all of which are discussed in complete detail in the CLUE
specification.
\begin{itemize}
\item {\bf Before actions},\index{before action}\index{action, before}
action functions which can be set to execute for {\em every} event
dispatched to a contact of a certain class, before event translation begins. 

\item {\bf Timers},\index{timer}
which send special {\tt :timer} events to a contact at a regular,
specified rate.

\item {\bf Synchronizing}\index{event, synchronizing}
event processing, so that a program can stop to process all pending
events before continuing.

\item {\bf
Sensitivity},\index{contact,
sensitivity} which allows a contact's input to be temporarily ``switched
off'' without changing its visibility.

\item {\bf Event compression},\index{event, compression}
which removes certain redundant events for faster performance.
\end{itemize}

\section{Resources\label{sec:resources}}
Imagine that you are the user of an interactive program. Maybe you are
not a programmer, but there are certain things about the UI that you
wish you could change ever so slightly. For example, maybe you wish you
could make the program use a different set of colors for various
objects. Or maybe you're left-handed, and you'd prefer to use the right
mouse button in place of the left one. Or maybe you can't read Japanese,
which happens to be the language all the menu items are written in.
Although most of the UI is defined by programmers  --- contact
programmers who designed the various UI objects and application
programmers who placed them neatly on the screen --- the truth is that
some aspects of the UI ought to be up to the {\em user}. Programmers,
keep your hands off!

Nowadays, thoughtful UI programmers recognize this problem, so CLUE
allows  programmers and users to cooperate in
defining the UI. The basis for this cooperation is {\bf resource
management}. A user can store various UI values as {\bf
resources}\index{resource} in a {\bf resource database}\index{resource,
database}, and CLUE contacts can read these resource values and modify
the UI accordingly. 

\subsection{The Resource Database}
Before a user can assign a resource value to a particular UI object, he
must have a way of identifying it, some kind of  name which
can be referenced outside of the program which creates the object.
Remember that a contact has a {\tt name} slot\index{contact, name}; this
is a symbol that could certainly help identify the contact. Consider
also that a contact is generally part of a nested hierarchy of UI
objects. For example, a paint program might have pattern choice
contact which contains several button contact children, one for each available
pattern style. A more complete name for a button child would be a
list which also includes its parent's name, e.g. {\tt '(pattern-choice
checkered)}. We can extend this notion all the way up the hierarchy to
come up with a {\bf complete resource name}\index{resource, complete
name} for the {\tt checkered} pattern button. This would be a list of
resource name symbols, starting with the resource name for the program
itself (i.e. the required argument to {\tt
open-contact-display}\index{open-contact-display}) and continuing with
symbols for the {\tt root}\index{root} ancestor, the top-level ancestor,
and so on:\index{resource, name}

\begin{verbatim}
    '(paint screen-1 pattern-choice checkered)
\end{verbatim}

Assuming that the {\tt checkered} button has a resource for its label
string named {\tt label} and another resource for the label font named
{\tt font,}
then the complete names for these resources would obviously be:


\begin{verbatim}
    '(paint screen-1 pattern-choice checkered label)
    '(paint screen-1 pattern-choice checkered font)
\end{verbatim}

As you might expect when dealing with long path names like these, it's
often useful to insert ``wildcards'' in the places where you don't want
to be so specific. For example, the following resource name would refer to
the {\tt font} of {\em everything} in the pattern choice contact, regardless of
the screen where it appears:

\begin{verbatim}
    '(paint * pattern-choice * font)
\end{verbatim}

Another way of identifying a contact is by its object class. So, for
example, you could refer to the font of {\em every} {\tt button} object in the
paint program with the following resource name.

\begin{verbatim}
    '(paint * button font)
\end{verbatim}

By associating values with these sorts of resource names in {\bf
resource bindings}\index{resource, binding}, a user can
create a resource database which represents his UI preferences.
CLUE\index{CLUE} provides a {\tt define-resources} macro for adding
resource bindings to the database.\index{define-resources}
\begin{verbatim}
(define-resources
  (paint * button background)               white
  (paint * pattern-choice * font)           helvetica-12
  (paint * pattern-choice checkered label) "Checkered")
\end{verbatim}

CLUE automatically reads the resource database whenever a contact is
created. A contact has a class and possibly other contact superclasses
on its class precedence list\index{class precedence list}. Each such
(super)class has a list of resource names that it uses (defined by the
{\tt :resources} option of the defining {\tt
defcontact}\index{defcontact}
form). Together, these form the set of contact resources that are looked
up in the resource database. Of course, finding a resource value is rather complicated,
because names in the database may not be complete resource
names and may contain a mixture of resource names and class symbols.
Nevertheless, during initialization,
CLUE will find the resource binding that is the closest match for each
contact resource. In general, a contact gets its resource values  first
from the user's resource database, then from programmer-specified
defaults when no user value is found.

\begin{center}
\framebox[5.5in]{
\hspace*{\fill}
{\bf Note}
\hspace*{\fill}
\parbox[t]{4.5in}{
The contact programmer decides which values a contact will look up in
the resource database; the user decides what values it will find there.
The contact programmer identifies resources by listing them in the {\tt
:resources} option of the {\tt defcontact} form for his contact class.\index{defcontact}}
\hspace*{\fill}} \end{center}

\subsection{Converting Resource Values} 
\index{resource, conversion}
\index{resource, types}
Another convenient feature of
CLUE resource management is automatic resource type conversion.  Often
the type of value specified by the user in the resource database is not
the data type actually used by the program.  For example, a user might
identify a font by a string such as ``{\tt helvetica},'' but the program
must convert this name into a CLX {\tt font} object before text is
displayed.  Another example is color: where a user might specify ``{\tt
red},'' a program must somehow determine a colormap and pixel value that
will yield this hue.

CLUE automatically converts user values out of the user's resource
database into the correct target data type (as specified in the {\tt
:resources} option of a {\tt defcontact}\index{defcontact} form). This
is done by calling the {\tt convert} function\index{convert}. For
example:

\begin{verbatim}
(setf color (convert         ; Convert a resource...
              a-blinker      ; ...for a blinker...            
              "red"          ; ...from a string value...              
              'pixel))       ; ...to a pixel target type.
\end{verbatim}

CLUE defines a number of {\tt convert} methods for various combinations
of source and target data types. Contact programmers can extend
this mechanism by defining their own special {\tt convert} methods.

\section{Programming Tips\label{sec:tips}}
\subsection*{Managing the Server Connection}
You'll get into trouble if you aren't careful to close down the X server
connection when your program terminates. The reason? Most X servers have a limit
to the number of client connections which they can serve at one time. When the
limit is reached, the server simply refuses to open any new connections, and
you'll end up in the error handler staring at an ``Unable to connect'' message.  Make
sure your program
{\em always} terminates with a call to {\tt close-display},\index{close-display}
even when it aborts unexpectedly. The best way is to put your event loop inside
an {\tt unwind-protect} form:

\begin{verbatim}
...
(unwind-protect 
  (catch :event-loop
    (loop
      (process-next-event display)))  
  (close-display display))
...
\end{verbatim}

One nifty thing about using this technique is that it's always safe to
abort out of a damaged program and start over.

Also, you should generally avoid binding {\em global} special variables
to {\tt contact-display} objects representing open server connections. It's too
easy to lose track of a global variable or, even worse, to garbage a {\tt
contact-display} while it's still open.\index{contact-display}

\subsection*{Debugging}
\index{debugging}
Debugging a CLUE program (or, indeed, any program using the X Window System)
requires an awareness of a simple fact of life: {\bf client-server communication
is buffered}. Just calling the CLX {\tt draw-line} function won't necessarily
cause a line to appear on the screen. Instead, CLX places the corresponding
server request into an output buffer and moves on; the request is sent to the
server and executed later, when the output buffer is flushed. Normally, you
don't have to worry about this because CLX will automatically flush the output
buffer at the ``right'' time\footnotemark\footnotetext{When the output buffer is
full and (by default) before reading the next input event.}. However, during
debugging, you may want requests to be executed immediately. There are
two ways to do this.

\begin{enumerate}
\item Call {\tt display-force-output} manually at the appropriate time.
\index{display-force-output}
\item Invoke {\tt (setf (dis\-play-af\-ter-func\-tion dis\-play)
\#'dis\-play-force-out\-put)}. \\
This tells CLX to flush the
output buffer automatically as soon as each request
is made.
\index{display-after-function} 
\end{enumerate}

\index{errors}
Output buffering also means that when a request is in error, you won't
see the error immediately. The X server reports an error by sending an {\bf
error reply} back to the program (in this sense, an error reply is another kind
of input ``event'').
The error will be reported after the invalid
request is sent to the server, but by this time, your program has usually sent a
number of other requests, too. The result? When you arrive at the error handler,
you almost never be anywhere close to the real scene of the crime.
Using the {\tt display-after-function}\index{display-after-function} for
``single-step'' output, as shown above, can help to mitigate this problem. But
at any rate, you will need to pay careful attention to the contents of the error
reply which is printed out in CLX's error message. Sometimes, your best recourse
is to compare the error
reply with the error behavior defined by the X Protocol Specification for the
offending request and thereby deduce the cause of the bug.

Another useful debugging technique is to (locally) bind your {\tt
contact-display} and important {\tt contact} objects to special variables within
your program. This makes it easier to {\tt break} at a convenient
point (say, while the program is idle, waiting for an event to occur) and then
examine the state of various contacts or make server requests to return
interesting data.

\section{Getting Started\label{sec:getting-started}}
\subsection{Building CLUE}
If you're an Explorer user, you'll want to use an Explorer Release 6 load
band. This band has the latest versions of CLOS, CLX, and CLUE already built in.
If this applies to you, you can stop here --- you're ready to go!

But if your system doesn't come with CLUE built in, you'll need to go through
the following procedures to get things properly loaded.
Three systems need to be loaded: CLOS\index{CLOS}, CLX\index{CLX}, and CLUE
itself. CLX needs to be loaded specially  --- a small patch to the standard X11
R3 version of CLX is needed to make it work with CLUE.

\begin{enumerate}
\item Decide which version of CLOS you want. Possible choices include:

\begin{itemize}
\item {\tt PCL}, the Portable Common Loops implementation from Xerox that has
recently been made compatible with the CLOS specification. See CLUE release
notes for more details.\index{PCL}
\item {\tt clos-kludge}, a simple implementation of a CLOS subset which comes
with
the other CLUE software and which is sufficient for getting started with CLUE.
See the {\tt defsystem} file in the {\tt clos-kludge} directory.\index{clos-kludge}
{\tt clos-kludge} works but beware: you should use this only as a temporary
stop-gap until you have a ``real'' CLOS in place.
\item Something else. Maybe you've done a CLOS implementation yourself! Or maybe
you've gotten one from the vendor of your Common Lisp.
\end{itemize}

\item Compile and load your preferred CLOS.

\begin{center} \framebox[5.5in]{
\hspace*{\fill}
{\bf Note}
\hspace*{\fill}
\parbox[t]{4.5in}{
Make sure that before both CLX and CLUE are {\em compiled}, you have loaded your
preferred version of CLOS. If you switch CLOS versions later, be sure to
recompile both CLX and CLUE.} \hspace*{\fill}} \end{center}


\item Compile and load CLX, with the special CLUE patches with are included with
the other CLUE software.  See the CLUE {\tt defsystem} file.  Lisp machine
users may find it convenient to perform {\tt (make-system 'clx-clos)}.

\item Compile and load CLUE. See the CLUE {\tt defsystem} file.  Lisp machine
users may find it convenient to perform {\tt (make-system 'clue)}

\end{enumerate}

     
  
\subsection{Packages}
\index{packages, CLUE}
\index{packages, CLUEI}\index{intrinsics}
All symbols defined by the CLUE ``intrinsics'' are external in the {\tt
CLUEI} package.  The {\tt CLUE} package exports all symbols defined by the
standard contact set, 
\index{contact, standard set}
in addition to all of the
``intrinsics'' in {\tt CLUEI}\footnotemark\footnotetext{For now, the standard
contact set is undefined so using either the {\tt CLUE} or {\tt CLUEI}
package is effectively equivalent. Similarly, the CLUE specification currently
describes
only the intrinsics and the CLUE package, and does not yet mention the
standard contact set.}.

All symbols defined by CLX\index{CLX} are external in the {\tt XLIB} package. The {\tt
CLUE} and {\tt CLUEI} packages both use {\tt XLIB}.

The package containing CLOS\index{CLOS} symbols may vary, depending on which
implementation of CLOS you are using. Usually, you will want to use the {\tt
CLOS} package which exports the Explorer CLOS system. 

\index{packages, CLOS}\index{packages, LISP}\index{packages, TICL}
\begin{center}
\framebox[5.5in]{
\hspace*{\fill}
{\bf Note}
\hspace*{\fill}
\parbox[t]{4.5in}{
If you are using the Explorer {\tt CLOS} package, be sure to
also use only {\tt LISP}, the standard Common Lisp package, not {\tt TICL}.
Using both {\tt TICL} and {\tt CLOS} together will result in some nasty name
collisions (for example, on {\tt make-instance} and {\tt defmethod}.)}
\hspace*{\fill}} \end{center}


Application programmers should define a separate package for application symbols
which uses either:
\begin{itemize}
\item {\tt LISP}, {\tt CLOS} and {\tt CLUE}, or
\item {\tt LISP}, {\tt CLOS}, {\tt CLUEI}, and a non-standard contact package.
\end{itemize}

Contact programmers should define a separate package for each set of
related contacts, which uses either: 
\begin{itemize}
\item {\tt LISP}, {\tt CLOS}, {\tt XLIB}, and {\tt CLUE}, (if using standard
contacts), or 
\item {\tt LISP}, {\tt CLOS}, {\tt XLIB}, and {\tt CLUEI}.
\end{itemize}

\subsection{Warming Up}
Here are some CLUE calisthenics to limber you up. Some of them
will require further study of CLUE, CLX, and the X Window System. Some of them
are rather challenging.

\begin{enumerate}
\item Complete the {\tt blinker} contact example. Write a {\tt display} method
that will fill the {\tt blinker} with its {\tt color} when its {\tt on-p}
state is
true and fill it with its {\tt background} otherwise. Ensure
that the {\tt blinker-on-p} accessor updates the display correctly.

\item Create two  {\tt blinker} instances --- one that will print out either
``{\tt Off!}'' when
its {\tt :blink} callback is called with a {\tt nil} argument or ``{\tt On!}''
otherwise, and another which will make a funny noise on your Explorer {\em only}
when its {\tt :blink} callback is called with a non-{\tt nil} argument.

\item Try out different contact attributes on your {\tt blinker}. For example,
try a different {\tt border-width}, {\tt border}, and {\tt background}. Try
initializing the {\tt background} to a {\tt '(float 0 1)} value. What happened?
Why?

\item Define resources to change initial {\tt blinker} attributes.

\item Create two ``funny noise'' {\tt blinker} instances, as in the previous
exercise, but give them different funny noises. Do {\em not} change the
definition of the {\tt blinker} class.

\item Change the {\tt blinker} class so that whenever the cursor enters a {\tt
blinker}, it turns into a picture of Gumby. Hint: use the {\tt xfd} program
(font displayer) to find an appropriate element of the {\tt cursor} font.
Another hint: Use ``{\tt man xfd}'' to learn how to use {\tt xfd}.

\item Change the {\tt blinker} class to include a {\tt string} slot, containing
a string which can be displayed in any font. Display the string so that it
appears in the {\tt blinker}'s {\tt color}, is centered horizontally and
vertically within the {\tt blinker}'s current interior size, and is visible
regardless of the {\tt blinker}'s {\tt on-p} state. Ensure that the {\tt string}
remains centered when the user changes the font or the size of the {\tt
blinker}.

\item Define a {\tt composite} class that {\em always} has a single {\tt blinker}
child, which is always centered horizontally and vertically within the {\tt
composite}. Ensure that the {\tt blinker} child
remains centered when the user changes the size of the {\tt composite}.

\item Define an {\tt etch-a-sketch} contact that will:
\begin{itemize}
\item 
Rubberband a horizontal
line as the user moves the pointer with {\tt :button-1} down, then
snap it into place when the button is released.
\item Rubberband a vertical
line as the user moves the pointer with {\tt :button-2} down, then
snap it into place when the button is released.
\item Print out a copy of its current line drawing on the Imagen
when the {\tt \#$\backslash$return} key is pressed.

\end{itemize}

\end{enumerate}
\pagebreak
\begin{thebibliography}{9}

\bibitem{clos} Bobrow, Daniel G., et al. The Common Lisp Object System
Specification (X3J13-88-002). American National Standards Institute, June,
1988. 

{\em [The definition of CLOS. Start with Chapter 1, which is a thorough and
readable explanation of all concepts. Then consult with Chapter 2 for
programming details.]}

\bibitem{clos-guide} Keene, Sonya E. {\em Object Oriented Programming in Common
Lisp}. Addison-Wesley (1989).

{\em [A textbook-style guide to CLOS, written by a member of its ANSI
development team. Discusses all CLOS features, with examples.]}

\bibitem{Xt} McCormack, Joel, et al. The X-Toolkit Intrinsics, Version
11, Revision 2 (March, 1988).

{\em [Xt, the C language counterpart to CLUE. This is important because
CLUE will tend to evolve in the same direction as Xt.]}

\bibitem{x-protocol} Scheifler, Robert W. The X Window System Protocol, Version
11, Revision 3 (December, 1988).

{\em [The X Bible. The definitive word on how X works. A reference manual with
no tutorials or examples.]}

\bibitem{x-window-system} Scheifler, Robert W. and Gettys, Jim. The X Window
System. {\em ACM Transactions on Graphics}, Vol. 5, No. 2 (April 1986).

{\em [An excellent technical overview of the design and features of the X Window
System. Somewhat obsolete (written before X Version 11) but still informative.]} 

\bibitem{clx} Scheifler, Robert W., et al. CLX Interface
Specification, Version 4 (September 1987).

{\em [Terse, but this is the definition of CLX. Describes only the language
binding, so it must be read in conjunction with the X Protocol Specification.]}

\end{thebibliography}


\pagebreak
\begin{theindex}
% -*- Mode:TeX -*-
%%%----------------------------------------------------------------------------------+
%%%                                                                                  |
%%%                          TEXAS INSTRUMENTS INCORPORATED                          |
%%%                                  P.O. BOX 149149                                 |
%%%                             AUSTIN, TEXAS 78714-9149                             |
%%%                                                                                  |
%%%                Copyright (C) 1989 Texas Instruments Incorporated.                |
%%%                                                                                  |
%%% Permission is granted to any individual or institution to use, copy, modify, and |
%%% distribute this document, provided that  this complete copyright and  permission |
%%% notice is maintained, intact, in all copies.                                     |
%%%                                                                                  |
%%% Texas Instruments Incorporated provides this document "as is" without express or |
%%% implied warranty.                                                                |
%%%                                                                                  |
%%%----------------------------------------------------------------------------------+
\documentstyle[twoside,12pt]{article} 
\pagestyle{plain} 
%%
%% Inserted from home:/usr/local/hacks/tex/setmargins.tex:
%%
%%    Original: Glenn Manuel 12-17-86
%%    Added optional [printer-offset]: Glenn Manuel 2-27-87
%% Sets the margins, taking into account the fact that LaTex
%% likes to start the left margin 1.0 inch from the left edge
%% of the paper.
%%   This macro prints on the screen and in the log file
%%   the values it sets for:
%%     Textwidth, Odd Page Left Margin, Even Page Left Margin,
%%     Marginparwidth, Printer Offset
%%     (the values are in points: 72.27 pts/inch).
%%     The Odd and Even Page Left Margins include LaTeX's
%%     built-in 1.0in offset, but NOT the [printer-offset],
%%     so these values always indicate what the actual
%%     printout SHOULD measure.
%%
%% USAGE:  Place the following between the
%%         \documentstyle   and  \begin{document} commands:
%% \input{this-file's-name}
%% \setmargins[printer-offset]{line-length}{inside-margin-width}
%%
%%   where ALL arguments to \setmargins are DIMENSIONS
%%           like {6.5in}, {65pt}, {23cm}, etc.
%%   The units MUST BE SUPPLIED, even if the dimension is zero {0in}.
%%
%%   [printer-offset] is optional.  Default is zero.
%%
%%   Examples:
%% \setmargins{0in}{0in}         % default line length & default margins
%% \setmargins[-.12in]{0in}{0in} % compensate for printer offset
%% \setmargins{0in}{1in}         % default line length, 1 inch inner margin
%% \setmargins{6.5in}{0in}       % 6.5 inch line length & default margins
%% \setmargins{5in}{1.5in}       % 5 inch line length & 1.5 inch inner margin
%%
%%   {inside-margin-width} is defined as follows:
%%       For 1-sided printing: left margin for all pages.
%%       For 2-sided printing: left  margin for odd pages,
%%                             right margin for even pages.
%%
%%   Defaults:
%%   Each argument has a default if {0in} is used as the argument:
%%      line-length default = 6.0in
%%      inside-margin-width default:
%%         For 1-sided printing, text is centered on the page
%%                     (each margin = [1/2]*[8.5in - line length]);
%%         For 2-sided printing, inside margin is twice the outside margin
%%                     (inside  margin = [2/3]*[8.5in - line length],
%%                      outside margin = [1/3]*[8.5in - line length]).
%%      printer-offset default = 0in
%%
%%  For all cases, the outside margin (and marginparwidth, the
%%  width of margin notes) is just whatever is left over after
%%  accounting for the inside margin and the line length.
%%
%% Note: LaTeX's built-in offset of 1.0 inch can vary somewhat,
%%       depending upon the alignment of the Laser printer.
%%       If you need it to be EXACT, you will have to supply
%%       the optional [printer-offset] argument.
%%       Subtract the actual measured left margin on an
%%       odd-numbered page from the printed Odd Page Left Margin
%%       value, and use the result as the [printer-offset].
%%       Positive values shift everything to the right,
%%       negative values shift everything to the left.
%%
\makeatletter
\def\setmargins{\@ifnextchar[{\@setmargins}{\@setmargins[0in]}}
\def\@setmargins[#1]#2#3{
%%%  Uses temporary dimension registers \dimen0, \dimen2, \dimen3, \dimen1
    \dimen1=#1                         % 1st argument [printer offset]
    \dimen2=#2                         % 2nd argument (line length)
    \dimen3=#3                         % 3rd argument (inner margin)
     \advance\dimen1 by -1.0in         % for LaTeX built-in offset
    \ifdim\dimen2=0in 
        \textwidth=6in  \dimen2=6in
    \else \textwidth=\dimen2
    \fi
    \dimen0=8.5in
    \advance\dimen0 by -\dimen2         % 8.5in - line length
    \if@twoside
       \ifdim\dimen3=0in  % use defaults: 2/3 inside, 1/3 outside
          \divide\dimen0 by 3           % (8.5in-line length)/3
          \dimen2=2\dimen0              % (2/3)*(8.5in-line length)
          \oddsidemargin=\dimen2
          \advance\oddsidemargin by \dimen1   % add in offset
          \dimen2=\dimen0               % (8.5in-line length)/3
          \evensidemargin=\dimen2
          \advance\evensidemargin by \dimen1  % add in offset
%  allow for space on each side of marginal note
          \advance\dimen0 by -2\marginparsep
          \marginparwidth=\dimen0
       \else            % use supplied 2-sided value
          \oddsidemargin=\dimen3             % inside-margin-width
          \advance\oddsidemargin by \dimen1  % add in offset
          \advance\dimen0 by -\dimen3   % 8.5in-line length-inside margin
          \evensidemargin=\dimen0
          \advance\evensidemargin by \dimen1 % add in offset
%  allow for space on each side of marginal note
          \advance\dimen0 by -2\marginparsep
          \marginparwidth=\dimen0
       \fi
%  one-sided
    \else \ifdim\dimen3=0in  % use defaults: center text 
              \divide\dimen0 by 2         % (8.5in-line length)/2
              \oddsidemargin=\dimen0      % (8.5in-line length)/2
              \advance\oddsidemargin by \dimen1   % add in offset
              \evensidemargin=\dimen0     % (8.5in-line length)/2
              \advance\evensidemargin by \dimen1  % add in offset
%  allow for space on each side of marginal note
              \advance\dimen0 by -2\marginparsep
              \marginparwidth=\dimen0
          \else  % use supplied values
              \advance\dimen0 by -\dimen3  % 8.5in-line length-left margin
%  allow for space on each side of marginal note
              \advance\dimen0 by -2\marginparsep
              \marginparwidth=\dimen0
              \advance\dimen3 by \dimen1   % add in offset
              \oddsidemargin=\dimen3
              \evensidemargin=\dimen3
          \fi
    \fi
  \immediate\write16{Textwidth = \the\textwidth}
  \dimen0=1.0in
  \advance\dimen0 by \oddsidemargin
  \immediate\write16{Odd Page Left Margin = \the\dimen0}
  \dimen0=1.0in
  \advance\dimen0 by \evensidemargin
  \immediate\write16{Even Page Left Margin = \the\dimen0}
  \immediate\write16{Marginparwidth = \the\marginparwidth}
  \dimen0=#1
  \immediate\write16{Printer Offset = \the\dimen0}
 }
%
\def\@outputpage{\begingroup\catcode`\ =10 \if@specialpage 
     \global\@specialpagefalse\@nameuse{ps@\@specialstyle}\fi
     \if@twoside 
       \ifodd\count\z@ \let\@thehead\@oddhead \let\@thefoot\@oddfoot
                       \let\@themargin\oddsidemargin
% treat page 0 (title page) as if it is an odd-numbered page
        \else \ifnum\count\z@=0 \let\@thehead\@oddhead \let\@thefoot\@oddfoot
                                \let\@themargin\oddsidemargin
              \else \let\@thehead\@evenhead
                    \let\@thefoot\@evenfoot \let\@themargin\evensidemargin
     \fi\fi\fi
     \shipout
     \vbox{\normalsize \baselineskip\z@ \lineskip\z@
           \vskip \topmargin \moveright\@themargin
           \vbox{\setbox\@tempboxa
                   \vbox to\headheight{\vfil \hbox to\textwidth{\@thehead}}
                 \dp\@tempboxa\z@
                 \box\@tempboxa
                 \vskip \headsep
                 \box\@outputbox
                 \baselineskip\footskip
                 \hbox to\textwidth{\@thefoot}}}\global\@colht\textheight
           \endgroup\stepcounter{page}\let\firstmark\botmark}
%
\makeatother
%%
%% End of home:/usr/local/hacks/tex/setmargins.tex:
%%
\setmargins{6.5in}{1in}
\topmargin = 0in
\headheight = 0mm
\headsep = 0mm
\textheight = 9in
\makeindex
\begin{document}
\setlength{\parskip}{6 mm}
\setlength{\parindent}{0 in}

\title{A Quick and Dirty Guide to CLUE}
\author{Kerry Kimbrough}
\date{Version 6.0 \\
July, 1989\\[2 in]
\copyright 1989\  Texas Instruments Incorporated\\[.5in]
\parbox{3.5in}{
     Permission is granted to any individual or institution to use,
     copy, modify and distribute this document, provided that  this
     complete  copyright  and   permission  notice   is maintained,
     intact, in  all  copies  and  supporting documentation.  Texas
     Instruments Incorporated  makes  no  representations about the
     suitability of the software described herein for any  purpose.
     It is provided ``as is'' without express or implied warranty.
}}
\maketitle

This guide is a no-frills introduction to programming with  the Common Lisp User
Interface Environment (CLUE). CLUE is a high-level programming interface to the
X Window System which extends (but does not supersede) the basic CLX\index{CLX} interface.
CLUE is also an object-oriented programming system based on the Common Lisp
Object System (CLOS\index{CLOS}). Moreover, CLUE is a ``toolkit'' for constructing X user
interfaces. As a result, CLUE is modelled closely on the standard toolkit used
by C programmers, commonly known as Xt\index{Xt}, or the X Toolkit.

All of these related systems are described in separate documents (these
references are listed at the end of the guide), but we won't spend much time on
them here. Read this guide and take the plunge into CLUE; you can dry off with
the details later.

\section{The Big Picture}
Before you can understand CLUE, you have to look at three ideas about how
to build an interactive application. 


\subsection{The Application Is Event-Driven} 
An interactive application is controlled by a human user. Typically, nothing
happens until the user lifts his finger to the side of his mouse and causes
an input {\bf event}.  There are two implications here.
\index{event} 

First, the fundamental
application control structure is an {\bf event loop}: wait for an event, figure
out how to handle it, process the event, then go back and wait for the next one.
The {\em structure} of the event loop is generic; there is nothing
application-specific about it. 
\index{event, loop}

Second, the application is passive. It knows how to do certain things, but it's
waiting for the user's command to do them. You can imagine programs that do
things on their own, without waiting for user instruction (for example, a
nuclear reactor control system). But, the {\em interactive} part of such a
program is still an event loop, which must synchronize with the separate
background task(s) to provide the user control.


\subsection{The UI Is Separate} 
The user interface part has nothing to do with the essential function of the
application. An oft-cited analogy is linguistic communication, which has
independent modules for lexical, syntactic, and semantic processing. You can
understand me whether I speak to you or whether I write to you, because these
are merely two different lexical forms (e.g. sound patterns or marks) for the
same language.  Also, you can express the same idea either in French or in
English; the lexical and syntactic elements of each language are distinct from
their corresponding meaning, or semantics. 

An interactive application can be modularized along the same lines. The UI
contains the lexical and syntactic components, while the application {\em per
se} is the semantic component. Take some application command, say, ``Stop
Nuclear Reactor.'' The user interface to this command could use various lexical
forms (press the {\tt STOP} key or type the string ``{\tt stop}'') and
various syntactic forms (prompt with ``{\tt Are you sure you want to stop?}'' or
not), but the application semantics remain the same (hopefully!). From here on,
we will
acknowledge this modularization by referring to the whole application program as
``the program,'' to the user interface
components as ``the UI,'' and to the application semantic components as ``the
application.''

Separating the UI and semantic modules of an interactive program is good
programming technique. This allows you to more easily create a new UI for an
existing application. Or to apply an existing set of UI techniques to a new
application, so that it is consistent in style with others. Or to divide the
programming effort more efficiently; a domain expert can write the application
semantics while a human factors specialist can design the UI and a window system
hacker can implement it. 

\subsection{The UI Is Object-Oriented} This idea is more recent than the
other two, because it wasn't until a few years ago that graphical user
interfaces and window systems came along.  When they did, UI's began to
be composed of multiple user activities represented by visually-distinct
and physically-manipulable objects on the screen.  Object-oriented
programming (OOP) is thus a natural programming style which has been
associated with UI since its beginning\footnotemark\footnotetext{In
fact, true OOP and window systems were invented at the same time at
Xerox PARC in the Smalltalk system.}.  For several reasons, OOP is a
nice fit for UI programming.  UI's tend to generate profuse variation of
detail around a basic theme (consider, for example, menus); that is,
they demonstrate several subclasses of a basic class.  The modularity
of objects also helps to separate the UI from the semantics; that is,
the implementation of UI methods from the program which simply calls
them.

The OOP methodology leads to a UI which is composed of user interface objects
(command lines, messages, menus, scroll bars, dialogs, etc.). Each of these UI
objects is an ``agent'' for some portion of the  underlying application.
Each UI object is responsible for presenting some ``view'' of the application to
the user. It is also responsible for receiving the users events which are
directed to it and for translating these into the appropriate application
responses. 


\subsection{How CLUE Works\label{sec:how-clue-works}} 
CLUE combines all of these ideas. A CLUE program consists of an event loop, a
set of UI objects called {\bf contacts}, and a set of application functions
called {\bf callbacks}. All CLUE programs look pretty much like this:
\index{event, loop}

\begin{verbatim}
;; Open a connection to the X server.
(let ((display (open-contact-display 'my-application :host "server-host" ...)))
  ;; Initialize UI objects and callbacks.  
  (setf c (make-contact ... ))
  (add-callback c ... )
       ...
  ;; Process events until the event loop is terminated.
  (catch :event-loop
    (loop
      (process-next-event display)))  
  (close-display display))
\end{verbatim}

A CLUE program creates a {\bf contact-display} object which
\index{contact-display}
represents a connection to a specific X server.  This {\tt contact-display} then
becomes a two-way channel through which the program requests the creation and
display of contact windows and receives input events sent to its contacts. 

Each contact is a UI ``agent'' that is prepared to present some
application information, to accept user
events which manipulate this information, and then to report the results back to
the application. This reporting action is the critical connection between the UI
and the application, and it's done using {\bf callbacks}.\index{callback}
\index{callback, name}\index{callback, function}
A callback consists of
a {\bf callback name} and an associated {\bf callback function}. A contact is programmed
to associate a specific result with a callback name, which it reports by calling
the associated callback function (with a predefined argument list). 

Here is where CLUE contributes to the separation of UI and application. 
\index{application programmer}\index{contact programmer}
Every
CLUE program usually has two programmers! A {\bf contact programmer} is one
who defines a contact class and implements its methods. An {\bf application
programmer} is one who decides to create a contact instance and employ it for a
particular UI role. The contact programmer decides what callback names his
contact class will use and what the argument lists to the associated callback
functions have to be. The application programmer defines callback functions to
perform specific jobs and plugs them into the appropriate contact callback.
The contact programmer will frequently use the basic CLX\index{CLX} interface, typically to
display the contact's information. The application programmer will use CLUE
interfaces but will almost never need to call CLX functions directly.

Now we can see the operation of a CLUE program in a bit more detail. The {\tt
make-contact} function is used to create and initialize contacts. 
\index{make-contact}
The {\tt
add-callback} function \index{add-callback} associates application semantics
(i.e. callback
functions) with each contact. Then, the program enters the event loop. Each call
to {\tt process-next-event} \index{process-next-event} reads the next input event and dispatches it to the
contact to which it belongs. This contact then goes through a process of {\bf
event translation}\index{event translation}, which determines which
contact {\bf action}\index{action} functions will be
executed. Contact action functions, in turn, perform all the details of input
feedback and display. Finally, if the user event has completed an application
result, the action function will look up and call the associated callback
function. What has happened? CLUE has provided all the machinery, the contact
programmer has defined the detailed UI behavior, and the application programmer has
furnished the real results.

The rest of this guide will be a closer look at each of these parts of a CLUE
program.

\section{The Contact Display\label{sec:contact-display}}
\index{contact-display}
\index{open-contact-display} 
A {\tt contact-display} is created by the {\tt
open-contact-display} function.  {\tt op\-en\--con\-tact\-dis\-play} has one required
argument: a symbol which acts as the name of your program.  Technically, this is
a {\bf resource name}\index{resource, name} which is used to access the values of
{\bf resources}\index{resource} associated with your program.  CLUE resource
management is explained later in Section~\ref{sec:resources}.  Generally
speaking, this program resource name is used to distinguish your program from
other application programs that may also be running. {\tt open-contact-display}
has several optional keyword arguments (mainly the same ones used by the CLX\index{CLX}
{\tt open-display} function), but the most frequently used is the {\tt host}, a
string which gives the network name for the X server host to which you are
connecting.

\begin{verbatim}
(setf display
      (open-contact-display
        'blink                    ; Application name
        :host "server-host"))     ; Server host name
\end{verbatim}

{\tt contact-display} objects are really a subclass of the {\tt display} data
type defined by CLX\index{CLX}\footnotemark\footnotetext{However, neither {\tt display}
objects nor {\tt contact-display} objects are actually implemented as CLOS
classes. There's no reason to define {\tt contact-display} methods or
subclasses.}.
\index{display}

\begin{center}
\framebox[5.5in]{
\hspace*{\fill}
{\bf Note}
\hspace*{\fill}
\parbox[t]{4.5in}{
Any CLX\index{CLX} function which takes a {\tt display} argument can be called with a {\tt
contact-display} argument instead.}
\hspace*{\fill}}
\end{center}

The CLX function {\tt close-display}\index{close-display} should always be used
to close the server connection when the program terminates.

\section{Contacts\label{sec:contacts}}
In simplest terms, a {\tt contact} is a kind of window. The behavior of a window
object,
which is represented in CLX\index{CLX} as an instance of the
{\tt window} data type,
is completely specified by the X Window System
protocol.  In CLUE, the class of {\tt contact} objects is defined to be
a subclass of the class of {\tt window} objects\footnotemark\footnotetext{To do
this, CLUE actually changes the definition of the CLX {\tt window} type to
make it a CLOS class.}.  Thus, all CLX {\tt window} functions
apply to any {\tt contact}. 

\begin{center}
\framebox[5.5in]{
\hspace*{\fill}
{\bf Note}
\hspace*{\fill}
\parbox[t]{4.5in}{
Any \index{CLX}CLX function which takes a {\tt window} argument can be called with a {\tt
contact} argument instead.}
\hspace*{\fill}}
\end{center}


As a window, a contact inherits several basic attributes,
such as a
rectangular geometry (given by the position of its upper left corner, its width,
and its height in
pixel coordinates), a border width and border color/pattern, a depth (i.e.
bits per pixel), and a background color/pattern.
\index{contact, attributes}
Contacts possess a few additional attributes (or {\bf slots}) not shared
by ordinary windows. 
\begin{itemize}
\index{contact, name}
\item {\tt name}: The name of
the contact is a
symbol  which can be used to access contact resources
stored in a resource database. See Section~\ref{sec:resources}.

\index{contact, parent}
\item {\tt parent}: All
contacts have a parent contact (except for a {\bf root}\index{root} contact, which is
created automatically by CLUE and which is the ancestor of all contacts on a
single display screen).  The parent-child relationship
is fundamentally one between two windows, as defined by the X
Window System. For example, a child window is always visually
contained
by its parent window, and its position is specified relative to its
parent's upper-left corner. However, in CLUE this relationship is
extended, with the parent assuming additional responsibilities, such as
geometry management (see Section~\ref{sec:composites}).

\index{contact, state}
\item {\tt state}: A switch which controls the visual effect of the contact on
the UI. If the {\tt state} is {\tt :withdrawn}, then the contact is invisible and
unavailable for input. If the {\tt state} is {\tt :mapped}, then the contact is
``viewable'' (i.e. visible unless it is covered by other windows) and available
for input. The {\tt state} can also have a third value --- {\tt :managed} ---
which is discussed in
Section~\ref{sec:resources}. The default {\tt state} is {\tt :mapped}.


\index{contact, event-translations}
\item {\tt event-translations}: A list which associates
different types of input events with contact actions. This is discussed in more
detail in Section~\ref{sec:events}.

\index{contact, callbacks}
\item {\tt callbacks}: An association list of callback name/function pairs.
\index{callback}
\end{itemize}

\subsection{Using Contacts}
\index{contact, application programmer interface}
If you are an application programmer, your basic view of a contact --- the generic
contact protocol --- is quite simple and consists of only a few functions.
\index{contact, generic protocol}

{\tt make-contact} 
\index{make-contact}
creates a new instance of any con\-tact subclass. This function is a
syn\-tac\-ti\-cal\-ly-identical extension of the basic CLOS\index{CLOS} {\tt make-instance}
function. Keyword arguments to  {\tt make-contact}  provide initial slot values
for the new instance, most of which can be defaulted. However, a {\tt parent}
argument must be given; specifying a {\tt contact-display} as the {\tt parent}
creates a {\bf top-level contact}\index{contact, top-level}. 

\begin{center}
\framebox[5.5in]{
\hspace*{\fill}
{\bf Note}
\hspace*{\fill}
\parbox[t]{4.5in}{
A contact ``belongs'' to the {\tt contact-display} connection on which it is
created. More precisely, since a contact must have a top-level ancestor, it
belongs to the {\tt contact-display} given as this ancestor's {\tt parent}.
This {\tt contact-display} connection receives all requests and events related
to the contact; it is also available as the value of the contact's {\tt display}
slot.} \hspace*{\fill}} \end{center}


Usually (and always
for top-level contacts), you must also specify the contact's {\tt width} and {\tt
height}. A non-top-level contact may find that its initial values for 
{\tt x}, {\tt y}, {\tt width}, {\tt
height}, and {\tt border-width} have been modified by the geometry management
policy of its parent (see Section~\ref{sec:composites}).

\begin{verbatim}
(setf a-blinker
      (make-contact
        'blinker                  ; Make an instance of the blinker class
        :parent display           ; Required argument
        :width  300               ; Usually required
        :height 400               ; Usually required
        :background 0))           ; Optional, defaults to parent's
\end{verbatim}



The {\tt (setf contact-state)} function 
changes the {\tt state} value of the contact.
\index{contact, state}

\begin{verbatim}
(setf (contact-state a-blinker) :mapped)    ; Make it viewable
(setf (contact-state a-blinker) :withdrawn) ; Make it invisible
\end{verbatim}

{\tt change-geometry} 
\index{change-geometry} 
requests a change to one or more components of the contact's geometry.
The keyword
arguments to {\tt change-geometry} --- {\tt x}, {\tt y}, {\tt width}, {\tt
height}, and {\tt border-width} --- specify the changed component(s); omitting a
keyword means ``Leave the current value unchanged.'' But the actual effect of
the request depends on the geometry management\index{geometry management}
policy applied to the contact (See Section~\ref{sec:composites}). If the
request is refused, then the return values give  acceptable alternatives for {\tt x}, {\tt y}, {\tt width}, {\tt
height}, and {\tt border-width}.  You may want to give an {\tt accept-p} keyword
value of {\tt t}, which means ``Go ahead and replace my request with whatever
alternative geometry may be returned.''

\begin{verbatim}
(change-geometry a-blinker
                 :x 100           ; Request new upper-left position...
                 :y 200
                 :accept-p t)     ; ...but accept closest alternative

(change-geometry b :accept-p t)   ; Just validate current values
\end{verbatim}

\begin{center}
\framebox[5.5in]{
\hspace*{\fill}
{\bf Note}
\hspace*{\fill}
\parbox[t]{4.5in}{
There are no accessor functions for {\tt setf}'ing a contact's {\tt x}, {\tt y}, {\tt width}, {\tt
height}, and {\tt bor\-der\--width} slots. Never modify these slots directly. Always
use {\tt change-geometry} instead.} 
\hspace*{\fill}}
\end{center}


The {\tt change-priority} function \index{change-priority}
requests a change to the contact's stacking priority relative to other
windows.  The stacking priority determines which window is ``on top'' of other
windows. This request is also subject to geometry management policy. 

The {\tt destroy} function is called \index{destroy}
only when the contact will no longer be referenced. This frees any display
resources allocated to the contact.

In addition to the generic protocol, the application programmer interface to a
contact includes its callback protocol, plus any special class-dependent functions (typically, for
initialization).

\subsection{Defining Contacts}

If you are a contact programmer, you will use the {\tt defcontact} macro to
define a new contact class.  The syntax of {\tt defcontact} is nearly identical
to that of the basic CLOS\index{CLOS} {\tt defclass} macro.  The only difference is the
additional {\tt :resources} option for specifying contact resources (see
Section~\ref{sec:resources}).

\begin{verbatim}
(defcontact
  blinker                         ; Class name  
  (contact)                       ; Superclasses  
  ((color                         ; Slot specs, with name...
     :type     pixel              ; ...data type...
     :accessor blinker-color      ; ...accessor function name ...
     :initarg  :color             ; ...initarg keyword for make-contact...
     :initform 0)                 ; ...and default initial value.
   (on-p
     :type     boolean
     :accessor blinker-on-p
     :initform nil))  
  (:resources color))             ; Resource specs
\end{verbatim}

A contact programmer must also define a {\tt display} method for a new contact
class. CLUE calls the {\tt display} function automatically whenever any portion
\index{display}\index{contact, exposure}\index{exposure}
of the contact image must be (re)displayed. In particular, {\tt display} is
called whenever an invisible contact is changed to the {\tt :mapped} state, or
when some previously-hidden part of a {\tt :mapped} contact is exposed.

\begin{verbatim}
(defmethod                        ; Define a method ...
  display                         ; ... for the display function...
  ((contact blinker)              ; ... when the contact arg is a blinker.
   &optional x y width height)    ; Defines the rectangular piece exposed.
  ...)
\end{verbatim}

\section{Composites and Geometry Management\label{sec:composites}}
A contact which is the parent of another contact is known as a
{\bf composite}\index{composite} and is an instance of the {\tt composite}
subclass of {\tt contact} objects. A {\tt composite} may be the parent of another
{\tt composite},
leading to a tree-structured contact hierarchy. A contact is said to be an
{\bf ancestor}\index{ancestor} of another contact (its 
{\bf descendant}\index{descendant}) when
it is its parent or an ancestor of its parent. 

A {\tt composite} represents a set of contacts which can be manipulated
(positioned, presented,
etc.) as a unit.  A {\tt composite} is useful whenever several contacts act in
concert to provide a
single component of the UI.  Typical examples include
``control panels'' and ``dialog boxes'' --- groups of contacts that are presented
together and are used to make related adjustments to application data.  
The fundamental aspects of the contact parent-child relationship are the same as
those of the window hierarchy defined by the X Window System. 
\index{geometry management}
But in CLUE, a {\tt composite} is also expected to act as the {\bf geometry
manager}\index{geometry manager} for its child contacts; that is, to
implement a style of layout for its children. 


Here is how it works.  A request to change the geometry of a contact is
forwarded to the contact's parent, which actually performs the resulting change.
It is important to understand that, due to its constraints, a geometry manager
may not be able to perform a change as requested.  For example, a request to
increase the size of a contact might be refused if its geometry manager enforces
a maximum size.  Even if a requested change cannot be done, a geometry manager
may be able to suggest a slightly different change which would be acceptable.

Placing geometry control in the hands of a geometry manager in this way
has several advantages.
\begin{itemize}
\item   A geometry manager can arbitrate the competing geometry change requests
        of several contacts in order to implement contraints among them.
        
\item   A given layout style can be applied to any collection of contacts.
        
\item   Contact layout style can be changed without the knowledge of individual
        contacts.
\end{itemize}


A composite's geometry management policy applies only to the set of its children
which are {\bf managed}. \index{contact, managed}
An unmanaged child is ignored by its geometry manager. Any
geometry change to an unmanaged child is performed immediately as requested. 
However, changes to a managed child are arbitrated by its parent's geometry
management policy. This can mean, for example, that a change to one child's
position/size/priority can affect that of other children. Indeed, requesting a
bigger size for a child might result in the parent asking {\em its} geometry
manager to become bigger, leading to a ripple of geometry changes throughout
the contact hierarchy.

How do you know if a contact is managed? This is determined by its {\tt
state}.\index{contact, state}
A contact is managed if and only if its {\tt state} is not {\tt :withdrawn}. This means
that a {\tt :mapped} (and viewable) contact is managed. But there is also
another possibility: a contact's {\tt state} may also be {\tt :managed}. This
value represents the (rather rare) case of a contact which is not visible, but
which is nevertheless taken into account by its geometry manager. 

\begin{center}
\framebox[5.5in]{
\hspace*{\fill}
{\bf Note}
\hspace*{\fill}
\parbox[t]{4.5in}{
An
unmanaged child can {\em never} be visible.}
\hspace*{\fill}}
\end{center}


Implementing geometry management is a topic for contact programmers only. If you
define a new {\tt composite} subclass, you will usually need to implement
methods for the following three functions.
\begin{itemize} 
\item {\tt change-layout}: 
\index{change-layout}
This function is called whenever a composite's set of
managed children changes (e.g. by creating or destroying a managed child or by
changing the {\tt state} of a child).

\item {\tt manage-geometry}: 
\index{manage-geometry}
This function is called by {\tt change-geometry} to
approve a geometry request. It must either approve the change or return a set of
alternative geometry values which will be acceptable.

\item {\tt manage-priority}: 
\index{manage-priority}
This function is called by {\tt change-priority} to
approve a stacking priority request. It must either approve the change or return
an alternative priority which will be acceptable.

\end{itemize}

\section{Events\label{sec:events}}
\index{event}\index{input}
Handling input events is CLUE's main job, and it's an area where CLUE does a lot
of the work for programmers automatically. Application programmers don't have to
worry about much more than setting up an event loop, as shown in
Section~\ref{sec:how-clue-works}. Contact programmers need to know how to use
CLUE's {\bf event translation}\index{event translation} mechanism in order to
implement the details of a contact's input behavior.

What exactly are these events we're talking about? Fundamentally, an event is
a message to the program from the X server, a packet of data describing
some occurrence that the program ought to be interested in.
Most events
represent an action performed by the user with  the X server's 
\index{input, devices}
keyboard or pointing device (i.e. mouse, tablet,
etc.). Keyboard
actions, for example, can generate {\tt :key-press} and {\tt :key-release}
events. The pointer may be used to generate {\tt :button-press} and {\tt
:button-release} (when the user presses one of the mouse buttons) and {\tt
:motion-notify} events (when the user changes the pointer position). These kind of
{\bf user events}\index{event, user} contain data for the button or key
involved, the coordinates of the pointer position, and other useful things, such as
the up/down state of {\tt :shift}, {\tt :control}, and the other modifier keys.
In addition to user events, there is another important group of {\bf side-effect
events},\index{event, side-effect}
events which occur as an indirect result of other user actions. For example,
when the user causes part of an obscured window to become visible, an {\tt
:exposure}\index{exposure} event may be sent. Other side-effect events include
{\tt :enter-notify}, which signifies that the user has moved the pointer cursor
inside a particular window, and {\tt :focus-in}, which happens when the user has
identified a particular window as the {\bf focus} for keyboard
events.\index{focus}\index{input, focus}
This is just the beginning, but the definition and meaning for all of the events
received by CLUE contacts can be found by reading the X Protocol Specification.


To understand CLUE event processing, let's start with the basic program event
loop\index{event, loop} and follow its operation, step by step.
A call to {\tt
process-next-event} \index{process-next-event}\index{timeout}
gets the ball rolling. {\tt
process-next-event} causes CLUE to read
the next event from the given {\tt contact-display} connection. By default,
{\tt pro\-cess\--next\--ev\-ent} does not return until the next event has been
completely processed. This means that if no event is yet available, then {\tt
process-next-event} will generally wait until one finally comes along. However,
an optional {\tt timeout} argument may be given, which says how long {\tt
process-next-event} will wait before giving up. Set the {\tt timeout} to {\tt 0}
if you don't want {\tt process-next-event} to wait at all. {\tt
process-next-event} returns {\tt nil} if a timeout occurred; otherwise, it
returns {\tt t}.

\begin{verbatim}
(catch :event-loop
  (loop
    (unless 
      (process-next-event display 5)  ; Timeout after waiting 5 sec.

      ;; No events yet -- do something useful
      (do-background-task))))
\end{verbatim}

\begin{center}
\framebox[5.5in]{
\hspace*{\fill}
{\bf Note}
\hspace*{\fill}
\parbox[t]{4.5in}{
{\tt process-next-event} does not return any result of the actual processing of
an event. This means
that all responses of a CLUE program to user input  --- even the termination
of the event loop --- occur as ``side-effects'' of contact event handling.
Always escape the event loop by {\tt throw}'ing to some well-defined tag.}
\hspace*{\fill}}
\end{center}

Inside {\tt process-next-event}, the next thing that happens is that the event is
{\bf dispatched}\index{event,
dispatching}. In other words, CLUE figures out which contact is supposed to
handle the event. This is fairly easy because almost every event message
contains an identifier for the {\tt window} object which is the
``addressee'' of the
event. CLUE knows how to convert this identifier into the corresponding {\tt
contact} object. At this point, the event
is {\bf handled}\index{event, handling} by calling the {\tt handle-event}
function\index{handle-event} with the receiving {\tt contact} and the event as
its arguments. {\tt handle-event} then implements the process of event
translation, i.e. figuring out which methods of the receiving {\tt contact} to
call in response to the event\footnotemark\footnotetext{{\tt
handle-event} is also the function which automatically calls the
contact's {\tt display} method, if the event is an {\tt :exposure}.\index{display}\index{exposure}}.
What happens next is of interest only to contact programmers.

\subsection{Event Translation}
Event translation involves searching through one or more association lists,
looking for an entry that matches the event. Each such entry is called an {\bf
event translation}, and it is a list containing an {\bf event
specification}\index{event, specification} and one or more {\bf
action}\index{action} names. An event specification describes a certain sort of
event, and an action name is simply the name of a special kind of function
--- 
an action function. So, during event translation, the event is compared with an
event
specification, and when it matches, then each of the corresponding action functions are
invoked in sequence. When all of them have completed, then event
translation is done; {\tt handle-event} returns and the event is said to
be ``handled.'' The {\tt process-next-event} function also returns, and we
go back to the top of the event loop to deal with the next event.
But where are these event translations and how are they created? There are two
different ways to create event translations: the {\tt add-event} function and
the {\tt defevent} macro.

{\tt add-event}\index{add-event}
creates an event translation and adds it to the association list
found in the {\tt event-translations} slot of a specific contact. This kind of
event translation is thus an {\bf instance event
translation}\index{event translation,
instance} which affects only one contact instance. An instance event
translation says ``When an event is dispatched to {\em this} contact and it
matches {\em this} event specification, then call {\em these} actions.''

\begin{verbatim}
(add-event a-blinker                 ; When this contact gets...
           '(:key-press #\ctrl-b)    ; ...this character event ...
           'beep)                    ; ...call the beep action.

(add-event a-blinker                 ; When this contact gets...
           '(:key-release #\ctrl-b)  ; ...this character event ...
           '(beep 1))                ; ...call the beep action with an arg.
\end{verbatim}

Note that in place of a simple action name, an event translation can also give a
list containing the action name plus a list of arguments to pass.
The process of event translation starts with the instance event translations in
the {\tt event-translations} slot of the receiving contact. Each entry is
examined in order, until a match is found. What happens if no match is found? In
this case, the process continues to search the event translations defined by the
{\tt defevent} macro.

{\tt defevent}\index{defevent}
creates a  {\bf class event translation}\index{event translation, class}
that applies to every instance of a given class.  A class event
translation says  ``When an event is dispatched to a contact of {\em this} class
and it matches {\em this} event specification, then call {\em these} actions.''

\begin{verbatim}
(defevent blinker                    ; When any blinker gets...
          (:button-press :button-1)  ; ...this button event...
          (blink t))                 ; ...blink on.

(defevent blinker                    ; When any blinker gets...
          (:button-release :button-1); ...this button event...
          (blink nil))               ; ...blink off.
\end{verbatim}

Class event translations are searched for each contact superclass of the
receiving contact, starting with the class of the contact, then continuing back
up the {\bf class precedence list}\index{class precedence list} which this
contact inherits\footnotemark\footnotetext{See the CLOS\index{CLOS} specification for a precise definition of the
class precedence list.}.  The idea is to match a class event translation
for the
most specific class possible.  However, if a matching event translation is {\em
still} not found, then CLUE gives up, and the event is ignored. After
all this, it couldn't be a very important event anyway!

\subsection{Event Specifications}
\index{event, specification}
As you may have noticed, an event specification is usually a list
composed of an event type keyword and some other qualifiers. In fact, an event
type keyword alone is also a valid event specification.
The exact syntax for event specifications is fairly complex. A sophisticated
programmer can even define his own event specification syntax. All of this is
completely spelled out in the CLUE specification. But for now, you can get an
idea of what an event specification is by looking at the following examples.

\begin{verbatim} 
:enter-notify         ; Matches any :enter-notify event

(:motion-notify       ; Matches a :motion-notify event..
  :button-1)          ; ...if :button-1 is down.

(:button-press        ; Matches a :button-press...
  :button-1           ; ...if it's :button-1...
  :shift)             ; ...and the shift key is down.

(:button-release      ; Matches a :button-release...
  :button-3           ; ...if it's :button-3...
  (:shift :control)   ; ...and the shift,control keys are down...
  :all)               ; ...and all other modifiers are up.

(:key-press           ; Matches a :key-press...
  :any                ; ...for any key...
  (:control :hyper))  ; ...when the control,hyper keys are down.

(:button-press        ; Matches a :button-press...
  :button-3           ; ...if it's :button-3...
  :double-click)      ; ...and it's a double-click.
\end{verbatim}

\begin{center}
\framebox[5.5in]{
\hspace*{\fill}
{\bf Note}
\hspace*{\fill}
\parbox[t]{4.5in}{A contact will receive {\em only} the types of events described
by
its event specifications, i.e. by event specifications in its instance and class
event translations. CLUE automatically sets the 
{\bf event mask}\index{event, mask} of the contact window, based on the
contact's event specifications. Event masks are a detail of CLUE internals, but
you can find more information about them in the X Protocol
Specification and the CLUE specification.}\hspace*{\fill}} \end{center}

\subsection{Actions and Callbacks}
\index{action}
An action is a  function which is designed to handle an input
event.  An action therefore implements a particular input behavior exhibited by
each instance of a certain contact class. 
In fact, an action function is typically a CLOS generic function, and actions
usually are defined as methods of a particular contact class. 

\begin{verbatim} 
(defmethod blink ((contact blinker) blink-on-p)
  (with-slots (on-p) contact
    ;; Set internal state variable on/off
    (setf on-p blink-on-p)
    ;; Redisplay based on new on-p state
    (display contact)
    ;; Invoke callback with new state
    (apply-callback contact :blink on-p))
\end{verbatim}

In general, an action represents a well-defined contact behavior that
could be done in
response to {\em any} event. In fact, it's possible to personalize an existing UI by
modifying its event translations so that contact actions are rebound to
the types of events (i.e. event specifications) which
are more to your taste.  In some cases,
however, an action may be designed to handle a very specific event type.

When an action computes a result that is important to an application, it invokes
a callback.\index{callback}
This is done via the {\tt apply-callback} macro.\index{apply-callback}
The example above shows that a {\tt blinker} contact has a callback whose name
is {\tt :blink}, which is called with a single argument, i.e. a boolean
representing the new on/off state of the {\tt blinker}. What significance
this has to the application is unknown to the contact programmer who wrote this
action; it all depends on what callback function the application programmer has
associated with {\tt :blink} for this {\tt blinker} instance.
In fact, the application programmer may have decided not to define a callback
function for {\tt :blink}, in which case {\tt :blink} has no application meaning
at all (and {\tt apply-callback} simply does nothing).
As shown in the {\tt :blink} example,
{\tt apply-callback} is usually (although not
necessarily)
called directly by an action method or somewhere within its dynamic extent.

\begin{center}
\framebox[5.5in]{
\hspace*{\fill}
{\bf Note}
\hspace*{\fill}
\parbox[t]{4.5in}{
Callback names are shared by a class, i.e. referenced by the class's
action methods.  But
callback name/function pairs are instance data and are recorded in a contact's
{\tt callbacks} slot. In our example, every {\tt blinker} is
expected to have a {\tt :blink} callback function, but each {\tt blinker}
usually has a different {\tt :blink} callback function. That is, each {\tt
blinker} usually has different application semantics.} \hspace*{\fill}}
\end{center}

In the previous example, the {\tt blink} action method does not depend on the
event which causes it to be invoked.  But what happens when an action's behavior
depends on information contained in the event?  CLUE handles this by
representing an event as an instance of the {\tt event} class.  An {\tt event}
object has slots which contact various kind of interesting information about the
event.  Since there are many types of X events, there are many different {\tt
event} slots, even though for a given event some slots are irrelevant and
therefore have a {\tt nil} value. You can look in the X Protocol Specification for
a complete description of event slots (and look in the CLX specification to find
out how these slot values are represented in Lisp).

But how does an action access an event object? This is done by using a special
CLUE macro called {\tt with-event}. For example:

\begin{verbatim}
(defmethod beep ((contact blinker) &optional (per-cent-volume 0))
  (with-event (state)
    ;; Was the shift key down?
    (when (member :shift (make-state-keys state))
      ;; Ring server's chime!
      (bell (contact-display contact) per-cent-volume)  
      ;; Invoke callback
      (apply-callback contact :beep))))
\end{verbatim}

The {\tt beep} action will beep only if it determines that the shift key was
down when the event occurred.  It does this by examining the {\tt state} slot of
the event, which defines which modifier keys were pressed at the time of the
event.  Notice the use of the {\tt with-event} macro.\index{with-event} This is
similar to the {\tt with-slots} macro of CLOS\index{CLOS}.  {\tt with-event}
binds slots of the ``current'' event argument within its lexical extent, so that
the action code can refer to them.  But why is the current event object hidden
in this way?  Because this allows for a much more efficient implementation of
{\tt event} objects than would be possible if their structure was fully exposed
to programmers.  So, there!

\subsection{Advanced Input}
\index{input}
Input events are a subject so dear to the heart of CLUE that it
contains many more input programming functions than can be
covered in this guide. Here's a quick list of CLUE's special input
features, all of which are discussed in complete detail in the CLUE
specification.
\begin{itemize}
\item {\bf Before actions},\index{before action}\index{action, before}
action functions which can be set to execute for {\em every} event
dispatched to a contact of a certain class, before event translation begins. 

\item {\bf Timers},\index{timer}
which send special {\tt :timer} events to a contact at a regular,
specified rate.

\item {\bf Synchronizing}\index{event, synchronizing}
event processing, so that a program can stop to process all pending
events before continuing.

\item {\bf
Sensitivity},\index{contact,
sensitivity} which allows a contact's input to be temporarily ``switched
off'' without changing its visibility.

\item {\bf Event compression},\index{event, compression}
which removes certain redundant events for faster performance.
\end{itemize}

\section{Resources\label{sec:resources}}
Imagine that you are the user of an interactive program. Maybe you are
not a programmer, but there are certain things about the UI that you
wish you could change ever so slightly. For example, maybe you wish you
could make the program use a different set of colors for various
objects. Or maybe you're left-handed, and you'd prefer to use the right
mouse button in place of the left one. Or maybe you can't read Japanese,
which happens to be the language all the menu items are written in.
Although most of the UI is defined by programmers  --- contact
programmers who designed the various UI objects and application
programmers who placed them neatly on the screen --- the truth is that
some aspects of the UI ought to be up to the {\em user}. Programmers,
keep your hands off!

Nowadays, thoughtful UI programmers recognize this problem, so CLUE
allows  programmers and users to cooperate in
defining the UI. The basis for this cooperation is {\bf resource
management}. A user can store various UI values as {\bf
resources}\index{resource} in a {\bf resource database}\index{resource,
database}, and CLUE contacts can read these resource values and modify
the UI accordingly. 

\subsection{The Resource Database}
Before a user can assign a resource value to a particular UI object, he
must have a way of identifying it, some kind of  name which
can be referenced outside of the program which creates the object.
Remember that a contact has a {\tt name} slot\index{contact, name}; this
is a symbol that could certainly help identify the contact. Consider
also that a contact is generally part of a nested hierarchy of UI
objects. For example, a paint program might have pattern choice
contact which contains several button contact children, one for each available
pattern style. A more complete name for a button child would be a
list which also includes its parent's name, e.g. {\tt '(pattern-choice
checkered)}. We can extend this notion all the way up the hierarchy to
come up with a {\bf complete resource name}\index{resource, complete
name} for the {\tt checkered} pattern button. This would be a list of
resource name symbols, starting with the resource name for the program
itself (i.e. the required argument to {\tt
open-contact-display}\index{open-contact-display}) and continuing with
symbols for the {\tt root}\index{root} ancestor, the top-level ancestor,
and so on:\index{resource, name}

\begin{verbatim}
    '(paint screen-1 pattern-choice checkered)
\end{verbatim}

Assuming that the {\tt checkered} button has a resource for its label
string named {\tt label} and another resource for the label font named
{\tt font,}
then the complete names for these resources would obviously be:


\begin{verbatim}
    '(paint screen-1 pattern-choice checkered label)
    '(paint screen-1 pattern-choice checkered font)
\end{verbatim}

As you might expect when dealing with long path names like these, it's
often useful to insert ``wildcards'' in the places where you don't want
to be so specific. For example, the following resource name would refer to
the {\tt font} of {\em everything} in the pattern choice contact, regardless of
the screen where it appears:

\begin{verbatim}
    '(paint * pattern-choice * font)
\end{verbatim}

Another way of identifying a contact is by its object class. So, for
example, you could refer to the font of {\em every} {\tt button} object in the
paint program with the following resource name.

\begin{verbatim}
    '(paint * button font)
\end{verbatim}

By associating values with these sorts of resource names in {\bf
resource bindings}\index{resource, binding}, a user can
create a resource database which represents his UI preferences.
CLUE\index{CLUE} provides a {\tt define-resources} macro for adding
resource bindings to the database.\index{define-resources}
\begin{verbatim}
(define-resources
  (paint * button background)               white
  (paint * pattern-choice * font)           helvetica-12
  (paint * pattern-choice checkered label) "Checkered")
\end{verbatim}

CLUE automatically reads the resource database whenever a contact is
created. A contact has a class and possibly other contact superclasses
on its class precedence list\index{class precedence list}. Each such
(super)class has a list of resource names that it uses (defined by the
{\tt :resources} option of the defining {\tt
defcontact}\index{defcontact}
form). Together, these form the set of contact resources that are looked
up in the resource database. Of course, finding a resource value is rather complicated,
because names in the database may not be complete resource
names and may contain a mixture of resource names and class symbols.
Nevertheless, during initialization,
CLUE will find the resource binding that is the closest match for each
contact resource. In general, a contact gets its resource values  first
from the user's resource database, then from programmer-specified
defaults when no user value is found.

\begin{center}
\framebox[5.5in]{
\hspace*{\fill}
{\bf Note}
\hspace*{\fill}
\parbox[t]{4.5in}{
The contact programmer decides which values a contact will look up in
the resource database; the user decides what values it will find there.
The contact programmer identifies resources by listing them in the {\tt
:resources} option of the {\tt defcontact} form for his contact class.\index{defcontact}}
\hspace*{\fill}} \end{center}

\subsection{Converting Resource Values} 
\index{resource, conversion}
\index{resource, types}
Another convenient feature of
CLUE resource management is automatic resource type conversion.  Often
the type of value specified by the user in the resource database is not
the data type actually used by the program.  For example, a user might
identify a font by a string such as ``{\tt helvetica},'' but the program
must convert this name into a CLX {\tt font} object before text is
displayed.  Another example is color: where a user might specify ``{\tt
red},'' a program must somehow determine a colormap and pixel value that
will yield this hue.

CLUE automatically converts user values out of the user's resource
database into the correct target data type (as specified in the {\tt
:resources} option of a {\tt defcontact}\index{defcontact} form). This
is done by calling the {\tt convert} function\index{convert}. For
example:

\begin{verbatim}
(setf color (convert         ; Convert a resource...
              a-blinker      ; ...for a blinker...            
              "red"          ; ...from a string value...              
              'pixel))       ; ...to a pixel target type.
\end{verbatim}

CLUE defines a number of {\tt convert} methods for various combinations
of source and target data types. Contact programmers can extend
this mechanism by defining their own special {\tt convert} methods.

\section{Programming Tips\label{sec:tips}}
\subsection*{Managing the Server Connection}
You'll get into trouble if you aren't careful to close down the X server
connection when your program terminates. The reason? Most X servers have a limit
to the number of client connections which they can serve at one time. When the
limit is reached, the server simply refuses to open any new connections, and
you'll end up in the error handler staring at an ``Unable to connect'' message.  Make
sure your program
{\em always} terminates with a call to {\tt close-display},\index{close-display}
even when it aborts unexpectedly. The best way is to put your event loop inside
an {\tt unwind-protect} form:

\begin{verbatim}
...
(unwind-protect 
  (catch :event-loop
    (loop
      (process-next-event display)))  
  (close-display display))
...
\end{verbatim}

One nifty thing about using this technique is that it's always safe to
abort out of a damaged program and start over.

Also, you should generally avoid binding {\em global} special variables
to {\tt contact-display} objects representing open server connections. It's too
easy to lose track of a global variable or, even worse, to garbage a {\tt
contact-display} while it's still open.\index{contact-display}

\subsection*{Debugging}
\index{debugging}
Debugging a CLUE program (or, indeed, any program using the X Window System)
requires an awareness of a simple fact of life: {\bf client-server communication
is buffered}. Just calling the CLX {\tt draw-line} function won't necessarily
cause a line to appear on the screen. Instead, CLX places the corresponding
server request into an output buffer and moves on; the request is sent to the
server and executed later, when the output buffer is flushed. Normally, you
don't have to worry about this because CLX will automatically flush the output
buffer at the ``right'' time\footnotemark\footnotetext{When the output buffer is
full and (by default) before reading the next input event.}. However, during
debugging, you may want requests to be executed immediately. There are
two ways to do this.

\begin{enumerate}
\item Call {\tt display-force-output} manually at the appropriate time.
\index{display-force-output}
\item Invoke {\tt (setf (dis\-play-af\-ter-func\-tion dis\-play)
\#'dis\-play-force-out\-put)}. \\
This tells CLX to flush the
output buffer automatically as soon as each request
is made.
\index{display-after-function} 
\end{enumerate}

\index{errors}
Output buffering also means that when a request is in error, you won't
see the error immediately. The X server reports an error by sending an {\bf
error reply} back to the program (in this sense, an error reply is another kind
of input ``event'').
The error will be reported after the invalid
request is sent to the server, but by this time, your program has usually sent a
number of other requests, too. The result? When you arrive at the error handler,
you almost never be anywhere close to the real scene of the crime.
Using the {\tt display-after-function}\index{display-after-function} for
``single-step'' output, as shown above, can help to mitigate this problem. But
at any rate, you will need to pay careful attention to the contents of the error
reply which is printed out in CLX's error message. Sometimes, your best recourse
is to compare the error
reply with the error behavior defined by the X Protocol Specification for the
offending request and thereby deduce the cause of the bug.

Another useful debugging technique is to (locally) bind your {\tt
contact-display} and important {\tt contact} objects to special variables within
your program. This makes it easier to {\tt break} at a convenient
point (say, while the program is idle, waiting for an event to occur) and then
examine the state of various contacts or make server requests to return
interesting data.

\section{Getting Started\label{sec:getting-started}}
\subsection{Building CLUE}
If you're an Explorer user, you'll want to use an Explorer Release 6 load
band. This band has the latest versions of CLOS, CLX, and CLUE already built in.
If this applies to you, you can stop here --- you're ready to go!

But if your system doesn't come with CLUE built in, you'll need to go through
the following procedures to get things properly loaded.
Three systems need to be loaded: CLOS\index{CLOS}, CLX\index{CLX}, and CLUE
itself. CLX needs to be loaded specially  --- a small patch to the standard X11
R3 version of CLX is needed to make it work with CLUE.

\begin{enumerate}
\item Decide which version of CLOS you want. Possible choices include:

\begin{itemize}
\item {\tt PCL}, the Portable Common Loops implementation from Xerox that has
recently been made compatible with the CLOS specification. See CLUE release
notes for more details.\index{PCL}
\item {\tt clos-kludge}, a simple implementation of a CLOS subset which comes
with
the other CLUE software and which is sufficient for getting started with CLUE.
See the {\tt defsystem} file in the {\tt clos-kludge} directory.\index{clos-kludge}
{\tt clos-kludge} works but beware: you should use this only as a temporary
stop-gap until you have a ``real'' CLOS in place.
\item Something else. Maybe you've done a CLOS implementation yourself! Or maybe
you've gotten one from the vendor of your Common Lisp.
\end{itemize}

\item Compile and load your preferred CLOS.

\begin{center} \framebox[5.5in]{
\hspace*{\fill}
{\bf Note}
\hspace*{\fill}
\parbox[t]{4.5in}{
Make sure that before both CLX and CLUE are {\em compiled}, you have loaded your
preferred version of CLOS. If you switch CLOS versions later, be sure to
recompile both CLX and CLUE.} \hspace*{\fill}} \end{center}


\item Compile and load CLX, with the special CLUE patches with are included with
the other CLUE software.  See the CLUE {\tt defsystem} file.  Lisp machine
users may find it convenient to perform {\tt (make-system 'clx-clos)}.

\item Compile and load CLUE. See the CLUE {\tt defsystem} file.  Lisp machine
users may find it convenient to perform {\tt (make-system 'clue)}

\end{enumerate}

     
  
\subsection{Packages}
\index{packages, CLUE}
\index{packages, CLUEI}\index{intrinsics}
All symbols defined by the CLUE ``intrinsics'' are external in the {\tt
CLUEI} package.  The {\tt CLUE} package exports all symbols defined by the
standard contact set, 
\index{contact, standard set}
in addition to all of the
``intrinsics'' in {\tt CLUEI}\footnotemark\footnotetext{For now, the standard
contact set is undefined so using either the {\tt CLUE} or {\tt CLUEI}
package is effectively equivalent. Similarly, the CLUE specification currently
describes
only the intrinsics and the CLUE package, and does not yet mention the
standard contact set.}.

All symbols defined by CLX\index{CLX} are external in the {\tt XLIB} package. The {\tt
CLUE} and {\tt CLUEI} packages both use {\tt XLIB}.

The package containing CLOS\index{CLOS} symbols may vary, depending on which
implementation of CLOS you are using. Usually, you will want to use the {\tt
CLOS} package which exports the Explorer CLOS system. 

\index{packages, CLOS}\index{packages, LISP}\index{packages, TICL}
\begin{center}
\framebox[5.5in]{
\hspace*{\fill}
{\bf Note}
\hspace*{\fill}
\parbox[t]{4.5in}{
If you are using the Explorer {\tt CLOS} package, be sure to
also use only {\tt LISP}, the standard Common Lisp package, not {\tt TICL}.
Using both {\tt TICL} and {\tt CLOS} together will result in some nasty name
collisions (for example, on {\tt make-instance} and {\tt defmethod}.)}
\hspace*{\fill}} \end{center}


Application programmers should define a separate package for application symbols
which uses either:
\begin{itemize}
\item {\tt LISP}, {\tt CLOS} and {\tt CLUE}, or
\item {\tt LISP}, {\tt CLOS}, {\tt CLUEI}, and a non-standard contact package.
\end{itemize}

Contact programmers should define a separate package for each set of
related contacts, which uses either: 
\begin{itemize}
\item {\tt LISP}, {\tt CLOS}, {\tt XLIB}, and {\tt CLUE}, (if using standard
contacts), or 
\item {\tt LISP}, {\tt CLOS}, {\tt XLIB}, and {\tt CLUEI}.
\end{itemize}

\subsection{Warming Up}
Here are some CLUE calisthenics to limber you up. Some of them
will require further study of CLUE, CLX, and the X Window System. Some of them
are rather challenging.

\begin{enumerate}
\item Complete the {\tt blinker} contact example. Write a {\tt display} method
that will fill the {\tt blinker} with its {\tt color} when its {\tt on-p}
state is
true and fill it with its {\tt background} otherwise. Ensure
that the {\tt blinker-on-p} accessor updates the display correctly.

\item Create two  {\tt blinker} instances --- one that will print out either
``{\tt Off!}'' when
its {\tt :blink} callback is called with a {\tt nil} argument or ``{\tt On!}''
otherwise, and another which will make a funny noise on your Explorer {\em only}
when its {\tt :blink} callback is called with a non-{\tt nil} argument.

\item Try out different contact attributes on your {\tt blinker}. For example,
try a different {\tt border-width}, {\tt border}, and {\tt background}. Try
initializing the {\tt background} to a {\tt '(float 0 1)} value. What happened?
Why?

\item Define resources to change initial {\tt blinker} attributes.

\item Create two ``funny noise'' {\tt blinker} instances, as in the previous
exercise, but give them different funny noises. Do {\em not} change the
definition of the {\tt blinker} class.

\item Change the {\tt blinker} class so that whenever the cursor enters a {\tt
blinker}, it turns into a picture of Gumby. Hint: use the {\tt xfd} program
(font displayer) to find an appropriate element of the {\tt cursor} font.
Another hint: Use ``{\tt man xfd}'' to learn how to use {\tt xfd}.

\item Change the {\tt blinker} class to include a {\tt string} slot, containing
a string which can be displayed in any font. Display the string so that it
appears in the {\tt blinker}'s {\tt color}, is centered horizontally and
vertically within the {\tt blinker}'s current interior size, and is visible
regardless of the {\tt blinker}'s {\tt on-p} state. Ensure that the {\tt string}
remains centered when the user changes the font or the size of the {\tt
blinker}.

\item Define a {\tt composite} class that {\em always} has a single {\tt blinker}
child, which is always centered horizontally and vertically within the {\tt
composite}. Ensure that the {\tt blinker} child
remains centered when the user changes the size of the {\tt composite}.

\item Define an {\tt etch-a-sketch} contact that will:
\begin{itemize}
\item 
Rubberband a horizontal
line as the user moves the pointer with {\tt :button-1} down, then
snap it into place when the button is released.
\item Rubberband a vertical
line as the user moves the pointer with {\tt :button-2} down, then
snap it into place when the button is released.
\item Print out a copy of its current line drawing on the Imagen
when the {\tt \#$\backslash$return} key is pressed.

\end{itemize}

\end{enumerate}
\pagebreak
\begin{thebibliography}{9}

\bibitem{clos} Bobrow, Daniel G., et al. The Common Lisp Object System
Specification (X3J13-88-002). American National Standards Institute, June,
1988. 

{\em [The definition of CLOS. Start with Chapter 1, which is a thorough and
readable explanation of all concepts. Then consult with Chapter 2 for
programming details.]}

\bibitem{clos-guide} Keene, Sonya E. {\em Object Oriented Programming in Common
Lisp}. Addison-Wesley (1989).

{\em [A textbook-style guide to CLOS, written by a member of its ANSI
development team. Discusses all CLOS features, with examples.]}

\bibitem{Xt} McCormack, Joel, et al. The X-Toolkit Intrinsics, Version
11, Revision 2 (March, 1988).

{\em [Xt, the C language counterpart to CLUE. This is important because
CLUE will tend to evolve in the same direction as Xt.]}

\bibitem{x-protocol} Scheifler, Robert W. The X Window System Protocol, Version
11, Revision 3 (December, 1988).

{\em [The X Bible. The definitive word on how X works. A reference manual with
no tutorials or examples.]}

\bibitem{x-window-system} Scheifler, Robert W. and Gettys, Jim. The X Window
System. {\em ACM Transactions on Graphics}, Vol. 5, No. 2 (April 1986).

{\em [An excellent technical overview of the design and features of the X Window
System. Somewhat obsolete (written before X Version 11) but still informative.]} 

\bibitem{clx} Scheifler, Robert W., et al. CLX Interface
Specification, Version 4 (September 1987).

{\em [Terse, but this is the definition of CLX. Describes only the language
binding, so it must be read in conjunction with the X Protocol Specification.]}

\end{thebibliography}


\pagebreak
\begin{theindex}
% -*- Mode:TeX -*-
%%%----------------------------------------------------------------------------------+
%%%                                                                                  |
%%%                          TEXAS INSTRUMENTS INCORPORATED                          |
%%%                                  P.O. BOX 149149                                 |
%%%                             AUSTIN, TEXAS 78714-9149                             |
%%%                                                                                  |
%%%                Copyright (C) 1989 Texas Instruments Incorporated.                |
%%%                                                                                  |
%%% Permission is granted to any individual or institution to use, copy, modify, and |
%%% distribute this document, provided that  this complete copyright and  permission |
%%% notice is maintained, intact, in all copies.                                     |
%%%                                                                                  |
%%% Texas Instruments Incorporated provides this document "as is" without express or |
%%% implied warranty.                                                                |
%%%                                                                                  |
%%%----------------------------------------------------------------------------------+
\documentstyle[twoside,12pt]{article} 
\pagestyle{plain} 
%%
%% Inserted from home:/usr/local/hacks/tex/setmargins.tex:
%%
%%    Original: Glenn Manuel 12-17-86
%%    Added optional [printer-offset]: Glenn Manuel 2-27-87
%% Sets the margins, taking into account the fact that LaTex
%% likes to start the left margin 1.0 inch from the left edge
%% of the paper.
%%   This macro prints on the screen and in the log file
%%   the values it sets for:
%%     Textwidth, Odd Page Left Margin, Even Page Left Margin,
%%     Marginparwidth, Printer Offset
%%     (the values are in points: 72.27 pts/inch).
%%     The Odd and Even Page Left Margins include LaTeX's
%%     built-in 1.0in offset, but NOT the [printer-offset],
%%     so these values always indicate what the actual
%%     printout SHOULD measure.
%%
%% USAGE:  Place the following between the
%%         \documentstyle   and  \begin{document} commands:
%% \input{this-file's-name}
%% \setmargins[printer-offset]{line-length}{inside-margin-width}
%%
%%   where ALL arguments to \setmargins are DIMENSIONS
%%           like {6.5in}, {65pt}, {23cm}, etc.
%%   The units MUST BE SUPPLIED, even if the dimension is zero {0in}.
%%
%%   [printer-offset] is optional.  Default is zero.
%%
%%   Examples:
%% \setmargins{0in}{0in}         % default line length & default margins
%% \setmargins[-.12in]{0in}{0in} % compensate for printer offset
%% \setmargins{0in}{1in}         % default line length, 1 inch inner margin
%% \setmargins{6.5in}{0in}       % 6.5 inch line length & default margins
%% \setmargins{5in}{1.5in}       % 5 inch line length & 1.5 inch inner margin
%%
%%   {inside-margin-width} is defined as follows:
%%       For 1-sided printing: left margin for all pages.
%%       For 2-sided printing: left  margin for odd pages,
%%                             right margin for even pages.
%%
%%   Defaults:
%%   Each argument has a default if {0in} is used as the argument:
%%      line-length default = 6.0in
%%      inside-margin-width default:
%%         For 1-sided printing, text is centered on the page
%%                     (each margin = [1/2]*[8.5in - line length]);
%%         For 2-sided printing, inside margin is twice the outside margin
%%                     (inside  margin = [2/3]*[8.5in - line length],
%%                      outside margin = [1/3]*[8.5in - line length]).
%%      printer-offset default = 0in
%%
%%  For all cases, the outside margin (and marginparwidth, the
%%  width of margin notes) is just whatever is left over after
%%  accounting for the inside margin and the line length.
%%
%% Note: LaTeX's built-in offset of 1.0 inch can vary somewhat,
%%       depending upon the alignment of the Laser printer.
%%       If you need it to be EXACT, you will have to supply
%%       the optional [printer-offset] argument.
%%       Subtract the actual measured left margin on an
%%       odd-numbered page from the printed Odd Page Left Margin
%%       value, and use the result as the [printer-offset].
%%       Positive values shift everything to the right,
%%       negative values shift everything to the left.
%%
\makeatletter
\def\setmargins{\@ifnextchar[{\@setmargins}{\@setmargins[0in]}}
\def\@setmargins[#1]#2#3{
%%%  Uses temporary dimension registers \dimen0, \dimen2, \dimen3, \dimen1
    \dimen1=#1                         % 1st argument [printer offset]
    \dimen2=#2                         % 2nd argument (line length)
    \dimen3=#3                         % 3rd argument (inner margin)
     \advance\dimen1 by -1.0in         % for LaTeX built-in offset
    \ifdim\dimen2=0in 
        \textwidth=6in  \dimen2=6in
    \else \textwidth=\dimen2
    \fi
    \dimen0=8.5in
    \advance\dimen0 by -\dimen2         % 8.5in - line length
    \if@twoside
       \ifdim\dimen3=0in  % use defaults: 2/3 inside, 1/3 outside
          \divide\dimen0 by 3           % (8.5in-line length)/3
          \dimen2=2\dimen0              % (2/3)*(8.5in-line length)
          \oddsidemargin=\dimen2
          \advance\oddsidemargin by \dimen1   % add in offset
          \dimen2=\dimen0               % (8.5in-line length)/3
          \evensidemargin=\dimen2
          \advance\evensidemargin by \dimen1  % add in offset
%  allow for space on each side of marginal note
          \advance\dimen0 by -2\marginparsep
          \marginparwidth=\dimen0
       \else            % use supplied 2-sided value
          \oddsidemargin=\dimen3             % inside-margin-width
          \advance\oddsidemargin by \dimen1  % add in offset
          \advance\dimen0 by -\dimen3   % 8.5in-line length-inside margin
          \evensidemargin=\dimen0
          \advance\evensidemargin by \dimen1 % add in offset
%  allow for space on each side of marginal note
          \advance\dimen0 by -2\marginparsep
          \marginparwidth=\dimen0
       \fi
%  one-sided
    \else \ifdim\dimen3=0in  % use defaults: center text 
              \divide\dimen0 by 2         % (8.5in-line length)/2
              \oddsidemargin=\dimen0      % (8.5in-line length)/2
              \advance\oddsidemargin by \dimen1   % add in offset
              \evensidemargin=\dimen0     % (8.5in-line length)/2
              \advance\evensidemargin by \dimen1  % add in offset
%  allow for space on each side of marginal note
              \advance\dimen0 by -2\marginparsep
              \marginparwidth=\dimen0
          \else  % use supplied values
              \advance\dimen0 by -\dimen3  % 8.5in-line length-left margin
%  allow for space on each side of marginal note
              \advance\dimen0 by -2\marginparsep
              \marginparwidth=\dimen0
              \advance\dimen3 by \dimen1   % add in offset
              \oddsidemargin=\dimen3
              \evensidemargin=\dimen3
          \fi
    \fi
  \immediate\write16{Textwidth = \the\textwidth}
  \dimen0=1.0in
  \advance\dimen0 by \oddsidemargin
  \immediate\write16{Odd Page Left Margin = \the\dimen0}
  \dimen0=1.0in
  \advance\dimen0 by \evensidemargin
  \immediate\write16{Even Page Left Margin = \the\dimen0}
  \immediate\write16{Marginparwidth = \the\marginparwidth}
  \dimen0=#1
  \immediate\write16{Printer Offset = \the\dimen0}
 }
%
\def\@outputpage{\begingroup\catcode`\ =10 \if@specialpage 
     \global\@specialpagefalse\@nameuse{ps@\@specialstyle}\fi
     \if@twoside 
       \ifodd\count\z@ \let\@thehead\@oddhead \let\@thefoot\@oddfoot
                       \let\@themargin\oddsidemargin
% treat page 0 (title page) as if it is an odd-numbered page
        \else \ifnum\count\z@=0 \let\@thehead\@oddhead \let\@thefoot\@oddfoot
                                \let\@themargin\oddsidemargin
              \else \let\@thehead\@evenhead
                    \let\@thefoot\@evenfoot \let\@themargin\evensidemargin
     \fi\fi\fi
     \shipout
     \vbox{\normalsize \baselineskip\z@ \lineskip\z@
           \vskip \topmargin \moveright\@themargin
           \vbox{\setbox\@tempboxa
                   \vbox to\headheight{\vfil \hbox to\textwidth{\@thehead}}
                 \dp\@tempboxa\z@
                 \box\@tempboxa
                 \vskip \headsep
                 \box\@outputbox
                 \baselineskip\footskip
                 \hbox to\textwidth{\@thefoot}}}\global\@colht\textheight
           \endgroup\stepcounter{page}\let\firstmark\botmark}
%
\makeatother
%%
%% End of home:/usr/local/hacks/tex/setmargins.tex:
%%
\setmargins{6.5in}{1in}
\topmargin = 0in
\headheight = 0mm
\headsep = 0mm
\textheight = 9in
\makeindex
\begin{document}
\setlength{\parskip}{6 mm}
\setlength{\parindent}{0 in}

\title{A Quick and Dirty Guide to CLUE}
\author{Kerry Kimbrough}
\date{Version 6.0 \\
July, 1989\\[2 in]
\copyright 1989\  Texas Instruments Incorporated\\[.5in]
\parbox{3.5in}{
     Permission is granted to any individual or institution to use,
     copy, modify and distribute this document, provided that  this
     complete  copyright  and   permission  notice   is maintained,
     intact, in  all  copies  and  supporting documentation.  Texas
     Instruments Incorporated  makes  no  representations about the
     suitability of the software described herein for any  purpose.
     It is provided ``as is'' without express or implied warranty.
}}
\maketitle

This guide is a no-frills introduction to programming with  the Common Lisp User
Interface Environment (CLUE). CLUE is a high-level programming interface to the
X Window System which extends (but does not supersede) the basic CLX\index{CLX} interface.
CLUE is also an object-oriented programming system based on the Common Lisp
Object System (CLOS\index{CLOS}). Moreover, CLUE is a ``toolkit'' for constructing X user
interfaces. As a result, CLUE is modelled closely on the standard toolkit used
by C programmers, commonly known as Xt\index{Xt}, or the X Toolkit.

All of these related systems are described in separate documents (these
references are listed at the end of the guide), but we won't spend much time on
them here. Read this guide and take the plunge into CLUE; you can dry off with
the details later.

\section{The Big Picture}
Before you can understand CLUE, you have to look at three ideas about how
to build an interactive application. 


\subsection{The Application Is Event-Driven} 
An interactive application is controlled by a human user. Typically, nothing
happens until the user lifts his finger to the side of his mouse and causes
an input {\bf event}.  There are two implications here.
\index{event} 

First, the fundamental
application control structure is an {\bf event loop}: wait for an event, figure
out how to handle it, process the event, then go back and wait for the next one.
The {\em structure} of the event loop is generic; there is nothing
application-specific about it. 
\index{event, loop}

Second, the application is passive. It knows how to do certain things, but it's
waiting for the user's command to do them. You can imagine programs that do
things on their own, without waiting for user instruction (for example, a
nuclear reactor control system). But, the {\em interactive} part of such a
program is still an event loop, which must synchronize with the separate
background task(s) to provide the user control.


\subsection{The UI Is Separate} 
The user interface part has nothing to do with the essential function of the
application. An oft-cited analogy is linguistic communication, which has
independent modules for lexical, syntactic, and semantic processing. You can
understand me whether I speak to you or whether I write to you, because these
are merely two different lexical forms (e.g. sound patterns or marks) for the
same language.  Also, you can express the same idea either in French or in
English; the lexical and syntactic elements of each language are distinct from
their corresponding meaning, or semantics. 

An interactive application can be modularized along the same lines. The UI
contains the lexical and syntactic components, while the application {\em per
se} is the semantic component. Take some application command, say, ``Stop
Nuclear Reactor.'' The user interface to this command could use various lexical
forms (press the {\tt STOP} key or type the string ``{\tt stop}'') and
various syntactic forms (prompt with ``{\tt Are you sure you want to stop?}'' or
not), but the application semantics remain the same (hopefully!). From here on,
we will
acknowledge this modularization by referring to the whole application program as
``the program,'' to the user interface
components as ``the UI,'' and to the application semantic components as ``the
application.''

Separating the UI and semantic modules of an interactive program is good
programming technique. This allows you to more easily create a new UI for an
existing application. Or to apply an existing set of UI techniques to a new
application, so that it is consistent in style with others. Or to divide the
programming effort more efficiently; a domain expert can write the application
semantics while a human factors specialist can design the UI and a window system
hacker can implement it. 

\subsection{The UI Is Object-Oriented} This idea is more recent than the
other two, because it wasn't until a few years ago that graphical user
interfaces and window systems came along.  When they did, UI's began to
be composed of multiple user activities represented by visually-distinct
and physically-manipulable objects on the screen.  Object-oriented
programming (OOP) is thus a natural programming style which has been
associated with UI since its beginning\footnotemark\footnotetext{In
fact, true OOP and window systems were invented at the same time at
Xerox PARC in the Smalltalk system.}.  For several reasons, OOP is a
nice fit for UI programming.  UI's tend to generate profuse variation of
detail around a basic theme (consider, for example, menus); that is,
they demonstrate several subclasses of a basic class.  The modularity
of objects also helps to separate the UI from the semantics; that is,
the implementation of UI methods from the program which simply calls
them.

The OOP methodology leads to a UI which is composed of user interface objects
(command lines, messages, menus, scroll bars, dialogs, etc.). Each of these UI
objects is an ``agent'' for some portion of the  underlying application.
Each UI object is responsible for presenting some ``view'' of the application to
the user. It is also responsible for receiving the users events which are
directed to it and for translating these into the appropriate application
responses. 


\subsection{How CLUE Works\label{sec:how-clue-works}} 
CLUE combines all of these ideas. A CLUE program consists of an event loop, a
set of UI objects called {\bf contacts}, and a set of application functions
called {\bf callbacks}. All CLUE programs look pretty much like this:
\index{event, loop}

\begin{verbatim}
;; Open a connection to the X server.
(let ((display (open-contact-display 'my-application :host "server-host" ...)))
  ;; Initialize UI objects and callbacks.  
  (setf c (make-contact ... ))
  (add-callback c ... )
       ...
  ;; Process events until the event loop is terminated.
  (catch :event-loop
    (loop
      (process-next-event display)))  
  (close-display display))
\end{verbatim}

A CLUE program creates a {\bf contact-display} object which
\index{contact-display}
represents a connection to a specific X server.  This {\tt contact-display} then
becomes a two-way channel through which the program requests the creation and
display of contact windows and receives input events sent to its contacts. 

Each contact is a UI ``agent'' that is prepared to present some
application information, to accept user
events which manipulate this information, and then to report the results back to
the application. This reporting action is the critical connection between the UI
and the application, and it's done using {\bf callbacks}.\index{callback}
\index{callback, name}\index{callback, function}
A callback consists of
a {\bf callback name} and an associated {\bf callback function}. A contact is programmed
to associate a specific result with a callback name, which it reports by calling
the associated callback function (with a predefined argument list). 

Here is where CLUE contributes to the separation of UI and application. 
\index{application programmer}\index{contact programmer}
Every
CLUE program usually has two programmers! A {\bf contact programmer} is one
who defines a contact class and implements its methods. An {\bf application
programmer} is one who decides to create a contact instance and employ it for a
particular UI role. The contact programmer decides what callback names his
contact class will use and what the argument lists to the associated callback
functions have to be. The application programmer defines callback functions to
perform specific jobs and plugs them into the appropriate contact callback.
The contact programmer will frequently use the basic CLX\index{CLX} interface, typically to
display the contact's information. The application programmer will use CLUE
interfaces but will almost never need to call CLX functions directly.

Now we can see the operation of a CLUE program in a bit more detail. The {\tt
make-contact} function is used to create and initialize contacts. 
\index{make-contact}
The {\tt
add-callback} function \index{add-callback} associates application semantics
(i.e. callback
functions) with each contact. Then, the program enters the event loop. Each call
to {\tt process-next-event} \index{process-next-event} reads the next input event and dispatches it to the
contact to which it belongs. This contact then goes through a process of {\bf
event translation}\index{event translation}, which determines which
contact {\bf action}\index{action} functions will be
executed. Contact action functions, in turn, perform all the details of input
feedback and display. Finally, if the user event has completed an application
result, the action function will look up and call the associated callback
function. What has happened? CLUE has provided all the machinery, the contact
programmer has defined the detailed UI behavior, and the application programmer has
furnished the real results.

The rest of this guide will be a closer look at each of these parts of a CLUE
program.

\section{The Contact Display\label{sec:contact-display}}
\index{contact-display}
\index{open-contact-display} 
A {\tt contact-display} is created by the {\tt
open-contact-display} function.  {\tt op\-en\--con\-tact\-dis\-play} has one required
argument: a symbol which acts as the name of your program.  Technically, this is
a {\bf resource name}\index{resource, name} which is used to access the values of
{\bf resources}\index{resource} associated with your program.  CLUE resource
management is explained later in Section~\ref{sec:resources}.  Generally
speaking, this program resource name is used to distinguish your program from
other application programs that may also be running. {\tt open-contact-display}
has several optional keyword arguments (mainly the same ones used by the CLX\index{CLX}
{\tt open-display} function), but the most frequently used is the {\tt host}, a
string which gives the network name for the X server host to which you are
connecting.

\begin{verbatim}
(setf display
      (open-contact-display
        'blink                    ; Application name
        :host "server-host"))     ; Server host name
\end{verbatim}

{\tt contact-display} objects are really a subclass of the {\tt display} data
type defined by CLX\index{CLX}\footnotemark\footnotetext{However, neither {\tt display}
objects nor {\tt contact-display} objects are actually implemented as CLOS
classes. There's no reason to define {\tt contact-display} methods or
subclasses.}.
\index{display}

\begin{center}
\framebox[5.5in]{
\hspace*{\fill}
{\bf Note}
\hspace*{\fill}
\parbox[t]{4.5in}{
Any CLX\index{CLX} function which takes a {\tt display} argument can be called with a {\tt
contact-display} argument instead.}
\hspace*{\fill}}
\end{center}

The CLX function {\tt close-display}\index{close-display} should always be used
to close the server connection when the program terminates.

\section{Contacts\label{sec:contacts}}
In simplest terms, a {\tt contact} is a kind of window. The behavior of a window
object,
which is represented in CLX\index{CLX} as an instance of the
{\tt window} data type,
is completely specified by the X Window System
protocol.  In CLUE, the class of {\tt contact} objects is defined to be
a subclass of the class of {\tt window} objects\footnotemark\footnotetext{To do
this, CLUE actually changes the definition of the CLX {\tt window} type to
make it a CLOS class.}.  Thus, all CLX {\tt window} functions
apply to any {\tt contact}. 

\begin{center}
\framebox[5.5in]{
\hspace*{\fill}
{\bf Note}
\hspace*{\fill}
\parbox[t]{4.5in}{
Any \index{CLX}CLX function which takes a {\tt window} argument can be called with a {\tt
contact} argument instead.}
\hspace*{\fill}}
\end{center}


As a window, a contact inherits several basic attributes,
such as a
rectangular geometry (given by the position of its upper left corner, its width,
and its height in
pixel coordinates), a border width and border color/pattern, a depth (i.e.
bits per pixel), and a background color/pattern.
\index{contact, attributes}
Contacts possess a few additional attributes (or {\bf slots}) not shared
by ordinary windows. 
\begin{itemize}
\index{contact, name}
\item {\tt name}: The name of
the contact is a
symbol  which can be used to access contact resources
stored in a resource database. See Section~\ref{sec:resources}.

\index{contact, parent}
\item {\tt parent}: All
contacts have a parent contact (except for a {\bf root}\index{root} contact, which is
created automatically by CLUE and which is the ancestor of all contacts on a
single display screen).  The parent-child relationship
is fundamentally one between two windows, as defined by the X
Window System. For example, a child window is always visually
contained
by its parent window, and its position is specified relative to its
parent's upper-left corner. However, in CLUE this relationship is
extended, with the parent assuming additional responsibilities, such as
geometry management (see Section~\ref{sec:composites}).

\index{contact, state}
\item {\tt state}: A switch which controls the visual effect of the contact on
the UI. If the {\tt state} is {\tt :withdrawn}, then the contact is invisible and
unavailable for input. If the {\tt state} is {\tt :mapped}, then the contact is
``viewable'' (i.e. visible unless it is covered by other windows) and available
for input. The {\tt state} can also have a third value --- {\tt :managed} ---
which is discussed in
Section~\ref{sec:resources}. The default {\tt state} is {\tt :mapped}.


\index{contact, event-translations}
\item {\tt event-translations}: A list which associates
different types of input events with contact actions. This is discussed in more
detail in Section~\ref{sec:events}.

\index{contact, callbacks}
\item {\tt callbacks}: An association list of callback name/function pairs.
\index{callback}
\end{itemize}

\subsection{Using Contacts}
\index{contact, application programmer interface}
If you are an application programmer, your basic view of a contact --- the generic
contact protocol --- is quite simple and consists of only a few functions.
\index{contact, generic protocol}

{\tt make-contact} 
\index{make-contact}
creates a new instance of any con\-tact subclass. This function is a
syn\-tac\-ti\-cal\-ly-identical extension of the basic CLOS\index{CLOS} {\tt make-instance}
function. Keyword arguments to  {\tt make-contact}  provide initial slot values
for the new instance, most of which can be defaulted. However, a {\tt parent}
argument must be given; specifying a {\tt contact-display} as the {\tt parent}
creates a {\bf top-level contact}\index{contact, top-level}. 

\begin{center}
\framebox[5.5in]{
\hspace*{\fill}
{\bf Note}
\hspace*{\fill}
\parbox[t]{4.5in}{
A contact ``belongs'' to the {\tt contact-display} connection on which it is
created. More precisely, since a contact must have a top-level ancestor, it
belongs to the {\tt contact-display} given as this ancestor's {\tt parent}.
This {\tt contact-display} connection receives all requests and events related
to the contact; it is also available as the value of the contact's {\tt display}
slot.} \hspace*{\fill}} \end{center}


Usually (and always
for top-level contacts), you must also specify the contact's {\tt width} and {\tt
height}. A non-top-level contact may find that its initial values for 
{\tt x}, {\tt y}, {\tt width}, {\tt
height}, and {\tt border-width} have been modified by the geometry management
policy of its parent (see Section~\ref{sec:composites}).

\begin{verbatim}
(setf a-blinker
      (make-contact
        'blinker                  ; Make an instance of the blinker class
        :parent display           ; Required argument
        :width  300               ; Usually required
        :height 400               ; Usually required
        :background 0))           ; Optional, defaults to parent's
\end{verbatim}



The {\tt (setf contact-state)} function 
changes the {\tt state} value of the contact.
\index{contact, state}

\begin{verbatim}
(setf (contact-state a-blinker) :mapped)    ; Make it viewable
(setf (contact-state a-blinker) :withdrawn) ; Make it invisible
\end{verbatim}

{\tt change-geometry} 
\index{change-geometry} 
requests a change to one or more components of the contact's geometry.
The keyword
arguments to {\tt change-geometry} --- {\tt x}, {\tt y}, {\tt width}, {\tt
height}, and {\tt border-width} --- specify the changed component(s); omitting a
keyword means ``Leave the current value unchanged.'' But the actual effect of
the request depends on the geometry management\index{geometry management}
policy applied to the contact (See Section~\ref{sec:composites}). If the
request is refused, then the return values give  acceptable alternatives for {\tt x}, {\tt y}, {\tt width}, {\tt
height}, and {\tt border-width}.  You may want to give an {\tt accept-p} keyword
value of {\tt t}, which means ``Go ahead and replace my request with whatever
alternative geometry may be returned.''

\begin{verbatim}
(change-geometry a-blinker
                 :x 100           ; Request new upper-left position...
                 :y 200
                 :accept-p t)     ; ...but accept closest alternative

(change-geometry b :accept-p t)   ; Just validate current values
\end{verbatim}

\begin{center}
\framebox[5.5in]{
\hspace*{\fill}
{\bf Note}
\hspace*{\fill}
\parbox[t]{4.5in}{
There are no accessor functions for {\tt setf}'ing a contact's {\tt x}, {\tt y}, {\tt width}, {\tt
height}, and {\tt bor\-der\--width} slots. Never modify these slots directly. Always
use {\tt change-geometry} instead.} 
\hspace*{\fill}}
\end{center}


The {\tt change-priority} function \index{change-priority}
requests a change to the contact's stacking priority relative to other
windows.  The stacking priority determines which window is ``on top'' of other
windows. This request is also subject to geometry management policy. 

The {\tt destroy} function is called \index{destroy}
only when the contact will no longer be referenced. This frees any display
resources allocated to the contact.

In addition to the generic protocol, the application programmer interface to a
contact includes its callback protocol, plus any special class-dependent functions (typically, for
initialization).

\subsection{Defining Contacts}

If you are a contact programmer, you will use the {\tt defcontact} macro to
define a new contact class.  The syntax of {\tt defcontact} is nearly identical
to that of the basic CLOS\index{CLOS} {\tt defclass} macro.  The only difference is the
additional {\tt :resources} option for specifying contact resources (see
Section~\ref{sec:resources}).

\begin{verbatim}
(defcontact
  blinker                         ; Class name  
  (contact)                       ; Superclasses  
  ((color                         ; Slot specs, with name...
     :type     pixel              ; ...data type...
     :accessor blinker-color      ; ...accessor function name ...
     :initarg  :color             ; ...initarg keyword for make-contact...
     :initform 0)                 ; ...and default initial value.
   (on-p
     :type     boolean
     :accessor blinker-on-p
     :initform nil))  
  (:resources color))             ; Resource specs
\end{verbatim}

A contact programmer must also define a {\tt display} method for a new contact
class. CLUE calls the {\tt display} function automatically whenever any portion
\index{display}\index{contact, exposure}\index{exposure}
of the contact image must be (re)displayed. In particular, {\tt display} is
called whenever an invisible contact is changed to the {\tt :mapped} state, or
when some previously-hidden part of a {\tt :mapped} contact is exposed.

\begin{verbatim}
(defmethod                        ; Define a method ...
  display                         ; ... for the display function...
  ((contact blinker)              ; ... when the contact arg is a blinker.
   &optional x y width height)    ; Defines the rectangular piece exposed.
  ...)
\end{verbatim}

\section{Composites and Geometry Management\label{sec:composites}}
A contact which is the parent of another contact is known as a
{\bf composite}\index{composite} and is an instance of the {\tt composite}
subclass of {\tt contact} objects. A {\tt composite} may be the parent of another
{\tt composite},
leading to a tree-structured contact hierarchy. A contact is said to be an
{\bf ancestor}\index{ancestor} of another contact (its 
{\bf descendant}\index{descendant}) when
it is its parent or an ancestor of its parent. 

A {\tt composite} represents a set of contacts which can be manipulated
(positioned, presented,
etc.) as a unit.  A {\tt composite} is useful whenever several contacts act in
concert to provide a
single component of the UI.  Typical examples include
``control panels'' and ``dialog boxes'' --- groups of contacts that are presented
together and are used to make related adjustments to application data.  
The fundamental aspects of the contact parent-child relationship are the same as
those of the window hierarchy defined by the X Window System. 
\index{geometry management}
But in CLUE, a {\tt composite} is also expected to act as the {\bf geometry
manager}\index{geometry manager} for its child contacts; that is, to
implement a style of layout for its children. 


Here is how it works.  A request to change the geometry of a contact is
forwarded to the contact's parent, which actually performs the resulting change.
It is important to understand that, due to its constraints, a geometry manager
may not be able to perform a change as requested.  For example, a request to
increase the size of a contact might be refused if its geometry manager enforces
a maximum size.  Even if a requested change cannot be done, a geometry manager
may be able to suggest a slightly different change which would be acceptable.

Placing geometry control in the hands of a geometry manager in this way
has several advantages.
\begin{itemize}
\item   A geometry manager can arbitrate the competing geometry change requests
        of several contacts in order to implement contraints among them.
        
\item   A given layout style can be applied to any collection of contacts.
        
\item   Contact layout style can be changed without the knowledge of individual
        contacts.
\end{itemize}


A composite's geometry management policy applies only to the set of its children
which are {\bf managed}. \index{contact, managed}
An unmanaged child is ignored by its geometry manager. Any
geometry change to an unmanaged child is performed immediately as requested. 
However, changes to a managed child are arbitrated by its parent's geometry
management policy. This can mean, for example, that a change to one child's
position/size/priority can affect that of other children. Indeed, requesting a
bigger size for a child might result in the parent asking {\em its} geometry
manager to become bigger, leading to a ripple of geometry changes throughout
the contact hierarchy.

How do you know if a contact is managed? This is determined by its {\tt
state}.\index{contact, state}
A contact is managed if and only if its {\tt state} is not {\tt :withdrawn}. This means
that a {\tt :mapped} (and viewable) contact is managed. But there is also
another possibility: a contact's {\tt state} may also be {\tt :managed}. This
value represents the (rather rare) case of a contact which is not visible, but
which is nevertheless taken into account by its geometry manager. 

\begin{center}
\framebox[5.5in]{
\hspace*{\fill}
{\bf Note}
\hspace*{\fill}
\parbox[t]{4.5in}{
An
unmanaged child can {\em never} be visible.}
\hspace*{\fill}}
\end{center}


Implementing geometry management is a topic for contact programmers only. If you
define a new {\tt composite} subclass, you will usually need to implement
methods for the following three functions.
\begin{itemize} 
\item {\tt change-layout}: 
\index{change-layout}
This function is called whenever a composite's set of
managed children changes (e.g. by creating or destroying a managed child or by
changing the {\tt state} of a child).

\item {\tt manage-geometry}: 
\index{manage-geometry}
This function is called by {\tt change-geometry} to
approve a geometry request. It must either approve the change or return a set of
alternative geometry values which will be acceptable.

\item {\tt manage-priority}: 
\index{manage-priority}
This function is called by {\tt change-priority} to
approve a stacking priority request. It must either approve the change or return
an alternative priority which will be acceptable.

\end{itemize}

\section{Events\label{sec:events}}
\index{event}\index{input}
Handling input events is CLUE's main job, and it's an area where CLUE does a lot
of the work for programmers automatically. Application programmers don't have to
worry about much more than setting up an event loop, as shown in
Section~\ref{sec:how-clue-works}. Contact programmers need to know how to use
CLUE's {\bf event translation}\index{event translation} mechanism in order to
implement the details of a contact's input behavior.

What exactly are these events we're talking about? Fundamentally, an event is
a message to the program from the X server, a packet of data describing
some occurrence that the program ought to be interested in.
Most events
represent an action performed by the user with  the X server's 
\index{input, devices}
keyboard or pointing device (i.e. mouse, tablet,
etc.). Keyboard
actions, for example, can generate {\tt :key-press} and {\tt :key-release}
events. The pointer may be used to generate {\tt :button-press} and {\tt
:button-release} (when the user presses one of the mouse buttons) and {\tt
:motion-notify} events (when the user changes the pointer position). These kind of
{\bf user events}\index{event, user} contain data for the button or key
involved, the coordinates of the pointer position, and other useful things, such as
the up/down state of {\tt :shift}, {\tt :control}, and the other modifier keys.
In addition to user events, there is another important group of {\bf side-effect
events},\index{event, side-effect}
events which occur as an indirect result of other user actions. For example,
when the user causes part of an obscured window to become visible, an {\tt
:exposure}\index{exposure} event may be sent. Other side-effect events include
{\tt :enter-notify}, which signifies that the user has moved the pointer cursor
inside a particular window, and {\tt :focus-in}, which happens when the user has
identified a particular window as the {\bf focus} for keyboard
events.\index{focus}\index{input, focus}
This is just the beginning, but the definition and meaning for all of the events
received by CLUE contacts can be found by reading the X Protocol Specification.


To understand CLUE event processing, let's start with the basic program event
loop\index{event, loop} and follow its operation, step by step.
A call to {\tt
process-next-event} \index{process-next-event}\index{timeout}
gets the ball rolling. {\tt
process-next-event} causes CLUE to read
the next event from the given {\tt contact-display} connection. By default,
{\tt pro\-cess\--next\--ev\-ent} does not return until the next event has been
completely processed. This means that if no event is yet available, then {\tt
process-next-event} will generally wait until one finally comes along. However,
an optional {\tt timeout} argument may be given, which says how long {\tt
process-next-event} will wait before giving up. Set the {\tt timeout} to {\tt 0}
if you don't want {\tt process-next-event} to wait at all. {\tt
process-next-event} returns {\tt nil} if a timeout occurred; otherwise, it
returns {\tt t}.

\begin{verbatim}
(catch :event-loop
  (loop
    (unless 
      (process-next-event display 5)  ; Timeout after waiting 5 sec.

      ;; No events yet -- do something useful
      (do-background-task))))
\end{verbatim}

\begin{center}
\framebox[5.5in]{
\hspace*{\fill}
{\bf Note}
\hspace*{\fill}
\parbox[t]{4.5in}{
{\tt process-next-event} does not return any result of the actual processing of
an event. This means
that all responses of a CLUE program to user input  --- even the termination
of the event loop --- occur as ``side-effects'' of contact event handling.
Always escape the event loop by {\tt throw}'ing to some well-defined tag.}
\hspace*{\fill}}
\end{center}

Inside {\tt process-next-event}, the next thing that happens is that the event is
{\bf dispatched}\index{event,
dispatching}. In other words, CLUE figures out which contact is supposed to
handle the event. This is fairly easy because almost every event message
contains an identifier for the {\tt window} object which is the
``addressee'' of the
event. CLUE knows how to convert this identifier into the corresponding {\tt
contact} object. At this point, the event
is {\bf handled}\index{event, handling} by calling the {\tt handle-event}
function\index{handle-event} with the receiving {\tt contact} and the event as
its arguments. {\tt handle-event} then implements the process of event
translation, i.e. figuring out which methods of the receiving {\tt contact} to
call in response to the event\footnotemark\footnotetext{{\tt
handle-event} is also the function which automatically calls the
contact's {\tt display} method, if the event is an {\tt :exposure}.\index{display}\index{exposure}}.
What happens next is of interest only to contact programmers.

\subsection{Event Translation}
Event translation involves searching through one or more association lists,
looking for an entry that matches the event. Each such entry is called an {\bf
event translation}, and it is a list containing an {\bf event
specification}\index{event, specification} and one or more {\bf
action}\index{action} names. An event specification describes a certain sort of
event, and an action name is simply the name of a special kind of function
--- 
an action function. So, during event translation, the event is compared with an
event
specification, and when it matches, then each of the corresponding action functions are
invoked in sequence. When all of them have completed, then event
translation is done; {\tt handle-event} returns and the event is said to
be ``handled.'' The {\tt process-next-event} function also returns, and we
go back to the top of the event loop to deal with the next event.
But where are these event translations and how are they created? There are two
different ways to create event translations: the {\tt add-event} function and
the {\tt defevent} macro.

{\tt add-event}\index{add-event}
creates an event translation and adds it to the association list
found in the {\tt event-translations} slot of a specific contact. This kind of
event translation is thus an {\bf instance event
translation}\index{event translation,
instance} which affects only one contact instance. An instance event
translation says ``When an event is dispatched to {\em this} contact and it
matches {\em this} event specification, then call {\em these} actions.''

\begin{verbatim}
(add-event a-blinker                 ; When this contact gets...
           '(:key-press #\ctrl-b)    ; ...this character event ...
           'beep)                    ; ...call the beep action.

(add-event a-blinker                 ; When this contact gets...
           '(:key-release #\ctrl-b)  ; ...this character event ...
           '(beep 1))                ; ...call the beep action with an arg.
\end{verbatim}

Note that in place of a simple action name, an event translation can also give a
list containing the action name plus a list of arguments to pass.
The process of event translation starts with the instance event translations in
the {\tt event-translations} slot of the receiving contact. Each entry is
examined in order, until a match is found. What happens if no match is found? In
this case, the process continues to search the event translations defined by the
{\tt defevent} macro.

{\tt defevent}\index{defevent}
creates a  {\bf class event translation}\index{event translation, class}
that applies to every instance of a given class.  A class event
translation says  ``When an event is dispatched to a contact of {\em this} class
and it matches {\em this} event specification, then call {\em these} actions.''

\begin{verbatim}
(defevent blinker                    ; When any blinker gets...
          (:button-press :button-1)  ; ...this button event...
          (blink t))                 ; ...blink on.

(defevent blinker                    ; When any blinker gets...
          (:button-release :button-1); ...this button event...
          (blink nil))               ; ...blink off.
\end{verbatim}

Class event translations are searched for each contact superclass of the
receiving contact, starting with the class of the contact, then continuing back
up the {\bf class precedence list}\index{class precedence list} which this
contact inherits\footnotemark\footnotetext{See the CLOS\index{CLOS} specification for a precise definition of the
class precedence list.}.  The idea is to match a class event translation
for the
most specific class possible.  However, if a matching event translation is {\em
still} not found, then CLUE gives up, and the event is ignored. After
all this, it couldn't be a very important event anyway!

\subsection{Event Specifications}
\index{event, specification}
As you may have noticed, an event specification is usually a list
composed of an event type keyword and some other qualifiers. In fact, an event
type keyword alone is also a valid event specification.
The exact syntax for event specifications is fairly complex. A sophisticated
programmer can even define his own event specification syntax. All of this is
completely spelled out in the CLUE specification. But for now, you can get an
idea of what an event specification is by looking at the following examples.

\begin{verbatim} 
:enter-notify         ; Matches any :enter-notify event

(:motion-notify       ; Matches a :motion-notify event..
  :button-1)          ; ...if :button-1 is down.

(:button-press        ; Matches a :button-press...
  :button-1           ; ...if it's :button-1...
  :shift)             ; ...and the shift key is down.

(:button-release      ; Matches a :button-release...
  :button-3           ; ...if it's :button-3...
  (:shift :control)   ; ...and the shift,control keys are down...
  :all)               ; ...and all other modifiers are up.

(:key-press           ; Matches a :key-press...
  :any                ; ...for any key...
  (:control :hyper))  ; ...when the control,hyper keys are down.

(:button-press        ; Matches a :button-press...
  :button-3           ; ...if it's :button-3...
  :double-click)      ; ...and it's a double-click.
\end{verbatim}

\begin{center}
\framebox[5.5in]{
\hspace*{\fill}
{\bf Note}
\hspace*{\fill}
\parbox[t]{4.5in}{A contact will receive {\em only} the types of events described
by
its event specifications, i.e. by event specifications in its instance and class
event translations. CLUE automatically sets the 
{\bf event mask}\index{event, mask} of the contact window, based on the
contact's event specifications. Event masks are a detail of CLUE internals, but
you can find more information about them in the X Protocol
Specification and the CLUE specification.}\hspace*{\fill}} \end{center}

\subsection{Actions and Callbacks}
\index{action}
An action is a  function which is designed to handle an input
event.  An action therefore implements a particular input behavior exhibited by
each instance of a certain contact class. 
In fact, an action function is typically a CLOS generic function, and actions
usually are defined as methods of a particular contact class. 

\begin{verbatim} 
(defmethod blink ((contact blinker) blink-on-p)
  (with-slots (on-p) contact
    ;; Set internal state variable on/off
    (setf on-p blink-on-p)
    ;; Redisplay based on new on-p state
    (display contact)
    ;; Invoke callback with new state
    (apply-callback contact :blink on-p))
\end{verbatim}

In general, an action represents a well-defined contact behavior that
could be done in
response to {\em any} event. In fact, it's possible to personalize an existing UI by
modifying its event translations so that contact actions are rebound to
the types of events (i.e. event specifications) which
are more to your taste.  In some cases,
however, an action may be designed to handle a very specific event type.

When an action computes a result that is important to an application, it invokes
a callback.\index{callback}
This is done via the {\tt apply-callback} macro.\index{apply-callback}
The example above shows that a {\tt blinker} contact has a callback whose name
is {\tt :blink}, which is called with a single argument, i.e. a boolean
representing the new on/off state of the {\tt blinker}. What significance
this has to the application is unknown to the contact programmer who wrote this
action; it all depends on what callback function the application programmer has
associated with {\tt :blink} for this {\tt blinker} instance.
In fact, the application programmer may have decided not to define a callback
function for {\tt :blink}, in which case {\tt :blink} has no application meaning
at all (and {\tt apply-callback} simply does nothing).
As shown in the {\tt :blink} example,
{\tt apply-callback} is usually (although not
necessarily)
called directly by an action method or somewhere within its dynamic extent.

\begin{center}
\framebox[5.5in]{
\hspace*{\fill}
{\bf Note}
\hspace*{\fill}
\parbox[t]{4.5in}{
Callback names are shared by a class, i.e. referenced by the class's
action methods.  But
callback name/function pairs are instance data and are recorded in a contact's
{\tt callbacks} slot. In our example, every {\tt blinker} is
expected to have a {\tt :blink} callback function, but each {\tt blinker}
usually has a different {\tt :blink} callback function. That is, each {\tt
blinker} usually has different application semantics.} \hspace*{\fill}}
\end{center}

In the previous example, the {\tt blink} action method does not depend on the
event which causes it to be invoked.  But what happens when an action's behavior
depends on information contained in the event?  CLUE handles this by
representing an event as an instance of the {\tt event} class.  An {\tt event}
object has slots which contact various kind of interesting information about the
event.  Since there are many types of X events, there are many different {\tt
event} slots, even though for a given event some slots are irrelevant and
therefore have a {\tt nil} value. You can look in the X Protocol Specification for
a complete description of event slots (and look in the CLX specification to find
out how these slot values are represented in Lisp).

But how does an action access an event object? This is done by using a special
CLUE macro called {\tt with-event}. For example:

\begin{verbatim}
(defmethod beep ((contact blinker) &optional (per-cent-volume 0))
  (with-event (state)
    ;; Was the shift key down?
    (when (member :shift (make-state-keys state))
      ;; Ring server's chime!
      (bell (contact-display contact) per-cent-volume)  
      ;; Invoke callback
      (apply-callback contact :beep))))
\end{verbatim}

The {\tt beep} action will beep only if it determines that the shift key was
down when the event occurred.  It does this by examining the {\tt state} slot of
the event, which defines which modifier keys were pressed at the time of the
event.  Notice the use of the {\tt with-event} macro.\index{with-event} This is
similar to the {\tt with-slots} macro of CLOS\index{CLOS}.  {\tt with-event}
binds slots of the ``current'' event argument within its lexical extent, so that
the action code can refer to them.  But why is the current event object hidden
in this way?  Because this allows for a much more efficient implementation of
{\tt event} objects than would be possible if their structure was fully exposed
to programmers.  So, there!

\subsection{Advanced Input}
\index{input}
Input events are a subject so dear to the heart of CLUE that it
contains many more input programming functions than can be
covered in this guide. Here's a quick list of CLUE's special input
features, all of which are discussed in complete detail in the CLUE
specification.
\begin{itemize}
\item {\bf Before actions},\index{before action}\index{action, before}
action functions which can be set to execute for {\em every} event
dispatched to a contact of a certain class, before event translation begins. 

\item {\bf Timers},\index{timer}
which send special {\tt :timer} events to a contact at a regular,
specified rate.

\item {\bf Synchronizing}\index{event, synchronizing}
event processing, so that a program can stop to process all pending
events before continuing.

\item {\bf
Sensitivity},\index{contact,
sensitivity} which allows a contact's input to be temporarily ``switched
off'' without changing its visibility.

\item {\bf Event compression},\index{event, compression}
which removes certain redundant events for faster performance.
\end{itemize}

\section{Resources\label{sec:resources}}
Imagine that you are the user of an interactive program. Maybe you are
not a programmer, but there are certain things about the UI that you
wish you could change ever so slightly. For example, maybe you wish you
could make the program use a different set of colors for various
objects. Or maybe you're left-handed, and you'd prefer to use the right
mouse button in place of the left one. Or maybe you can't read Japanese,
which happens to be the language all the menu items are written in.
Although most of the UI is defined by programmers  --- contact
programmers who designed the various UI objects and application
programmers who placed them neatly on the screen --- the truth is that
some aspects of the UI ought to be up to the {\em user}. Programmers,
keep your hands off!

Nowadays, thoughtful UI programmers recognize this problem, so CLUE
allows  programmers and users to cooperate in
defining the UI. The basis for this cooperation is {\bf resource
management}. A user can store various UI values as {\bf
resources}\index{resource} in a {\bf resource database}\index{resource,
database}, and CLUE contacts can read these resource values and modify
the UI accordingly. 

\subsection{The Resource Database}
Before a user can assign a resource value to a particular UI object, he
must have a way of identifying it, some kind of  name which
can be referenced outside of the program which creates the object.
Remember that a contact has a {\tt name} slot\index{contact, name}; this
is a symbol that could certainly help identify the contact. Consider
also that a contact is generally part of a nested hierarchy of UI
objects. For example, a paint program might have pattern choice
contact which contains several button contact children, one for each available
pattern style. A more complete name for a button child would be a
list which also includes its parent's name, e.g. {\tt '(pattern-choice
checkered)}. We can extend this notion all the way up the hierarchy to
come up with a {\bf complete resource name}\index{resource, complete
name} for the {\tt checkered} pattern button. This would be a list of
resource name symbols, starting with the resource name for the program
itself (i.e. the required argument to {\tt
open-contact-display}\index{open-contact-display}) and continuing with
symbols for the {\tt root}\index{root} ancestor, the top-level ancestor,
and so on:\index{resource, name}

\begin{verbatim}
    '(paint screen-1 pattern-choice checkered)
\end{verbatim}

Assuming that the {\tt checkered} button has a resource for its label
string named {\tt label} and another resource for the label font named
{\tt font,}
then the complete names for these resources would obviously be:


\begin{verbatim}
    '(paint screen-1 pattern-choice checkered label)
    '(paint screen-1 pattern-choice checkered font)
\end{verbatim}

As you might expect when dealing with long path names like these, it's
often useful to insert ``wildcards'' in the places where you don't want
to be so specific. For example, the following resource name would refer to
the {\tt font} of {\em everything} in the pattern choice contact, regardless of
the screen where it appears:

\begin{verbatim}
    '(paint * pattern-choice * font)
\end{verbatim}

Another way of identifying a contact is by its object class. So, for
example, you could refer to the font of {\em every} {\tt button} object in the
paint program with the following resource name.

\begin{verbatim}
    '(paint * button font)
\end{verbatim}

By associating values with these sorts of resource names in {\bf
resource bindings}\index{resource, binding}, a user can
create a resource database which represents his UI preferences.
CLUE\index{CLUE} provides a {\tt define-resources} macro for adding
resource bindings to the database.\index{define-resources}
\begin{verbatim}
(define-resources
  (paint * button background)               white
  (paint * pattern-choice * font)           helvetica-12
  (paint * pattern-choice checkered label) "Checkered")
\end{verbatim}

CLUE automatically reads the resource database whenever a contact is
created. A contact has a class and possibly other contact superclasses
on its class precedence list\index{class precedence list}. Each such
(super)class has a list of resource names that it uses (defined by the
{\tt :resources} option of the defining {\tt
defcontact}\index{defcontact}
form). Together, these form the set of contact resources that are looked
up in the resource database. Of course, finding a resource value is rather complicated,
because names in the database may not be complete resource
names and may contain a mixture of resource names and class symbols.
Nevertheless, during initialization,
CLUE will find the resource binding that is the closest match for each
contact resource. In general, a contact gets its resource values  first
from the user's resource database, then from programmer-specified
defaults when no user value is found.

\begin{center}
\framebox[5.5in]{
\hspace*{\fill}
{\bf Note}
\hspace*{\fill}
\parbox[t]{4.5in}{
The contact programmer decides which values a contact will look up in
the resource database; the user decides what values it will find there.
The contact programmer identifies resources by listing them in the {\tt
:resources} option of the {\tt defcontact} form for his contact class.\index{defcontact}}
\hspace*{\fill}} \end{center}

\subsection{Converting Resource Values} 
\index{resource, conversion}
\index{resource, types}
Another convenient feature of
CLUE resource management is automatic resource type conversion.  Often
the type of value specified by the user in the resource database is not
the data type actually used by the program.  For example, a user might
identify a font by a string such as ``{\tt helvetica},'' but the program
must convert this name into a CLX {\tt font} object before text is
displayed.  Another example is color: where a user might specify ``{\tt
red},'' a program must somehow determine a colormap and pixel value that
will yield this hue.

CLUE automatically converts user values out of the user's resource
database into the correct target data type (as specified in the {\tt
:resources} option of a {\tt defcontact}\index{defcontact} form). This
is done by calling the {\tt convert} function\index{convert}. For
example:

\begin{verbatim}
(setf color (convert         ; Convert a resource...
              a-blinker      ; ...for a blinker...            
              "red"          ; ...from a string value...              
              'pixel))       ; ...to a pixel target type.
\end{verbatim}

CLUE defines a number of {\tt convert} methods for various combinations
of source and target data types. Contact programmers can extend
this mechanism by defining their own special {\tt convert} methods.

\section{Programming Tips\label{sec:tips}}
\subsection*{Managing the Server Connection}
You'll get into trouble if you aren't careful to close down the X server
connection when your program terminates. The reason? Most X servers have a limit
to the number of client connections which they can serve at one time. When the
limit is reached, the server simply refuses to open any new connections, and
you'll end up in the error handler staring at an ``Unable to connect'' message.  Make
sure your program
{\em always} terminates with a call to {\tt close-display},\index{close-display}
even when it aborts unexpectedly. The best way is to put your event loop inside
an {\tt unwind-protect} form:

\begin{verbatim}
...
(unwind-protect 
  (catch :event-loop
    (loop
      (process-next-event display)))  
  (close-display display))
...
\end{verbatim}

One nifty thing about using this technique is that it's always safe to
abort out of a damaged program and start over.

Also, you should generally avoid binding {\em global} special variables
to {\tt contact-display} objects representing open server connections. It's too
easy to lose track of a global variable or, even worse, to garbage a {\tt
contact-display} while it's still open.\index{contact-display}

\subsection*{Debugging}
\index{debugging}
Debugging a CLUE program (or, indeed, any program using the X Window System)
requires an awareness of a simple fact of life: {\bf client-server communication
is buffered}. Just calling the CLX {\tt draw-line} function won't necessarily
cause a line to appear on the screen. Instead, CLX places the corresponding
server request into an output buffer and moves on; the request is sent to the
server and executed later, when the output buffer is flushed. Normally, you
don't have to worry about this because CLX will automatically flush the output
buffer at the ``right'' time\footnotemark\footnotetext{When the output buffer is
full and (by default) before reading the next input event.}. However, during
debugging, you may want requests to be executed immediately. There are
two ways to do this.

\begin{enumerate}
\item Call {\tt display-force-output} manually at the appropriate time.
\index{display-force-output}
\item Invoke {\tt (setf (dis\-play-af\-ter-func\-tion dis\-play)
\#'dis\-play-force-out\-put)}. \\
This tells CLX to flush the
output buffer automatically as soon as each request
is made.
\index{display-after-function} 
\end{enumerate}

\index{errors}
Output buffering also means that when a request is in error, you won't
see the error immediately. The X server reports an error by sending an {\bf
error reply} back to the program (in this sense, an error reply is another kind
of input ``event'').
The error will be reported after the invalid
request is sent to the server, but by this time, your program has usually sent a
number of other requests, too. The result? When you arrive at the error handler,
you almost never be anywhere close to the real scene of the crime.
Using the {\tt display-after-function}\index{display-after-function} for
``single-step'' output, as shown above, can help to mitigate this problem. But
at any rate, you will need to pay careful attention to the contents of the error
reply which is printed out in CLX's error message. Sometimes, your best recourse
is to compare the error
reply with the error behavior defined by the X Protocol Specification for the
offending request and thereby deduce the cause of the bug.

Another useful debugging technique is to (locally) bind your {\tt
contact-display} and important {\tt contact} objects to special variables within
your program. This makes it easier to {\tt break} at a convenient
point (say, while the program is idle, waiting for an event to occur) and then
examine the state of various contacts or make server requests to return
interesting data.

\section{Getting Started\label{sec:getting-started}}
\subsection{Building CLUE}
If you're an Explorer user, you'll want to use an Explorer Release 6 load
band. This band has the latest versions of CLOS, CLX, and CLUE already built in.
If this applies to you, you can stop here --- you're ready to go!

But if your system doesn't come with CLUE built in, you'll need to go through
the following procedures to get things properly loaded.
Three systems need to be loaded: CLOS\index{CLOS}, CLX\index{CLX}, and CLUE
itself. CLX needs to be loaded specially  --- a small patch to the standard X11
R3 version of CLX is needed to make it work with CLUE.

\begin{enumerate}
\item Decide which version of CLOS you want. Possible choices include:

\begin{itemize}
\item {\tt PCL}, the Portable Common Loops implementation from Xerox that has
recently been made compatible with the CLOS specification. See CLUE release
notes for more details.\index{PCL}
\item {\tt clos-kludge}, a simple implementation of a CLOS subset which comes
with
the other CLUE software and which is sufficient for getting started with CLUE.
See the {\tt defsystem} file in the {\tt clos-kludge} directory.\index{clos-kludge}
{\tt clos-kludge} works but beware: you should use this only as a temporary
stop-gap until you have a ``real'' CLOS in place.
\item Something else. Maybe you've done a CLOS implementation yourself! Or maybe
you've gotten one from the vendor of your Common Lisp.
\end{itemize}

\item Compile and load your preferred CLOS.

\begin{center} \framebox[5.5in]{
\hspace*{\fill}
{\bf Note}
\hspace*{\fill}
\parbox[t]{4.5in}{
Make sure that before both CLX and CLUE are {\em compiled}, you have loaded your
preferred version of CLOS. If you switch CLOS versions later, be sure to
recompile both CLX and CLUE.} \hspace*{\fill}} \end{center}


\item Compile and load CLX, with the special CLUE patches with are included with
the other CLUE software.  See the CLUE {\tt defsystem} file.  Lisp machine
users may find it convenient to perform {\tt (make-system 'clx-clos)}.

\item Compile and load CLUE. See the CLUE {\tt defsystem} file.  Lisp machine
users may find it convenient to perform {\tt (make-system 'clue)}

\end{enumerate}

     
  
\subsection{Packages}
\index{packages, CLUE}
\index{packages, CLUEI}\index{intrinsics}
All symbols defined by the CLUE ``intrinsics'' are external in the {\tt
CLUEI} package.  The {\tt CLUE} package exports all symbols defined by the
standard contact set, 
\index{contact, standard set}
in addition to all of the
``intrinsics'' in {\tt CLUEI}\footnotemark\footnotetext{For now, the standard
contact set is undefined so using either the {\tt CLUE} or {\tt CLUEI}
package is effectively equivalent. Similarly, the CLUE specification currently
describes
only the intrinsics and the CLUE package, and does not yet mention the
standard contact set.}.

All symbols defined by CLX\index{CLX} are external in the {\tt XLIB} package. The {\tt
CLUE} and {\tt CLUEI} packages both use {\tt XLIB}.

The package containing CLOS\index{CLOS} symbols may vary, depending on which
implementation of CLOS you are using. Usually, you will want to use the {\tt
CLOS} package which exports the Explorer CLOS system. 

\index{packages, CLOS}\index{packages, LISP}\index{packages, TICL}
\begin{center}
\framebox[5.5in]{
\hspace*{\fill}
{\bf Note}
\hspace*{\fill}
\parbox[t]{4.5in}{
If you are using the Explorer {\tt CLOS} package, be sure to
also use only {\tt LISP}, the standard Common Lisp package, not {\tt TICL}.
Using both {\tt TICL} and {\tt CLOS} together will result in some nasty name
collisions (for example, on {\tt make-instance} and {\tt defmethod}.)}
\hspace*{\fill}} \end{center}


Application programmers should define a separate package for application symbols
which uses either:
\begin{itemize}
\item {\tt LISP}, {\tt CLOS} and {\tt CLUE}, or
\item {\tt LISP}, {\tt CLOS}, {\tt CLUEI}, and a non-standard contact package.
\end{itemize}

Contact programmers should define a separate package for each set of
related contacts, which uses either: 
\begin{itemize}
\item {\tt LISP}, {\tt CLOS}, {\tt XLIB}, and {\tt CLUE}, (if using standard
contacts), or 
\item {\tt LISP}, {\tt CLOS}, {\tt XLIB}, and {\tt CLUEI}.
\end{itemize}

\subsection{Warming Up}
Here are some CLUE calisthenics to limber you up. Some of them
will require further study of CLUE, CLX, and the X Window System. Some of them
are rather challenging.

\begin{enumerate}
\item Complete the {\tt blinker} contact example. Write a {\tt display} method
that will fill the {\tt blinker} with its {\tt color} when its {\tt on-p}
state is
true and fill it with its {\tt background} otherwise. Ensure
that the {\tt blinker-on-p} accessor updates the display correctly.

\item Create two  {\tt blinker} instances --- one that will print out either
``{\tt Off!}'' when
its {\tt :blink} callback is called with a {\tt nil} argument or ``{\tt On!}''
otherwise, and another which will make a funny noise on your Explorer {\em only}
when its {\tt :blink} callback is called with a non-{\tt nil} argument.

\item Try out different contact attributes on your {\tt blinker}. For example,
try a different {\tt border-width}, {\tt border}, and {\tt background}. Try
initializing the {\tt background} to a {\tt '(float 0 1)} value. What happened?
Why?

\item Define resources to change initial {\tt blinker} attributes.

\item Create two ``funny noise'' {\tt blinker} instances, as in the previous
exercise, but give them different funny noises. Do {\em not} change the
definition of the {\tt blinker} class.

\item Change the {\tt blinker} class so that whenever the cursor enters a {\tt
blinker}, it turns into a picture of Gumby. Hint: use the {\tt xfd} program
(font displayer) to find an appropriate element of the {\tt cursor} font.
Another hint: Use ``{\tt man xfd}'' to learn how to use {\tt xfd}.

\item Change the {\tt blinker} class to include a {\tt string} slot, containing
a string which can be displayed in any font. Display the string so that it
appears in the {\tt blinker}'s {\tt color}, is centered horizontally and
vertically within the {\tt blinker}'s current interior size, and is visible
regardless of the {\tt blinker}'s {\tt on-p} state. Ensure that the {\tt string}
remains centered when the user changes the font or the size of the {\tt
blinker}.

\item Define a {\tt composite} class that {\em always} has a single {\tt blinker}
child, which is always centered horizontally and vertically within the {\tt
composite}. Ensure that the {\tt blinker} child
remains centered when the user changes the size of the {\tt composite}.

\item Define an {\tt etch-a-sketch} contact that will:
\begin{itemize}
\item 
Rubberband a horizontal
line as the user moves the pointer with {\tt :button-1} down, then
snap it into place when the button is released.
\item Rubberband a vertical
line as the user moves the pointer with {\tt :button-2} down, then
snap it into place when the button is released.
\item Print out a copy of its current line drawing on the Imagen
when the {\tt \#$\backslash$return} key is pressed.

\end{itemize}

\end{enumerate}
\pagebreak
\begin{thebibliography}{9}

\bibitem{clos} Bobrow, Daniel G., et al. The Common Lisp Object System
Specification (X3J13-88-002). American National Standards Institute, June,
1988. 

{\em [The definition of CLOS. Start with Chapter 1, which is a thorough and
readable explanation of all concepts. Then consult with Chapter 2 for
programming details.]}

\bibitem{clos-guide} Keene, Sonya E. {\em Object Oriented Programming in Common
Lisp}. Addison-Wesley (1989).

{\em [A textbook-style guide to CLOS, written by a member of its ANSI
development team. Discusses all CLOS features, with examples.]}

\bibitem{Xt} McCormack, Joel, et al. The X-Toolkit Intrinsics, Version
11, Revision 2 (March, 1988).

{\em [Xt, the C language counterpart to CLUE. This is important because
CLUE will tend to evolve in the same direction as Xt.]}

\bibitem{x-protocol} Scheifler, Robert W. The X Window System Protocol, Version
11, Revision 3 (December, 1988).

{\em [The X Bible. The definitive word on how X works. A reference manual with
no tutorials or examples.]}

\bibitem{x-window-system} Scheifler, Robert W. and Gettys, Jim. The X Window
System. {\em ACM Transactions on Graphics}, Vol. 5, No. 2 (April 1986).

{\em [An excellent technical overview of the design and features of the X Window
System. Somewhat obsolete (written before X Version 11) but still informative.]} 

\bibitem{clx} Scheifler, Robert W., et al. CLX Interface
Specification, Version 4 (September 1987).

{\em [Terse, but this is the definition of CLX. Describes only the language
binding, so it must be read in conjunction with the X Protocol Specification.]}

\end{thebibliography}


\pagebreak
\begin{theindex}
\input{turbo-clue.index}
\end{theindex}

\end{document}


\end{theindex}

\end{document}


\end{theindex}

\end{document}


\end{theindex}

\end{document}

